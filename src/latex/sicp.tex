\documentclass[a4paper,unicode]{book}
\usepackage[colorlinks]{hyperref}
\usepackage{epigraph}
\usepackage{color}
\usepackage{amsmath}
\usepackage{slatex}
\usepackage{listings}
\usepackage{tikz}
\usetikzlibrary{backgrounds}
%\usepackage{exercise}
\usepackage{amsthm}
\setlength\epigraphwidth{3 in}
\newcommand{\placeholder}[1][8]{\begin{tikzpicture}[red]
    \draw (0,0) -- (#1,#1) (#1,0) -- (0,#1) (0,0) -- (0,#1) -- (#1,#1) -- (#1,0) -- (0,0);
  \end{tikzpicture}}
\theoremstyle{definition}\newtheorem{Exercise}{Exercise}[chapter]
\theoremstyle{plain}\newtheorem{theorem}{Theorem}[chapter]

\definecolor{keyword}{rgb}{0.6,0,0}
\definecolor{variable}{rgb}{0,0,0.6}
\definecolor{constant}{rgb}{0,0.6,0}
\def\keywordfont#1{{\color{keyword}\tt #1}}
\def\variablefont#1{{\color{variable}\tt #1}}
\def\constantfont#1{{\color{constant}\tt #1}}
\let\datafont\constantfont 
\let\schemecodehook\tt

\lstset{language=Lisp}
\makeindex
% A definition that should appear in the index
\newcommand{\slot}[1]{$<\!\!#1\!\!>$}
\newcommand{\idef}[1]{\index{#1}\textit{#1}}
%\newenvironment{epigraph}[1]{\begin{quotation}}{\end{quotation}}
% TODO: print alert on missing file.
\title{Structure and Interpretation of Computer Programs}
\begin{document}
\maketitle
\frontmatter
% TODO: Fill in missing bits from book-Z-H-2
This book is one of a series of texts written by faculty of the
Electrical Engineering and Computer Science Department at the
Massachusetts Institute of Technology.  It was edited and produced by
The MIT Press under a joint production-distribution arrangement with
the McGraw-Hill Book Company.

\textbf{Ordering Information:}
\begin{description}
\item{North America}
  Text orders should be addressed to the McGraw-Hill Book Company.\\
  All other orders should be addressed to The MIT Press.
\item{Outside North America}
  All orders should be addressed to The MIT Press or its local distributor.
\end{description}

% TODO: fix copyright symbol
(c) 1996 by The Massachusetts Institute of Technology\hfill\newline{}
Second edition

All rights reserved.  No part of this book may be reproduced in any
form or by any electronic or mechanical means (including photocopying,
recording, or information storage and retrieval) without permission in
writing from the publisher.

% TODO: insert license here.
This book was set by TQ Hirsch using the \LaTeX{} typesetting system,
based on HTML generated from \LaTeX{} source written by the authors.

% TODO: insert LoC cataloging info


\newpage
This book is dedicated, in respect and admiration, to the spirit that
lives in the computer.

\begin{quotation}
``I think that it's extraordinarily important that we in computer
science keep fun in computing.  When it started out, it was an awful
lot of fun.  Of course, the paying customers got shafted every now and
then, and after a while we began to take their complaints seriously.
We began to feel as if we really were responsible for the successful,
error-free perfect use of these machines.  I don't think we are.  I
think we're responsible for stretching them, setting them off in new
directions, and keeping fun in the house.  I hope the field of
computer science never loses its sense of fun.  Above all, I hope we
don't become missionaries.  Don't feel as if you're Bible salesmen.
The world has too many of those already.  What you know about
computing other people will learn.  Don't feel as if the key to
successful computing is only in your hands.  What's in your hands, I
think and hope, is intelligence: the ability to see the machine as
more than when you were first led up to it, that you can make it
more.''

% TODO: typeset this better
Alan J. Perlis (April 1, 1922-February 7, 1990)
\end{quotation}

% TODO: Make TOC

\chapter{Forward}

Educators, generals, dieticians, psychologists, and parents program.
Armies, students, and some societies are programmed.  An assault on
large problems employs a succession of programs, most of which spring
into existence en route.  These programs are rife with issues that
appear to be particular to the problem at hand.  To appreciate
programming as an intellectual activity in its own right you must turn
to computer programming; you must read and write computer
programs -- many of them.  It doesn't matter much what the programs are
about or what applications they serve.  What does matter is how well
they perform and how smoothly they fit with other programs in the
creation of still greater programs.  The programmer must seek both
perfection of part and adequacy of collection.  In this book the use
of ``program'' is focused on the creation, execution, and study of
programs written in a dialect of Lisp for execution on a digital
computer.  Using Lisp we restrict or limit not what we may program,
but only the notation for our program descriptions.

Our traffic with the subject matter of this book involves us with
three foci of phenomena: the human mind, collections of computer
programs, and the computer.  Every computer program is a model,
hatched in the mind, of a real or mental process.  These processes,
arising from human experience and thought, are huge in number,
intricate in detail, and at any time only partially understood.  They
are modeled to our permanent satisfaction rarely by our computer
programs.  Thus even though our programs are carefully handcrafted
discrete collections of symbols, mosaics of interlocking functions,
they continually evolve: we change them as our perception of the model
deepens, enlarges, generalizes until the model ultimately attains a
metastable place within still another model with which we struggle.
The source of the exhilaration associated with computer programming is
the continual unfolding within the mind and on the computer of
mechanisms expressed as programs and the explosion of perception they
generate.  If art interprets our dreams, the computer executes them in
the guise of programs!

For all its power, the computer is a harsh taskmaster.  Its programs
must be correct, and what we wish to say must be said accurately in
every detail.  As in every other symbolic activity, we become
convinced of program truth through argument.  Lisp itself can be
assigned a semantics (another model, by the way), and if a program's
function can be specified, say, in the predicate calculus, the proof
methods of logic can be used to make an acceptable correctness
argument.  Unfortunately, as programs get large and complicated, as
they almost always do, the adequacy, consistency, and correctness of
the specifications themselves become open to doubt, so that complete
formal arguments of correctness seldom accompany large programs.
Since large programs grow from small ones, it is crucial that we
develop an arsenal of standard program structures of whose correctness
we have become sure -- we call them idioms -- and learn to combine them
into larger structures using organizational techniques of proven
value.  These techniques are treated at length in this book, and
understanding them is essential to participation in the Promethean
enterprise called programming.  More than anything else, the
uncovering and mastery of powerful organizational techniques
accelerates our ability to create large, significant programs.
Conversely, since writing large programs is very taxing, we are
stimulated to invent new methods of reducing the mass of function and
detail to be fitted into large programs.

Unlike programs, computers must obey the laws of physics.  If they
wish to perform rapidly -- a few nanoseconds per state change -- they
must transmit electrons only small distances (at most $1\frac{1}{2}$
feet).  The heat generated by the huge number of devices so
concentrated in space has to be removed.  An exquisite engineering art
has been developed balancing between multiplicity of function and
density of devices.  In any event, hardware always operates at a level
more primitive than that at which we care to program.  The processes
that transform our Lisp programs to ``machine'' programs are
themselves abstract models which we program.  Their study and creation
give a great deal of insight into the organizational programs
associated with programming arbitrary models.  Of course the computer
itself can be so modeled.  Think of it: the behavior of the smallest
physical switching element is modeled by quantum mechanics described
by differential equations whose detailed behavior is captured by
numerical approximations represented in computer programs executing on
computers composed of\ldots{}!

It is not merely a matter of tactical convenience to separately
identify the three foci.  Even though, as they say, it's all in the
head, this logical separation induces an acceleration of symbolic
traffic between these foci whose richness, vitality, and potential is
exceeded in human experience only by the evolution of life itself.  At
best, relationships between the foci are metastable.  The computers
are never large enough or fast enough.  Each breakthrough in hardware
technology leads to more massive programming enterprises, new
organizational principles, and an enrichment of abstract models.
Every reader should ask himself periodically ``Toward what end, toward
what end?'' -- but do not ask it too often lest you pass up the fun of
programming for the constipation of bittersweet philosophy.

Among the programs we write, some (but never enough) perform a precise
mathematical function such as sorting or finding the maximum of a
sequence of numbers, determining primality, or finding the square
root.  We call such programs algorithms, and a great deal is known of
their optimal behavior, particularly with respect to the two important
parameters of execution time and data storage requirements.  A
programmer should acquire good algorithms and idioms.  Even though
some programs resist precise specifications, it is the responsibility
of the programmer to estimate, and always to attempt to improve, their
performance.

Lisp is a survivor, having been in use for about a quarter of a
century.  Among the active programming languages only Fortran has had
a longer life.  Both languages have supported the programming needs of
important areas of application, Fortran for scientific and engineering
computation and Lisp for artificial intelligence.  These two areas
continue to be important, and their programmers are so devoted to
these two languages that Lisp and Fortran may well continue in active
use for at least another quarter-century.

Lisp changes.  The Scheme dialect used in this text has evolved from
the original Lisp and differs from the latter in several important
ways, including static scoping for variable binding and permitting
functions to yield functions as values.  In its semantic structure
Scheme is as closely akin to Algol 60 as to early Lisps.  Algol 60,
never to be an active language again, lives on in the genes of Scheme
and Pascal.  It would be difficult to find two languages that are the
communicating coin of two more different cultures than those gathered
around these two languages.  Pascal is for building
pyramids -- imposing, breathtaking, static structures built by armies
pushing heavy blocks into place.  Lisp is for building
organisms -- imposing, breathtaking, dynamic structures built by squads
fitting fluctuating myriads of simpler organisms into place.  The
organizing principles used are the same in both cases, except for one
extraordinarily important difference: The discretionary exportable
functionality entrusted to the individual Lisp programmer is more than
an order of magnitude greater than that to be found within Pascal
enterprises.  Lisp programs inflate libraries with functions whose
utility transcends the application that produced them.  The list,
Lisp's native data structure, is largely responsible for such growth
of utility.  The simple structure and natural applicability of lists
are reflected in functions that are amazingly nonidiosyncratic.  In
Pascal the plethora of declarable data structures induces a
specialization within functions that inhibits and penalizes casual
cooperation.  It is better to have 100 functions operate on one data
structure than to have 10 functions operate on 10 data structures.  As
a result the pyramid must stand unchanged for a millennium; the
organism must evolve or perish.

To illustrate this difference, compare the treatment of material and
exercises within this book with that in any first-course text using
Pascal.  Do not labor under the illusion that this is a text
digestible at MIT only, peculiar to the breed found there.  It is
precisely what a serious book on programming Lisp must be, no matter
who the student is or where it is used.

Note that this is a text about programming, unlike most Lisp books,
which are used as a preparation for work in artificial intelligence.
After all, the critical programming concerns of software engineering
and artificial intelligence tend to coalesce as the systems under
investigation become larger.  This explains why there is such growing
interest in Lisp outside of artificial intelligence.

As one would expect from its goals, artificial intelligence research
generates many significant programming problems.  In other
programming cultures this spate of problems spawns new languages.
Indeed, in any very large programming task a useful organizing
principle is to control and isolate traffic within the task modules
via the invention of language.  These languages tend to become less
primitive as one approaches the boundaries of the system where we
humans interact most often.  As a result, such systems contain complex
language-processing functions replicated many times.  Lisp has such a
simple syntax and semantics that parsing can be treated as an
elementary task.  Thus parsing technology plays almost no role in Lisp
programs, and the construction of language processors is rarely an
impediment to the rate of growth and change of large Lisp systems.
Finally, it is this very simplicity of syntax and semantics that is
responsible for the burden and freedom borne by all Lisp programmers.
No Lisp program of any size beyond a few lines can be written without
being saturated with discretionary functions.  Invent and fit; have
fits and reinvent!  We toast the Lisp programmer who pens his thoughts
within nests of parentheses.

%TODO: Typeset this better
Alan J. Perlis\\
New Haven, Connecticut

\chapter{Preface to the Second Edition}

\epigraph{
Is it possible that software is not like anything else, that it
is meant to be discarded: that the whole point is to 
always see it as a soap bubble?}{Alan J. Perlis}

The material in this book has been the basis of MIT's entry-level
computer science subject since 1980.  We had been teaching this
material for four years when the first edition was published, and
twelve more years have elapsed until the appearance of this second
edition.  We are pleased that our work has been widely adopted and
incorporated into other texts.  We have seen our students take the
ideas and programs in this book and build them in as the core of new
computer systems and languages.  In literal realization of an ancient
Talmudic pun, our students have become our builders.  We are lucky to
have such capable students and such accomplished builders.

In preparing this edition, we have incorporated hundreds of 
clarifications suggested by our own teaching experience and the
comments of colleagues at MIT and elsewhere.  We have redesigned
most of the major programming systems in the book, including
the generic-arithmetic system, the interpreters, the register-machine
simulator, and the compiler; and we have rewritten all the program
examples to ensure that any Scheme implementation conforming to
the IEEE Scheme standard (IEEE 1990) will be able to run the code.

This edition emphasizes several new themes.  The most important
of these is the central role played by different approaches to
dealing with time in computational models: objects with state,
concurrent programming, functional programming, lazy evaluation,
and nondeterministic programming.  We have included new sections on 
concurrency and nondeterminism, and we have tried to integrate this
theme throughout the book.

The first edition of the book closely followed the syllabus of our MIT
one-semester subject.  With all the new material in the second
edition, it will not be possible to cover everything in a single
semester, so the instructor will have to pick and choose.  In our own
teaching, we sometimes skip the section on logic programming (section
\ref{sec-4.4}), we have students use the register-machine simulator
but we do not cover its implementation (section \ref{sec-5.2}), and we
give only a cursory overview of the compiler (section \ref{sec-5.5}).
Even so, this is still an intense course.  Some instructors may wish
to cover only the first three or four chapters, leaving the other
material for subsequent courses.

The World-Wide-Web site www-mitpress.mit.edu/sicp % TODO: Hyperlink this
provides support for users of this book.
This includes programs from the book,
sample programming assignments, supplementary materials,
and downloadable implementations of the Scheme dialect of Lisp.


\chapter{Preface to the First Edition}
\begin{epigraph}{Marvin Minsky}
  A computer is like a violin.  You can imagine a novice trying first a
phonograph and then a violin.  The latter, he says, sounds terrible.
That is the argument we have heard from our humanists and most of our
computer scientists.  Computer programs are good, they say, for
particular purposes, but they aren't flexible.  Neither is a violin,
or a typewriter, until you learn how to use it.

--Marvin Minsky, ``Why Programming Is a Good Medium for Expressing
Poorly-Understood and Sloppily-Formulated Ideas''
\end{epigraph}

``The Structure and Interpretation of Computer Programs'' is the
entry-level subject in computer science at the Massachusetts Institute
of Technology.  It is required of all students at MIT who major
in electrical engineering or in computer science, as one-fourth of the
``common core curriculum,'' which also includes two subjects on
circuits and linear systems and a subject on the design of digital
systems.  We have been involved in the development of this subject
since 1978, and we have taught this material in its present form since
the fall of 1980 to between 600 and 700 students each year.  Most of
these students have had little or no prior formal training in
computation, although many have played with computers a bit and a few
have had extensive programming or hardware-design experience.

Our design of this introductory computer-science subject reflects two
major concerns.  First, we want to establish the idea that a computer
language is not just a way of getting a computer to perform operations
but rather that it is a novel formal medium for expressing ideas about
methodology.  Thus, programs must be written for people to read, and
only incidentally for machines to execute.  Second, we believe that
the essential material to be addressed by a subject at this level is
not the syntax of particular programming-language constructs, nor
clever algorithms for computing particular functions efficiently, nor
even the mathematical analysis of algorithms and the foundations of
computing, but rather the techniques used to control the intellectual
complexity of large software systems.

Our goal is that students who complete this subject should have a good
feel for the elements of style and the aesthetics of programming.
They should have command of the major techniques for controlling
complexity in a large system. They should be capable of reading a
50-page-long program, if it is written in an exemplary style. They
should know what not to read, and what they need not understand at any
moment.  They should feel secure about modifying a program, retaining
the spirit and style of the original author.
These skills are by no means unique to computer programming.  The
techniques we teach and draw upon are common to all of engineering
design.  We control complexity by building abstractions that hide
details when appropriate.  We control complexity by establishing
conventional interfaces that enable us to construct systems by
combining standard, well-understood pieces in a ``mix and match'' way.
We control complexity by establishing new languages for describing a
design, each of which emphasizes particular aspects of the design and
deemphasizes others.

Underlying our approach to this subject is our conviction that
``computer science'' is not a science and that its significance has
little to do with computers.  The computer revolution is a revolution
in the way we think and in the way we express what we think.  The
essence of this change is the emergence of what might best be called
\textit{procedural epistemology} -- the study of the structure of
knowledge from an imperative point of view, as opposed to the more
declarative point of view taken by classical mathematical subjects.
Mathematics provides a framework for dealing precisely with notions of
``what is.''  Computation provides a framework for dealing precisely
with notions of ``how to.''

In teaching our material we use a dialect of the programming language
Lisp.  We never formally teach the language, because we don't have to.
We just use it, and students pick it up in a few days.  This is one
great advantage of Lisp-like languages: They have very few ways of
forming compound expressions, and almost no syntactic structure.  All
of the formal properties can be covered in an hour, like the rules of
chess.  After a short time we forget about syntactic details of the
language (because there are none) and get on with the real
issues -- figuring out what we want to compute, how we will decompose
problems into manageable parts, and how we will work on the parts.
Another advantage of Lisp is that it supports (but does not enforce)
more of the large-scale strategies for modular decomposition of
programs than any other language we know.  We can make procedural and
data abstractions, we can use higher-order functions to capture common
patterns of usage, we can model local state using assignment and data
mutation, we can link parts of a program with streams and delayed
evaluation, and we can easily implement embedded languages.  All of
this is embedded in an interactive environment with excellent support
for incremental program design, construction, testing, and debugging.
We thank all the generations of Lisp wizards, starting with John
McCarthy, who have fashioned a fine tool of unprecedented power and
elegance.

Scheme, the dialect of Lisp that we use, is an attempt to bring
together the power and elegance of Lisp and Algol.  From Lisp we take
the metalinguistic power that derives from the simple syntax, the
uniform representation of programs as data objects, and the
garbage-collected heap-allocated data.  From Algol we take lexical
scoping and block structure, which are gifts from the pioneers of
programming-language design who were on the Algol committee.  We wish
to cite John Reynolds and Peter Landin for their insights into the
relationship of Church's lambda calculus to the structure of
programming languages.  We also recognize our debt to the
mathematicians who scouted out this territory decades before computers
appeared on the scene.  These pioneers include Alonzo Church, Barkley
Rosser, Stephen Kleene, and Haskell Curry.

\chapter{Acknowledgments}
We would like to thank the many people who have helped us develop this
book and this curriculum.

Our subject is a clear intellectual descendant of ``6.231,'' a
wonderful subject on programming linguistics and the lambda calculus
taught at MIT in the late 1960s by Jack Wozencraft and Arthur Evans,
Jr.

We owe a great debt to Robert Fano, who reorganized MIT's introductory
curriculum in electrical engineering and computer science to emphasize
the principles of engineering design.  He led us in starting out on
this enterprise and wrote the first set of subject notes from which
this book evolved.

Much of the style and aesthetics of programming that we try to teach
were developed in conjunction with Guy Lewis Steele Jr., who
collaborated with Gerald Jay Sussman in the initial development of the
Scheme language.  In addition, David Turner, Peter Henderson, Dan
Friedman, David Wise, and Will Clinger have taught us many of the
techniques of the functional programming community that appear in this
book.

Joel Moses taught us about structuring large systems.  His experience
with the Macsyma system for symbolic computation provided the insight
that one should avoid complexities of control and concentrate on
organizing the data to reflect the real structure of the world being
modeled.

Marvin Minsky and Seymour Papert formed many of our attitudes about
programming and its place in our intellectual lives.  To them we owe
the understanding that computation provides a means of expression for
exploring ideas that would otherwise be too complex to deal with
precisely.  They emphasize that a student's ability to write and
modify programs provides a powerful medium in which exploring becomes
a natural activity.

We also strongly agree with Alan Perlis that programming is lots of
fun and we had better be careful to support the joy of programming.
Part of this joy derives from observing great masters at work.  We are
fortunate to have been apprentice programmers at the feet of Bill
Gosper and Richard Greenblatt.

It is difficult to identify all the people who have contributed to the
development of our curriculum.  We thank all the lecturers, recitation
instructors, and tutors who have worked with us over the past fifteen
years and put in many extra hours on our subject, especially Bill
Siebert, Albert Meyer, Joe Stoy, Randy Davis, Louis Braida, Eric
Grimson, Rod Brooks, Lynn Stein, and Peter Szolovits.
We would like to specially acknowledge the outstanding teaching
contributions of Franklyn Turbak, now at Wellesley; his work
in undergraduate instruction set a standard that we can
all aspire to.
We are grateful to Jerry Saltzer and Jim Miller for
helping us grapple with the mysteries of concurrency, and to
Peter Szolovits and David McAllester for their contributions
to the exposition of nondeterministic evaluation in chapter 4.

Many people have put in significant effort presenting this material at
other universities.  Some of the people we have worked closely with
are Jacob Katzenelson at the Technion, Hardy Mayer at the University
of California at Irvine, Joe Stoy at Oxford, Elisha Sacks at Purdue,
and Jan Komorowski at the Norwegian University of Science and
Technology.  We are exceptionally proud of our colleagues who have
received major teaching awards for their adaptations of this subject
at other universities, including Kenneth Yip at Yale, Brian Harvey at
the University of California at Berkeley, and Dan Huttenlocher at
Cornell.

% TODO: Fix accent
Al Moy\'e arranged for us to teach this material to engineers at
Hewlett-Packard, and for the production of videotapes of these
lectures.
We would like to thank the talented instructors -- in
particular Jim Miller, Bill Siebert, and Mike Eisenberg -- who have
designed continuing education courses incorporating these tapes and
taught them at universities and industry all over the world.

Many educators in other countries have put in significant
work translating the first edition.
Michel Briand, Pierre Chamard, and Andr\'e Pic produced a French edition;
Susanne Daniels-Herold produced a German
edition; and Fumio Motoyoshi produced a Japanese edition.
We do not know who produced the Chinese edition,
but we consider it an honor to have been selected as the
subject of an ``unauthorized'' translation.

It is hard to enumerate all the people who have made technical
contributions to the development of the Scheme systems we use for
instructional purposes.  In addition to Guy Steele, principal wizards
have included Chris Hanson, Joe Bowbeer, Jim Miller, Guillermo Rozas,
and Stephen Adams.  Others who have put in significant time are
Richard Stallman, Alan Bawden, Kent Pitman, Jon Taft, Neil Mayle, John
Lamping, Gwyn Osnos, Tracy Larrabee, George Carrette, Soma
Chaudhuri, Bill Chiarchiaro, Steven Kirsch, Leigh Klotz, Wayne Noss,
Todd Cass, Patrick O'Donnell, Kevin Theobald, Daniel Weise, Kenneth
Sinclair, Anthony Courtemanche, Henry M. Wu, Andrew Berlin, and Ruth
Shyu.

Beyond the MIT implementation, we would like to thank the many people
who worked on the IEEE Scheme standard, including William Clinger and
Jonathan Rees, who edited the $\mathrm{R^4RS}$, and Chris Haynes, David
Bartley, Chris Hanson, and Jim Miller, who prepared the IEEE standard.

Dan Friedman has been a long-time leader of the Scheme community.
The community's broader work goes beyond issues of language design to
encompass significant educational innovations, such as the high-school
curriculum based on EdScheme by Schemer's Inc., and the wonderful
books by Mike Eisenberg and by Brian Harvey and Matthew Wright.

We appreciate the work of those who contributed to making this a real
book, especially Terry Ehling, Larry Cohen, and Paul Bethge at the MIT
Press.  Ella Mazel found the wonderful cover image.  For the second
edition we are particularly grateful to Bernard and Ella Mazel for
help with the book design, and to David Jones, \TeX{} wizard
extraordinaire.  We also are indebted to those readers who made
penetrating comments on the new draft: Jacob Katzenelson, Hardy
Mayer, Jim Miller, and especially Brian Harvey, who did unto this book
as Julie did unto his book \textit{Simply Scheme}.

Finally, we would like to acknowledge the support of the organizations
that have encouraged this work over the years, including support from
Hewlett-Packard, made possible by Ira Goldstein and Joel Birnbaum, and
support from DARPA, made possible by Bob Kahn.

\mainmatter
\chapter{Building abstractions with Procedures}
\label{chap-1}
\begin{epigraph}{John Locke}
The acts of the mind, wherein it exerts its power over simple ideas,
are chiefly these three: 1. Combining several simple ideas into one
compound one, and thus all complex ideas are made.  2. The second is
bringing two ideas, whether simple or complex, together, and setting
them by one another so as to take a view of them at once, without
uniting them into one, by which it gets all its ideas of relations.
3.  The third is separating them from all other ideas that accompany
them in their real existence: this is called abstraction, and thus all
its general ideas are made.

--John Locke, \textit{An Essay Concerning Human Understanding}
(1690)
\end{epigraph}

We are about to study the idea of a \idef{computational process}.
Computational processes are abstract beings that inhabit computers.
As they evolve, processes manipulate other abstract things called
\idef{data}.  The evolution of a process is directed by a pattern of
rules called a \idef{program}.  People create programs to direct
processes.  In effect, we conjure the spirits of the computer with our
spells.

A computational process is indeed much like a sorcerer's idea of a
spirit.  It cannot be seen or touched.  It is not composed of matter
at all.  However, it is very real.  It can perform intellectual work.
It can answer questions.  It can affect the world by disbursing money
at a bank or by controlling a robot arm in a factory.  The programs we
use to conjure processes are like a sorcerer's spells.  They are
carefully composed from symbolic expressions in arcane and esoteric
\idef{programming languages} that prescribe the tasks we want our
processes to perform.

A computational process, in a correctly working computer, executes
programs precisely and accurately.  Thus, like the sorcerer's
apprentice, novice programmers must learn to understand and to
anticipate the consequences of their conjuring.  Even small errors
(usually called \idef{bugs} or \idef{glitches}) in programs can have
complex and unanticipated consequences.

Fortunately, learning to program is considerably less dangerous than
learning sorcery, because the spirits we deal with are conveniently
contained in a secure way.  Real-world programming, however,
requires care, expertise, and wisdom.  A small bug in a computer-aided
design program, for example, can lead to the catastrophic collapse of
an airplane or a dam or the self-destruction of an industrial robot.

Master software engineers have the ability to organize programs so
that they can be reasonably sure that the resulting processes will
perform the tasks intended.  They can visualize the behavior of their
systems in advance.  They know how to structure programs so that
unanticipated problems do not lead to catastrophic consequences, and
when problems do arise, they can \idef{debug} their programs.
Well-designed computational systems, like well-designed automobiles or
nuclear reactors, are designed in a modular manner, so that the parts
can be constructed, replaced, and debugged separately.

\section*{Programming in Lisp}

We need an appropriate language for describing processes, and we will
use for this purpose the programming language Lisp.  Just as our
everyday thoughts are usually expressed in our natural language (such
as English, French, or Japanese), and descriptions of quantitative
phenomena are expressed with mathematical notations, our procedural
thoughts will be expressed in Lisp.  \index{Lisp}Lisp was invented in
the late 1950s as a formalism for reasoning about the use of certain
kinds of logical expressions, called \idef{recursion equations}, as a
model for computation.  The language was conceived by \index{John
  McCarthy}John McCarthy and is based on his paper ``Recursive
Functions of Symbolic Expressions and Their Computation by Machine''
(McCarthy 1960).

Despite its inception as a mathematical formalism, Lisp is a practical
programming language.  A Lisp \idef{interpreter} is a machine that
carries out processes described in the Lisp language.  The first Lisp
interpreter was implemented by \index{John McCarthy}McCarthy with the
help of colleagues and students in the Artificial Intelligence Group
of the MIT Research Laboratory of Electronics and in the MIT
Computation Center.\footnote{The \textit{Lisp 1 Programmer's Manual}
  appeared in 1960, and the \textit{Lisp 1.5 Programmer's Manual}
  (McCarthy 1965) was published in 1962.  The early history of Lisp is
  described in McCarthy 1978} \index{Lisp}Lisp, whose name is an
acronym for LISt Processing, was designed to provide
symbol-manipulating capabilities for attacking programming problems
such as the symbolic differentiation and integration of algebraic
expressions.  It included for this purpose new data objects known as
atoms and lists, which most strikingly set it apart from all other
languages of the period.

Lisp was not the product of a concerted design effort.  Instead, it
evolved informally in an experimental manner in response to users'
needs and to pragmatic implementation considerations.  Lisp's informal
evolution has continued through the years, and the community of Lisp
users has traditionally resisted attempts to promulgate any
``official'' idefnition of the language.  This evolution, together
with the flexibility and elegance of the initial conception, has
enabled Lisp, which is the second oldest language in widespread use
today (only \index{Fortran}Fortran is older), to continually adapt to
encompass the most modern ideas about program design.  Thus, Lisp is
by now a family of dialects, which, while sharing most of the original
features, may differ from one another in significant ways.  The
dialect of Lisp used in this book is called
\index{Scheme}Scheme.\footnote{The two dialects in which most major
  Lisp programs of the 1970s were written are \index{MacLisp}MacLisp
  (Moon 1978; Pitman 1983), developed at the MIT Project MAC, and
  \index{Interlisp}Interlisp (Teitelman 1974), developed at Bolt
  Beranek and Newman Inc. and the Xerox Palo Alto Research Center.
  \index{Portable Standard Lisp}Portable Standard Lisp (Griss 1981)
  was a Lisp dialect designed to be easily portable between different
  machines. MacLisp spawned a number of subdialects, such as
  \index{Franz Lisp}Franz Lisp, whch was developed at the University
  of California at Berkeley, and \index{Zetalisp} (Moon 1981) which
  was based on a special-purpose processor designed at the \index{MIT
    Artificial Intelligence Laboratory}MIT Artificial Intelligence
  Laboratory to run Lisp very efficiently. The Lisp dialect used in
  this book, called \index{Scheme} (Steele 1975) was invented in 1975
  by Guy Lewis Steele, Jr. and Gerald Jay Sussman of the MIT
  Artificial Intelligence Laboratory and later reimplemented for
  instructional use at MIT. Scheme became and IEEE standard in 1990
  (IEEE 1990). The Common Lisp dialect (Steele 1982, Steele 1990) was
  developed by the Lisp community to combine features from the earlier
  Lisp dialects to make an industrial standard for Lisp.  Common Lisp
  became an ANSI standard in 1994 (ANSI 1994).}

Because of its experimental character and its emphasis on symbol
manipulation, \index{Lisp}Lisp was at first very inefficient for
numerical computations, at least in comparison with Fortran.  Over the
years, however, Lisp compilers have been developed that translate
programs into machine code that can perform numerical computations
reasonably efficiently.  And for special applications, Lisp has been
used with great effectiveness.\footnote{One such special application
  was a breakthrough computation of scientific importance -- an
  integration of the motion of the Solar System that extended
  previous results by nearly 2 orders of magnitude, and demonstrated
  that the dynamics of the Solar System is chaotic.  This computation
  was made possible by new integration algorithms, a special-purpose
  compiler, and a special-purpose computer, all implemented with the
  aid of software tools written in Lisp (Abelson et al. 1992; Sussman
  and Wisdom 1992).} Although Lisp has not yet overcome its old
reputation as hopelessly inefficient, Lisp is now used in many
applications where efficiency is not the central concern.  For
example, Lisp has become a language of choice for operating-system
shell languages and for extension languages for editors and
computer-aided design systems.

If Lisp is not a mainstream language, why are we using it as the
framework for our discussion of programming?  Because the language
possesses unique features that make it an excellent medium for
studying important programming constructs and data structures and for
relating them to the linguistic features that support them.  The most
significant of these features is the fact that Lisp descriptions of
processes, called \idef{procedures}, can themselves be represented and
manipulated as Lisp data.  The importance of this is that there are
powerful program-design techniques that rely on the ability to blur
the traditional distinction between ``passive'' data and ``active''
processes.  As we shall discover, Lisp's flexibility in handling
procedures as data makes it one of the most convenient languages in
existence for exploring these techniques.  The ability to represent
procedures as data also makes Lisp an excellent language for writing
programs that must manipulate other programs as data, such as the
interpreters and compilers that support computer languages.  Above and
beyond these considerations, programming in Lisp is great fun.

\section{The Elements of Programming}
\label{sec:1.1}

A powerful programming language is more than just a means for
instructing a computer to perform tasks.  The language also serves as
a framework within which we organize our ideas about processes.  Thus,
when we describe a language, we should pay particular attention to the
means that the language provides for combining simple ideas to form
more complex ideas.  Every powerful language has three mechanisms for
accomplishing this:

\begin{description}
\item[primitive expressions], which represent the simplest entities
  the language is concerned with,
\item[means of combination], by which compound elements are built from
  simpler ones, and
\item[means of abstraction], by which compound elements can be named
  and manipulated as units.
\end{description}


In programming, we deal with two kinds of elements: \idef{procedures}
and \idef{data}. (Later we will discover that they are really not so
distinct.)  Informally, data is ``stuff'' that we want to manipulate,
and procedures are descriptions of the rules for manipulating the
data.  Thus, any powerful programming language should be able to
describe primitive data and primitive procedures and should have
methods for combining and abstracting procedures and data.

In this chapter we will deal only with simple numerical data so that
we can focus on the rules for building procedures.\footnote{The
  characterization of numbers as ``simple data'' is a barefaced bluff.
  In fact, the treatment of numbers is one of the trickiest and most
  confusing aspects of any programming language.  Some typical issues
  involved are these: Some computer systems distinguish
  \textit{integers}, such as 2, from \textit{real numbers}, such as
  2.71.  Is the real number 2.00 different from the integer 2?  Are
  the arithmetic operations used for integers the same as the
  operations used for real numbers?  Does 6 divided by 2 produce 3, or
  3.0?  How large a number can we represent?  How many decimal places
  of accuracy can we represent?  Is the range of integers the same as
  the range of real numbers?  Above and beyond these questions, of
  course, lies a collection of issues concerning roundoff and
  truncation errors -- the entire science of numerical analysis.
  Since our focus in this book is on large-scale program design rather
  than on numerical techniques, we are going to ignore these problems.
  The numerical examples in this chapter will exhibit the usual
  roundoff behavior that one observes when using arithmetic operations
  that preserve a limited number of decimal places of accuracy in
  noninteger operations.} In later chapters we will see that these
same rules allow us to build procedures to manipulate compound data as
well.

\subsection{Expressions}
\label{sec:1.1.1}

One easy way to get started at programming is to examine some typical
interactions with an interpreter for the Scheme dialect of Lisp.
Imagine that you are sitting at a computer terminal.  You type an
\idef{expression}, and the interpreter responds by displaying the result of
its \idef{evaluating} that expression.

One kind of primitive expression you might type is a number.  (More
precisely, the expression that you type consists of the numerals that
represent the number in base 10.)  If you present Lisp with
a number

\texttt{486}

\noindent
the interpreter will respond by printing\footnote{Throughout this
  book, when we wish to emphasize the distinction between the input
  typed by the user and the response printed by the interpreter, we
  will show the latter in slanted characterization}\relax

\texttt{\textit{486}}

\noindent Expressions representing numbers may be combined with an
expression representing a primitive procedure (such as \texttt{+} or
\texttt{*}) to form a compound expression that represents the
application of the procedure to those numbers.  For example:

\begin{verbatim}
% TODO: Convert to a typescript environment
> (+ 137 349)
486
> (- 1000 334)
666
> (* 5 99)
495
> (/ 10 5)
2
> (+ 2.7 10)
12.7
\end{verbatim}

Expressions such as these, formed by delimiting a list of expressions
within parentheses in order to denote procedure application, are
called combinations.  The leftmost element in the list is called the
\idef{operator}, and the other elements are called \idef{operands}.
The value of a combination is obtained by applying the procedure
specified by the operator to the \idef{arguments} that are the values of the
operands.

The convention of placing the operator to the left of the operands is
known as \idef{prefix notation}, and it may be somewhat confusing at
first because it departs significantly from the customary mathematical
convention.  Prefix notation has several advantages, however.  One of
them is that it can accommodate procedures that may take an arbitrary
number of arguments, as in the following examples:

\begin{verbatim}
% TODO: Convert to a typescript environment
> (+ 21 35 12 7)
75
> (* 25 4 12)
1200
\end{verbatim}

No ambiguity can arise, because the operator is always the leftmost
element and the entire combination is delimited by the parentheses.

A second advantage of prefix notation is that it extends in a
straightforward way to allow combinations to be \textbf{nested}, that
is, to have combinations whose elements are themselves
combinations:

\begin{verbatim}
% TODO: Convert to a typescript environment
> (+ (* 3 5) (- 10 6))
19
\end{verbatim}

There is no limit (in principle) to the depth of such nesting and to
the overall complexity of the expressions that the Lisp interpreter
can evaluate.  It is we humans who get confused by still relatively
simple expressions such as

\begin{verbatim}
> (+ (* 3 (+ (* 2 4) (+ 3 5))) (+ (- 10 7) 6))
\end{verbatim}

\noindent which the interpreter would readily evaluate to be 57.  We can help
ourselves by writing such an expression in the form

\begin{schemedisplay}
(+ (* 3
      (+ (* 2 4)
         (+ 3 5)))
   (+ (- 10 7)
      6))
\end{schemedisplay}

\noindent following a formatting convention known as
\idef{pretty-printing}, in which each long combination is written so
that the operands are aligned vertically.  The resulting indentations
display clearly the structure of the expression. \footnote{Lisp
  systems typically provide features to aid the user in formatting
  expressions.  Two especially useful features are one that
  automatically indents to the proper pretty-print position whenever a
  new line is started and one that highlights the matching left
  parenthesis whenever a right parenthesis is typed.}

Even with complex expressions, the interpreter always operates in the
same basic cycle: It reads an expression from the terminal, evaluates
the expression, and prints the result.  This mode of operation is
often expressed by saying that the interpreter runs in a
\idef{read-eval-print loop}.  Observe in particular that it is not
necessary to explicitly instruct the interpreter to print the value of
the expression.\footnote{Lisp obeys the convention that every
  expression has a value. This convention, together with the old
  reputation of Lisp as an inefficient language, is the source of the
  quip by Alan Perlis (paraphrasing Oscar Wilde) that ``Lisp
  programmers know the value of everything but the cost of nothing.''}

\subsection{Naming and the Environment}
\label{sec:1.1.2}

A critical aspect of a programming language is the means it provides
for using names to refer to computational objects.  We say that the
name identifies a \idef{variable} whose \idef{value} is the object.

In the Scheme dialect of Lisp, we name things with \idef{define}.
Typing

\begin{verbatim}
> (define size 2)
\end{verbatim}

\noindent causes the interpreter to associate the value 2 with the
name \texttt{size}.\footnote{In this book, we do not show the
  interpreter's response to evaluating definitions, since this is
  highly implementation-dependent.}  Once the name \texttt{size} has
been associated with the number 2, we can refer to the value 2 by
name:

\begin{verbatim}
> size
2
> (* 5 size)
10
\end{verbatim}

Here are further examples of the use of \texttt{define}:

\begin{verbatim}
> (define pi 3.14159)
> (define radius 10)
> (* pi (* radius radius))
314.159
> (define circumference (* 2 pi radius))
> circumference
62.8318
\end{verbatim}

\texttt{Define} is our language's simplest means of abstraction, for
it allows us to use simple names to refer to the results of compound
operations, such as the \texttt{circumference} computed above.  In
general, computational objects may have very complex structures, and
it would be extremely inconvenient to have to remember and repeat
their details each time we want to use them.  Indeed, complex programs
are constructed by building, step by step, computational objects of
increasing complexity. The interpreter makes this step-by-step program
construction particularly convenient because name-object associations
can be created incrementally in successive interactions.  This feature
encourages the incremental development and testing of programs and is
largely responsible for the fact that a Lisp program usually consists
of a large number of relatively simple procedures.

It should be clear that the possibility of associating values with
symbols and later retrieving them means that the interpreter must
maintain some sort of memory that keeps track of the name-object
pairs.  This memory is called the \idef{environment} (more precisely
the \idef{global environment}, since we will see later that a
computation may involve a number of different environments).\footnote{
  \ref{chap:3} will show that this notion of environment is crucial,
  both for understanding how the interpreter works and for
  implementing interpreters.}

\subsection{Evaluating Combinations}
\label{sec:1.1.3}

One of our goals in this chapter is to isolate issues about thinking
procedurally.  As a case in point, let us consider that, in evaluating
combinations, the interpreter is itself following a procedure.

% TODO: Improve this formatting.
To evaluate a combination, do the following:

\begin{enumerate}
\item Evaluate the subexpressions of the combination.<p>

\item Apply the procedure that is the value of the leftmost 
subexpression (the operator) to the arguments that are the values of
the other subexpressions (the operands).
\end{enumerate}

Even this simple rule illustrates some important points about
processes in general.  First, observe that the first step dictates
that in order to accomplish the evaluation process for a combination
we must first perform the evaluation process on each element of the
combination.  Thus, the evaluation rule is \idef{recursive} in nature;
that is, it includes, as one of its steps, the need to invoke the rule
itself.\footnote{ It may seem strange that the evaluation rule says,
  as part of the first step, that we should evaluate the leftmost
  element of a combination, since at this point that can only be an
  operator such as \texttt{+} or \texttt{*} representing a built-in
  primitive procedure such as addition or multiplication.  We will see
  later that it is useful to be able to work with combinations whose
  operators are themselves compound expressions.}

Notice how succinctly the idea of recursion can be used to express
what, in the case of a deeply nested combination, would otherwise be
viewed as a rather complicated process.  For example, evaluating

\begin{schemedisplay}
(* (+ 2 (* 4 6))
   (+ 3 5 7))
\end{schemedisplay}

\noindent requires that the evaluation rule be applied to four
different combinations.  We can obtain a picture of this process by
representing the combination in the form of a tree, as shown in figure
% TODO: debug why this shows up as 1.3
\ref{fig:1.1}.  Each combination is represented by a node with
branches corresponding to the operator and the operands of the
combination stemming from it.  The terminal nodes (that is, nodes with
no branches stemming from them) represent either operators or numbers.
Viewing evaluation in terms of the tree, we can imagine that the
values of the operands percolate upward, starting from the terminal
nodes and then combining at higher and higher levels.  In general, we
shall see that recursion is a very powerful technique for dealing with
hierarchical, treelike objects.  In fact, the ``percolate values
upward'' form of the evaluation rule is an example of a general kind
of process known as \idef{tree accumulation}.

\begin{figure}
\begin{tikzpicture}
  % TODO: Improve this diagram
  [level 1/.style={sibling distance=3em},
   level 2/.style={sibling distance=1.5em}]
  \node {390}
    child { node {*} }
    child { node {26}
      child { node {+} }
      child { node {2} }
      child { node {24}
        child { node {*} }
        child { node {4} }
        child { node {6} } } }
    child[missing] {}
    child { node {15}
      child { node {+} }
      child { node {3} }
      child { node {5} }
      child { node {7} } } ;
\end{tikzpicture}
\caption{Tree representation, showing the value of each subcombination.}
\label{fig:1.1} % TODO: This should label the figure rather than the section.
\end{figure}

Next, observe that the repeated application of the first step brings
us to the point where we need to evaluate, not combinations, but
primitive expressions such as numerals, built-in operators, or other
names.  We take care of the primitive cases by stipulating that

\begin{itemize}
\item the values of numerals are the numbers that they name,
\item the values of built-in operators are the machine instruction
  sequences that carry out the corresponding operations, and
\item the values of other names are the objects associated with those
  names in the environment.
\end{itemize}

We may regard the second rule as a special case of the third one by
stipulating that symbols such as \texttt{+} and \texttt{*} are also
included in the global environment, and are associated with the
sequences of machine instructions that are their ``values.''  The key
point to notice is the role of the environment in determining the
meaning of the symbols in expressions.  In an interactive language
such as Lisp, it is meaningless to speak of the value of an expression
such as \texttt{(+ x 1)} without specifying any information about the
environment that would provide a meaning for the symbol \texttt{x} (or
even for the symbol \texttt{+}).  As we shall see in \ref{chap:3}, the
general notion of the environment as providing a context in which
evaluation takes place will play an important role in our
understanding of program execution.

Notice that the evaluation rule given above does not handle
definitions.  For instance, evaluating \texttt{(define x 3)} does not
apply \texttt{define} to two arguments, one of which is the value of
the symbol \texttt{x} and the other of which is 3, since the purpose
of the \texttt{define} is precisely to associate \texttt{x} with a
value.  (That is, \texttt{(define x 3)} is not a combination.)

Such exceptions to the general evaluation rule are called
\idef{special forms}.  \texttt{Define} is the only example of a
special form that we have seen so far, but we will meet others
shortly.  Each special form has its own evaluation rule. The various
kinds of expressions (each with its associated evaluation rule)
constitute the \idef{syntax} of the programming language.  In
comparison with most other programming languages, Lisp has a very
simple syntax; that is, the evaluation rule for expressions can be
described by a simple general rule together with specialized rules for
a small number of special forms.\footnote{Special syntactic forms that
  are simply convenient alternative surface structures for things that
  can be written in more uniform ways are sometimes called
  \idef{syntactic sugar}, to use a phrase coined by Peter Landin.  In
  comparison with users of other languages, Lisp programmers, as a
  rule, are less concerned with matters of syntax.  (By contrast,
  examine any Pascal manual and notice how much of it is devoted to
  descriptions of syntax.)  This disdain for syntax is due partly to
  the flexibility of Lisp, which makes it easy to change surface
  syntax, and partly to the observation that many ``convenient''
  syntactic constructs, which make the language less uniform, end up
  causing more trouble than they are worth when programs become large
  and complex.  In the words of Alan Perlis, ``Syntactic sugar causes
  cancer of the semicolon.''}

\subsection{Compound Procedures}
\label{sec:1.1.4}

We have identified in Lisp some of the elements that must appear in
any powerful programming language:

\begin{itemize}
\item Numbers and arithmetic operations are 
primitive data and procedures.
\item Nesting of combinations provides a means of 
combining operations.
\item Definitions that associate names with values provide a
limited means of abstraction.
\end{itemize}

Now we will learn about
\idef{procedure definitions}, a much more powerful abstraction
technique by which a compound operation can be given a name and then
referred to as a unit.

We begin by examining how to express the idea of ``squaring.''  We
might say, ``To square something, multiply it by itself.''  This is
expressed in our language as 

\begin{schemedisplay}
> (define (square x) (* x x))
\end{schemedisplay}

We can understand this in the following way:

% TODO: Fill in this diagram
\begin{tabular}{cccccc}
  $\texttt{(define}$ & $\texttt{(square}$ & $\texttt{x)}$ & $\texttt{(*}$ & $\texttt{x}$ & $\texttt{x))}$ \\
  $\uparrow{}$ & $\uparrow{}$ & $\uparrow{}$ & $\uparrow{}$ & $\uparrow{}$ & $\uparrow{}$ \\
  To & square & something, & multiply & it & by itself.
\end{tabular}

We have here a \idef{compound procedure}, which has been given the
name \texttt{square}.  The procedure represents the operation of
multiplying something by itself.  The thing to be multiplied is given
a local name, \texttt{x}, which plays the same role that a pronoun
plays in natural language.  Evaluating the definition creates this
compound procedure and associates it with the name \texttt{square}.
\footnote{Observe that there are two different operations being
  combined here: we are creating the procedure, and we are giving it
  the name \texttt{square}.  It is possible, indeed important, to be
  able to separate these two notions -- to create procedures without
  naming them, and to give names to procedures that have already been
  created.  We will see how to do this in section \ref{sec:1.3.2}.}


The general form of a procedure definition is

\begin{schemedisplay}
(define (<name> <formal parameters>) <body>)
\end{schemedisplay}

The \slot{name}; is a symbol to be associated with the procedure
definition in the environment.\footnote{Throughout this book, we will
  describe the general syntax of expressions by using italic symbols
  delimited by angle brackets -- e.g. \slot{name} -- to denote the
  ``slots'' in the expression to be filled in when such an expression
  is actually used.} The \slot{formal parameters} are the names used
within the body of the procedure to refer to the corresponding
arguments of the procedure.  The \slot{body} is an expression that
will yield the value of the procedure application when the formal
parameters are replaced by the actual arguments to which the procedure
is applied.\footnote{More generally, the body of the procedure can be
  a sequence of expressions.  In this case, the interpreter evaluates
  each expression in the sequence in turn and returns the value of the
  final expression as the value of the procedure application.}  The
\slot{name} and the \slot{formal parameters} are grouped within
parentheses, just as they would be in an actual call to the procedure
being defined.

Having defined \texttt{square}, we can now use it:

\begin{verbatim}
> (square 21)
441
> (square (+ 2 5))
49
> (square (square 3))
81
\end{verbatim}

We can also use \texttt{square} as a building block in defining other
procedures.  For example, $x^2 + y^2$

\begin{verbatim}
> (+ (square x) (square y))
\end{verbatim}

We can easily define a procedure \texttt{sum-of-squares} that, given
any two numbers as arguments, produces the sum of their squares:

\begin{verbatim}
>(define (sum-of-squares x y)
>  (+ (square x) (square y)))

> (sum-of-squares 3 4)
25
\end{verbatim}

Now we can use \texttt{sum-of-squares} as a building block in constructing
further procedures:

\begin{verbatim}
> (define (f a)
>  (sum-of-squares (+ a 1) (* a 2)))

> (f 5)
136
\end{verbatim}

Compound procedures are used in exactly the same way as primitive
procedures.  Indeed, one could not tell by looking at the definition
of \texttt{sum-of-squares} given above whether \texttt{square} was
built into the interpreter, like \texttt{+} and \texttt{*}, or defined
as a compound procedure.

\subsection{The Substitution Model for Procedure Application}
\label{sec:1.1.5}

To evaluate a combination whose operator names a compound procedure, the
interpreter follows much the same process as for combinations whose
operators name primitive procedures, which we described in
section \ref{sec:1.1.3}.  That is, the interpreter
evaluates the elements of the combination and applies the procedure
(which is the value of the operator of the combination) to the
arguments (which are the values of the operands of the combination).

We can assume that the mechanism for applying primitive procedures to
arguments is built into the interpreter.  For compound procedures, the
application process is as follows:

% TODO: Format this as pseudocode
\begin{verbatim}
To apply a compound procedure to arguments, evaluate the body of the
procedure with each formal parameter replaced by the corresponding
argument.
\end{verbatim}

To illustrate this process, let's evaluate the combination

\begin{schemedisplay}
(f 5)
\end{verbatim}

where \texttt{f} is the procedure defined in
section \ref{sec:1.1.4}.  We begin by retrieving the
body of \texttt{f}

\begin{schemedisplay}
(sum-of-squares (+ a 1) (* a 2))
\end{schemedisplay}

Then we replace the formal parameter \texttt{a} by the argument 5:

\begin{schemedisplay}
(sum-of-squares (+ 5 1) (* 5 2))
\end{schemedisplay}

Thus the problem reduces to the evaluation of a combination with two
operands and an operator \texttt{sum-of-squares}.  Evaluating this
combination involves three subproblems.  We must evaluate the operator
to get the procedure to be applied, and we must evaluate the operands
to get the arguments.  Now \texttt{(+ 5 1)} produces 6 and \texttt{(*
  5 2)} produces 10, so we must apply the \texttt{sum-of-squares}
procedure to 6 and 10.  These values are substituted for the formal
parameters \texttt{x} and \texttt{y} in the body of
\texttt{sum-of-squares}, reducing the expression to

\begin{schemedisplay}
(+ (square 6) (square 10))
\end{schemedisplay}

If we use the definition of \texttt{square}, this reduces to

\begin{schemedisplay}
(+ (* 6 6) (* 10 10))
\end{schemedisplay}

which reduces by multiplication to


\begin{schemedisplay}
(+ 36 100)
\end{schemedisplay}

and finally to

\begin{schemedisplay}
136
\end{schemedisplay}

The process we have just described is called the \textit{substitution
model} for procedure application.  It can be taken as a model that
determines the ``meaning'' of procedure application, insofar as the
procedures in this chapter are concerned.  However, there are two
points that should be stressed:

\begin{itemize}
\item The purpose of the substitution is to help us think about procedure
application, not to provide a description of how the interpreter
really works.  Typical interpreters do not evaluate procedure
applications by manipulating the text of a procedure to substitute
values for the formal parameters.  In practice, the ``substitution''
is accomplished by using a local environment for the formal
parameters.  We will discuss this more fully in chapters 3 and 4 when
we examine the implementation of an interpreter in detail.

\item Over the course of this book, we will present a sequence of
  increasingly elaborate models of how interpreters work, culminating
  with a complete implementation of an interpreter and compiler in
  chapter 5.  The substitution model is only the first of these models
  -- a way to get started thinking formally about the evaluation
  process.  In general, when modeling phenomena in science and
  engineering, we begin with simplified, incomplete models.  As we
  examine things in greater detail, these simple models become
  inadequate and must be replaced by more refined models.  The
  substitution model is no exception.  In particular, when we address
  in chapter 3 the use of procedures with ``mutable data,'' we will
  see that the substitution model breaks down and must be replaced by
  a more complicated model of procedure application.\footnote{Despite
    the simplicity of the substitution idea, it turns out to be
    surprisingly complicated to give a rigorous mathematical
    definition of the substitution process.  The problem arises from
    the possibility of confusion between the names used for the formal
    parameters of a procedure and the (possibly identical) names used
    in the expressions to which the procedure may be applied.  Indeed,
    there is a long history of erroneous definitions of
    \textit{substitution} in the literature of logic and programming
    semantics.  See Stoy 1977 for a careful discussion of
    substitution.}
\end{itemize}

\subsection*{Applicative order versus normal order}

According to the description of evaluation given in section
\ref{sec:1.1.3}, the interpreter first evaluates the operator and operands
and then applies the resulting procedure to the resulting arguments.
This is not the only way to perform evaluation.  An alternative
evaluation model would not evaluate the operands until their values
were needed.  Instead it would first substitute operand expressions
for parameters until it obtained an expression involving only
primitive operators, and would then perform the evaluation.  If we
used this method, the evaluation of

\begin{schemedisplay}
(f 5)
\end{schemedisplay}

\noindent would proceed according to the sequence of expansions

\begin{schemedisplay}
(sum-of-squares (+ 5 1) (* 5 2))

(+    (square (+ 5 1))      (square (* 5 2))  )

(+    (* (+ 5 1) (+ 5 1))   (* (* 5 2) (* 5 2)))
\end{schemedisplay}

\noindent followed by the reductions

\begin{schemedisplay}
(+         (* 6 6)             (* 10 10))

(+           36                   100)

                    136
\end{schemedisplay}

This gives the same answer as our previous evaluation model, but the
process is different.  In particular, the evaluations
of \texttt{(+ 5 1)} and \texttt{(* 5 2)} are each performed twice here,
corresponding to the reduction of the expression

\begin{schemedisplay}
(* x x)
\end{schemedisplay}

\noindent with \texttt{x} replaced respectively by \texttt{(+ 5 1)} and \texttt{(* 5 2)}.

This alternative ``fully expand and then reduce'' evaluation method is
known as \idef{normal-order evaluation}, in contrast to the ``evaluate
the arguments and then apply'' method that the interpreter actually
uses, which is called \idef{applicative-order evaluation}.  It can be
shown that, for procedure applications that can be modeled using
substitution (including all the procedures in the first two chapters
of this book) and that yield legitimate values, normal-order and
applicative-order evaluation produce the same value.  (See exercise
\ref{exc:1.5} for an instance of an ``illegitimate'' value where
normal-order and applicative-order evaluation do not give the same
result.)

Lisp uses applicative-order evaluation, partly because of the
additional efficiency obtained from avoiding multiple evaluations of
expressions such as those illustrated with \texttt{(+ 5 1)} and
\texttt{(* 5 2)} above and, more significantly, because normal-order
evaluation becomes much more complicated to deal with when we leave
the realm of procedures that can be modeled by substitution.  On the
other hand, normal-order evaluation can be an extremely valuable tool,
and we will investigate some of its implications in chapters 3 and
4.\footnote{In chapter 3, we will introduce \textit{stream
    processing}, which is a way of handling apparently ``infinite''
  data structures by incorporating a limited form of normal-order
  evaluation.  In section \ref{sec:4.2} we will modify the scheme
  interpreter to produce a normal-order variant of Scheme.}

\subsection{Conditional Expressions and Predicates}
\label{sec:1.1.6}

The expressive power of the class of procedures that we can define at
this point is very limited, because we have no way to make tests and
to perform different operations depending on the result of a test.
For instance, we cannot define a procedure that computes the absolute
value of a number by testing whether the number is positive, negative,
or zero and taking different actions in the different cases according
to the rule

\begin{equation*}
  % TODO: Format this equation nicely
  |x| = 
    \begin{cases}
       \quad x & \text{if } x > 0 \\
       \quad 0 & \text{if } x = 0 \\
       -x & \text{if } x < 0
    \end{cases}
\end{equation*}

This construct is called a \idef{case analysis}, and
there is a special form in Lisp for notating such a case
analysis.  It is called \texttt{cond} (which stands for
``conditional''), and it is used as follows:

\begin{schemedisplay}
(define (abs x)
  (cond ((> x 0) x)
        ((= x 0) 0)
        ((< x 0) (- x))))
\end{schemedisplay}

The general form of a conditional expression is

\begin{schemedisplay}
% TODO: Format this properly
(cond (<<em>p_1</em>> <<em>e_1</em>>)
      (<<em>p_2</em>> <<em>e_2</em>>)
      <img src="book-Z-G-D-18.gif" border="0">
      (<<em>p_{<em>n</em>}</em>> <<em>e_{<em>n</em>}</em>>))
\end{schemedisplay}

consisting of the symbol \texttt{cond} followed by parenthesized pairs
of expressions \texttt{(\slot{p} \slot{e})} called \idef{clauses}. The
first expression in each pair is a \idef{predicate} -- that is, an
expression whose value is interpreted as either true or
false.\footnote{``Interpreted as either true or false'' means this: In
  Scheme, there are two distinguished values that are denoted by the
  constants \texttt{\#t} and \texttt{\#f}.  When the interpreter checks
  a predicate's value, it interprets \texttt{\#f} as false.  Any other
  value is treated as true.  (Thus, providing \texttt{\#t} is logically
  unnecessary, but it is convenient.)  In this book we will use names
  \texttt{true} and \texttt{false}, which are associated with the
  values \texttt{\#t} and \texttt{\#f} respectively.}


\idef{Conditional expressions} are evaluated as follows.  The predicate
\slot{p_1} is evaluated first.  If its value is false, then
\slot{p_2} is evaluated.  If \slot{p_2}'s value is also false, then
\slot{p_3} is evaluated.  This process continues until a predicate is
found whose value is true, in which case the interpreter returns the
value of the corresponding \idef{consequent expression} \slot{e} of the
clause as the value of the conditional expression.  If none of the
\slot{p}'s is found to be true, the value of the \texttt{cond} is
undefined.

The word \idef{predicate} is used for procedures that return true or
false, as well as for expressions that evaluate to true or false.  The
absolute-value procedure \texttt{abs} makes use of the
\index{primitive}primitive predicates \texttt{$>$}, \texttt{$<$}, and
\texttt{=}.\footnote{\texttt{abs} also uses the ``minus'' operator
  \texttt{-}, which, when used with a single operand, as in \texttt{(-
    x)}, indicates negation.}  These take two numbers as arguments and
test whether the first number is, respectively, greater than, less
than, or equal to the second number, returning true or false
accordingly.

Another way to write the absolute-value procedure is

\begin{schemedisplay}
(define (abs x)
  (cond ((< x 0) (- x))
        (else x)))
\end{schemedisplay}

\noindent which could be expressed in English as ``If $x$ is less than
zero return $-x$; otherwise return $x$.''  \texttt{Else} is a special
symbol that can be used in place of the \slot{p} in the final clause
of a \texttt{cond}.  This causes the \texttt{cond} to return as its
value the value of the corresponding \slot{e} whenever all previous
clauses have been bypassed.  In fact, any expression that always
evaluates to a true value could be used as the \slot{p} here.

Here is yet another way to write the absolute-value procedure:

\begin{schemedisplay}
(define (abs x)
  (if (< x 0)
      (- x)
      x))
\end{schemedisplay}

This uses the special form \texttt{if}, a restricted type of conditional
that can be used when there are precisely two cases in the case
analysis.  The general form of an \texttt{if} expression is

\begin{schemedisplay}
(if \slot{predicate} \slot{consequent} \slot{alternative})
\end{schemedisplay}

To evaluate an \texttt{if} expression, the interpreter starts by
evaluating the \slot{predicate} part of the expression.  If the
\slot{predicate} evaluates to a true value, the interpreter then
evaluates the \slot{consequent} and returns its value.  Otherwise it
evaluates the \slot{alternative} and returns its value.\footnote{A
  minor difference between \texttt{if} and \texttt{cond} is that the
  \slot{e} part of each \texttt{cond} clause may be a sequence of
  expressions. If the corresponding \slot{p} is found to be true, the
  expressions \slot{e} are evaluates in sequence and the value of the
  final expression in the sequence is returned as the value of the
  \texttt{cond}. In an \texttt{if} expression, however, the
  \slot{consequent} and \slot{alternative} must be single
  expressions.}

In addition to primitive predicates such as \texttt{<}, \texttt{=},
and \texttt{>}, there are logical composition operations, which enable
us to construct compound predicates.  The three most frequently used
are these:

\begin{itemize}
\item \texttt{(and \slot{e_1} \ldots{} \slot{e_n})}

The interpreter evaluates the expressions \slot{e} one at a time, in
left-to-right order.  If any \slot{e} evaluates to false, the value of
the \texttt{and} expression is false, and the rest of the \slot{e}'s
are not evaluated.  If all \slot{e}'s evaluate to true values, the
value of the \texttt{and} expression is the value of the last one.

\item \texttt{(or \slot{e_1} \ldots{} \slot{e_n})}

The interpreter evaluates the expressions \slot{e} one at a time, in
left-to-right order.  If any \slot{e} evaluates to a true value, that
value is returned as the value of the \texttt{or} expression, and the
rest of the \slot{e}'s are not evaluated.  If all \slot{e}'s evaluate
to false, the value of the \texttt{or} expression is false.

\item \texttt{(not \slot{e})}

The value of a \texttt{not} expression is true
when the expression \slot{e} evaluates to false, and false otherwise.
\end{itemize}

Notice that \texttt{and} and \texttt{or} are special forms, not
procedures, because the subexpressions are not necessarily all
evaluated.  \texttt{Not} is an ordinary procedure.

As an example of how these are used, the condition that a number
$x$ be in the range $5 < x < 10$ may be expressed as

\begin{schemedisplay}
(and (> x 5) (< x 10))
\end{schemedisplay}

As another example, we can define a predicate to test whether one
number is greater than or equal to another as

\begin{schemedisplay}
(define (>= x y)
  (or (> x y) (= x y)))
\end{schemedisplay}

\noindent or alternatively as

\begin{schemedisplay}
(define (>= x y)
  (not (< x y)))
\end{schemedisplay}

\begin{Exercise}
  \label{exc:1.1}
Below is a sequence of expressions.  What is the result printed by the
interpreter in response to each expression?  Assume that the sequence
is to be evaluated in the order in which it is presented.

\begin{schemedisplay}
10
(+ 5 3 4)
(- 9 1)
(/ 6 2)
(+ (* 2 4) (- 4 6))
(define a 3)
(define b (+ a 1))
(+ a b (* a b))
(= a b)
(if (and (> b a) (< b (* a b)))
    b
    a)
(cond ((= a 4) 6)
      ((= b 4) (+ 6 7 a))
      (else 25))
(+ 2 (if (> b a) b a))
(* (cond ((> a b) a)
         ((< a b) b)
         (else -1))
   (+ a 1))
\end{schemedisplay}
\end{Exercise}

\begin{Exercise}
\label{exc:1.2}
Translate the following expression into prefix form
\begin{equation*}
  \frac{5 + 4 + 3 + (2 - (3 - (6 + \frac{4}{5})))}
  {3 (6-2) (2-7)}
\end{equation*}
\end{Exercise}

\begin{Exercise}
\label{exc:1.3}
Define a procedure that takes three numbers as arguments and returns
the sum of the squares of the two larger numbers.
\end{Exercise}

\begin{Exercise}
\label{exc:1.4}
Observe that our model of evaluation allows for combinations whose
operators are compound expressions.  Use this observation to
describe the behavior of the following procedure:
\begin{schemedisplay}
(define (a-plus-abs-b a b)
  ((if (> b 0) + -) a b))
\end{schemedisplay}
\end{Exercise}

\begin{Exercise}
\label{exc:1.5}
Ben Bitdiddle has invented a test to determine whether the interpreter
he is faced with is using applicative-order evaluation or normal-order
evaluation.  He defines the following two procedures:

\begin{schemedisplay}
(define (p) (p))

(define (test x y)
  (if (= x 0)
      0
      y))
\end{schemedisplay}

Then he evaluates the expression

\begin{schemedisplay}
(test 0 (p))
\end{schemedisplay}

What behavior will Ben observe with an interpreter that uses
applicative-order evaluation?  What behavior will he observe with an
interpreter that uses normal-order evaluation?  Explain your answer.
(Assume that the evaluation rule for the special form \texttt{if} is
the same whether the interpreter is using normal or applicative order:
The predicate expression is evaluated first, and the result determines
whether to evaluate the consequent or the alternative expression.)
\end{Exercise}

\subsection{Example: Square Roots by Newton's Method}
\label{sec:1.1.7}

Procedures, as introduced above, are much like ordinary mathematical
functions.  They specify a value that is determined by one or more
parameters.  But there is an important difference between
mathematical functions and computer procedures.  Procedures must be
effective.

As a case in point, consider the problem of computing square
roots.  We can define the square-root function as

\begin{displaymath}
  \sqrt{x} = \mathrm{the}~ y ~\mathrm{such~that}~ y \ge 0 ~\mathrm{and}~ {y^2} = x
\end{displaymath}

This describes a perfectly legitimate mathematical function.  We could
use it to recognize whether one number is the square root of another, or
to derive facts about square roots in general.  On the other hand, the
definition does not describe a procedure.  Indeed, it tells us almost
nothing about how to actually find the square root of a given number.  It
will not help matters to rephrase this definition in pseudo-Lisp:

\begin{schemedisplay}
(define (sqrt x)
  (the y (and (>= y 0)
              (= (square y) x))))
\end{schemedisplay}

This only begs the question.

The contrast between function and procedure is a reflection of the
general distinction between describing properties of things and
describing how to do things, or, as it is sometimes referred to, the
distinction between declarative knowledge and imperative knowledge.
In mathematics we are usually concerned with declarative (what is)
descriptions, whereas in computer science we are usually concerned
with imperative (how to) descriptions.\footnote{ Declarative and
  imperative descriptions are intimately related, as indeed are
  mathematics and computer science.  For instance, to say that the
  answer produced by a program is ``correct'' is to make a declarative
  statement about the program.  There is a large amount of research
  aimed at establishing techniques for proving that programs are
  correct, and much of the technical difficulty of this subject has to
  do with negotiating the transition between imperative statements
  (from which programs are constructed) and declarative statements
  (which can be used to deduce things).  In a related vein, an
  important current area in programming-language design is the
  exploration of so-called very high-level languages, in which one
  actually programs in terms of declarative statements.  The idea is
  to make interpreters sophisticated enough so that, given ``what is''
  knowledge specified by the programmer, they can generate ``how to''
  knowledge automatically.  This cannot be done in general, but there
  are important areas where progress has been made.  We shall revisit
  this idea in chapter \ref{chap:4}.}


How does one compute square roots?  The most common way is to use
Newton's method of successive approximations, which says that whenever
we have a guess $y$ for the value of the square root of a number $x$,
we can perform a simple manipulation to get a better guess (one closer
to the actual square root) by averaging $y$ with
$\frac{x}{y}$.\footnote{This square-root algorithm is actually a
  special case of Newton's method, which is a general technique for
  finding roots of equations.  The square-root algorithm itself was
  developed by Heron of Alexandria in the first century \textsc{a.d}.
  We will see how to express the general Newton's method as a Lisp
  procedure in section \ref{sec:1.3.4}.}  For example, we can compute
the square root of 2 as follows.  Suppose our initial guess is 1:

\begin{tabular}{|c|c|c|}
  \hline{}
  Guess & Quotient & Average \\
  \hline{}
  1 & $2/1 = 2$ & $\frac{2+1}{2} = 1.5$ \\
  1.5 & $\frac{2}{1.5} = 1.3333$ & $\frac{1.3333 + 1.5}{2} = 1.4167$ \\
  1.4167 & $\frac{2}{1.4167} = 1.4118$ & $\frac{1.4167 + 1.4118}{2} = 1.4142$ \\
  1.4142 & \ldots & \ldots \\
  \hline{}
  % TODO: Fix the lines coming out of the bottom of this table.
\end{tabular}

Continuing this process, we obtain better and better
approximations to the square root.

Now let's formalize the process in terms of procedures.  We start with
a value for the radicand (the number whose square root we are trying
to compute) and a value for the guess.  If the guess is good enough
for our purposes, we are done; if not, we must repeat the process with an
improved guess.  We write this basic strategy as a procedure:

\begin{schemedisplay}
(define (sqrt-iter guess x)
  (if (good-enough? guess x)
      guess
      (sqrt-iter (improve guess x)
                 x)))
\end{schemedisplay}

A guess is improved by averaging
it with the quotient of the radicand and the old guess:

\begin{schemedisplay}
(define (improve guess x)
  (average guess (/ x guess)))
\end{schemedisplay}

\noindent where

\begin{schemedisplay}
(define (average x y)
  (/ (+ x y) 2))
\end{schemedisplay}

We also have to say what we mean by ``good enough.''  The following
will do for illustration, but it is not really a very good test.  (See
exercise \ref{exc:1.7}.)  The idea is to improve the answer until it
is close enough so that its square differs from the radicand by less
than a predetermined tolerance (here 0.001):\footnote{We will usually
  give predicates names ending with question marks, to help us
  remember that they are predicates.  This is just a stylistic
  convention.  As far as the interpreter is concerned, the question
  mark is just an ordinary character.}

\begin{schemedisplay}
(define (good-enough? guess x)
  (< (abs (- (square guess) x)) 0.001))
\end{schemedisplay}

Finally, we need a way to get started.  For instance, we can always
guess that the square root of any number is 1:\footnote{Observe that
  we express our initial guess as 1.0 rather than 1.  This would not
  make any difference in many Lisp implementations. MIT Scheme,
  however, distinguishes between exact integers and decimal values,
  and dividing two integers produces a rational number rather than a
  decimal.  For example, dividing 10 by 6 yields 5/3, while dividing
  10.0 by 6.0 yields 1.6666666666666667.  (We will learn how to
  implement arithmetic on rational numbers in section
  \ref{sec:2.1.1}.)  If we start with an initial guess of 1 in our
  square-root program, and $x$ is an exact integer, all subsequent
  values produced in the square-root computation will be rational
  numbers rather than decimals.  Mixed operations on rational numbers
  and decimals always yield decimals, so starting with an initial
  guess of 1.0 forces all subsequent values to be decimals.}

\begin{schemedisplay}
(define (sqrt x)
  (sqrt-iter 1.0 x))
\end{schemedisplay}

If we type these definitions to the interpreter, we can use \texttt{sqrt}
just as we can use any procedure:

\begin{schemedisplay}
> (sqrt 9)
3.00009155413138
> (sqrt (+ 100 37))
11.704699917758145
> (sqrt (+ (sqrt 2) (sqrt 3)))
> 1.7739279023207892
(square (sqrt 1000))
> 1000.000369924366
\end{schemedisplay}

The \texttt{sqrt} program also illustrates that the simple procedural
language we have introduced so far is sufficient for writing any
purely numerical program that one could write in, say, C or Pascal.
This might seem surprising, since we have not included in our language
any iterative (looping) constructs that direct the computer to do
something over and over again.  \texttt{Sqrt-iter}, on the other hand,
demonstrates how iteration can be accomplished using no special
construct other than the ordinary ability to call a procedure.
\footnote{Readers who are worried about the efficiency issues involved
  in using procedure calls to implement iteration should note the
  remarks on ``tail recursion'' in section \ref{sec:1.2.1}}

\begin{Exercise}
\label{exc:1.6}
Alyssa P. Hacker doesn't see why \texttt{if} needs to be provided as a
special form.  ``Why can't I just define it as an ordinary procedure
in terms of \texttt{cond}?'' she asks.  Alyssa's friend Eva Lu Ator
claims this can indeed be done, and she defines a new version of
\texttt{if}:
\begin{schemedisplay}
(define (new-if predicate then-clause else-clause)
  (cond (predicate then-clause)
        (else else-clause)))
\end{schemedisplay}

Eva demonstrates the program for Alyssa:

\begin{verbatim}
> (new-if (= 2 3) 0 5)
5
> (new-if (= 1 1) 0 5)
0
\end{verbatim}

Delighted, Alyssa uses \texttt{new-if} to rewrite the square-root
program:

\begin{schemedisplay}
(define (sqrt-iter guess x)
  (new-if (good-enough? guess x)
          guess
          (sqrt-iter (improve guess x)
                     x)))
\end{schemedisplay}

What happens when Alyssa attempts to use this to compute square roots?
Explain.
\end{Exercise}

\begin{Exercise}
\label{exc:1.7}
The \texttt{good-enough?} test used in computing square roots will not
be very effective for finding the square roots of very small numbers.
Also, in real computers, arithmetic operations are almost always
performed with limited precision.  This makes our test inadequate for
very large numbers.  Explain these statements, with examples showing
how the test fails for small and large numbers.  An alternative
strategy for implementing \texttt{good-enough?} is to watch how
\texttt{guess} changes from one iteration to the next and to stop when
the change is a very small fraction of the guess.  Design a
square-root procedure that uses this kind of end test.  Does this work
better for small and large numbers?
\end{Exercise}

\begin{Exercise}
\label{exc:1.8}
Newton's method for cube roots is based on the fact that if \textit{y} is an
approximation to the cube root of \textit{x}, then a better approximation is
given by the value $$x / y^2 + 2y \over 3$$
Use this formula to implement a cube-root procedure analogous to the
square-root procedure.  (In section \ref{sec:1.3.4} we
will see how to implement Newton's method in general as an abstraction
of these square-root and cube-root procedures.)
\end{Exercise}

\section{Procedures as Black-Box Abstractions}
\label{sec:1.1.8}

\texttt{Sqrt} is our first example of a process defined by a set of
mutually defined procedures.  Notice that the definition of
\texttt{sqrt-iter} is \idef{recursive}; that is, the procedure is
defined in terms of itself.  The idea of being able to define a
procedure in terms of itself may be disturbing; it may seem unclear
how such a ``circular'' definition could make sense at all, much less
specify a well-defined process to be carried out by a computer.  This
will be addressed more carefully in section \ref{sec:1.2}.  But first
let's consider some other important points illustrated by the
\texttt{sqrt} example.

Observe that the problem of computing square roots breaks up naturally
into a number of subproblems: how to tell whether a guess is good
enough, how to improve a guess, and so on.  Each of these tasks is
accomplished by a separate procedure.  The entire \texttt{sqrt} program
can be viewed as a cluster of procedures (shown in
figure \ref{fig:1.2}) that mirrors the decomposition of
the problem into subproblems.

\begin{figure}
  % TODO: Improve this figure.
  \begin{tikzpicture}[]
    \node {sqrt}
      child { node {sqrt-iter}
        child { node {good-enough}
          child { node {square} }
          child { node {abs} } }
        child { node {improve}
          child { node {average} } } } ;
\end{tikzpicture}
  \caption{Procedural decomposition of the \texttt{sqrt} program}
  \label{fig:1.2}
\end{figure}
          
            
The importance of this decomposition strategy is not simply that one
is dividing the program into parts.  After all, we could take any
large program and divide it into parts -- the first ten lines, the next
ten lines, the next ten lines, and so on.  Rather, it is crucial that
each procedure accomplishes an identifiable task that can be used as a
module in defining other procedures.  For example, when we define the
\texttt{good-enough?} procedure in terms of \texttt{square}, we are able to
regard the \texttt{square} procedure as a ``black box.''  We are not at
that moment concerned with \textit{how} the procedure computes its
result, only with the fact that it computes the square.  The details
of how the square is computed can be suppressed, to be considered at a
later time.  Indeed, as far as the \texttt{good-enough?} procedure is
concerned, \texttt{square} is not quite a procedure but rather an
abstraction of a procedure, a so-called \idef{procedural abstraction}.
At this level of abstraction, any procedure that computes the square
is equally good.

Thus, considering only the values they return, the following two
procedures for squaring a number should be indistinguishable.  Each
takes a numerical argument and produces the square of that number as
the value. \footnote{It is not even clear which of these procedures is
  a more efficient implementation.  This depends upon the hardware
  available.  There are machines for which the ``obvious''
  implementation is the less efficient one.  Consider a machine that
  has extensive tables of logarithms and antilogarithms stored in a
  very efficient manner.}

\begin{schemedisplay}
(define (square x) (* x x))

(define (square x) 
  (exp (double (log x))))

(define (double x) (+ x x))
\end{schemedisplay}

So a procedure definition should be able to suppress detail.  The
users of the procedure may not have written the procedure themselves,
but may have obtained it from another programmer as a black box.  A
user should not need to know how the procedure is implemented in order
to use it.

\subsubsection*{Local names}

One detail of a procedure's implementation that should not matter to
the user of the procedure is the implementer's choice of names for the
procedure's formal parameters.  Thus, the following procedures should
not be distinguishable:

\begin{schemedisplay}
(define (square x) (* x x))

(define (square y) (* y y))
\end{schemedisplay}

This principle -- that the meaning of a procedure should be independent
of the parameter names used by its author -- seems on the surface to be
self-evident, but its consequences are profound.  The simplest
consequence is that the parameter names of a procedure must be local
to the body of the procedure.  For example, we used \texttt{square} in
the definition of \texttt{good-enough?} in our square-root procedure:

\begin{schemedisplay}
(define (good-enough? guess x)
  (< (abs (- (square guess) x)) 0.001))
\end{schemedisplay}

The intention of the author of \texttt{good-enough?} is to determine if
the square of the first argument is within a given tolerance of the
second argument.  We see that the author of \texttt{good-enough?} used
the name \texttt{guess} to refer to the first argument and \texttt{x} to
refer to the second argument.  The argument of \texttt{square} is \texttt{guess}.  If the author of \texttt{square} used \texttt{x} (as above)
to refer to that argument, we see that the \texttt{x} in \texttt{good-enough?} must be a different \texttt{x} than the one in \texttt{square}.  Running the procedure \texttt{square} must not affect the value
of \texttt{x} that is used by \texttt{good-enough?}, because that value of
\texttt{x} may be needed by \texttt{good-enough?} after \texttt{square} is done
computing.

If the parameters were not local to the bodies of their respective
procedures, then the parameter \texttt{x} in \texttt{square} could be
confused with the parameter \texttt{x} in \texttt{good-enough?}, and the
behavior of \texttt{good-enough?} would depend upon which version of
\texttt{square} we used.  Thus, \texttt{square} would not be the black box
we desired.

A formal parameter of a procedure has a very special role in the
procedure definition, in that it doesn't matter what name the formal
parameter has.  Such a name is called a \idef{bound variable}, and we
say that the procedure definition \idef{binds} its formal parameters.
The meaning of a procedure definition is unchanged if a bound variable
is consistently renamed throughout the definition.\footnote{The
concept of consistent renaming is actually subtle and difficult to
define formally.  Famous logicians have made embarrassing errors
here.}  If a variable is not bound, we say that it is \idef{free}.  The
set of expressions for which a binding defines a name is called the
\idef{scope} of that name.
In a procedure definition, the bound variables
declared as the \idef{formal parameters} of the procedure have the body of
the procedure as their scope.

In the definition of \texttt{good-enough?} above, \texttt{guess} and \texttt{x} are
bound variables but \texttt{<}, \texttt{-}, \texttt{abs}, and \texttt{square} are free.
The meaning of \texttt{good-enough?} should be independent of the names we
choose for \texttt{guess} and \texttt{x} so long as they are distinct and
different from \texttt{<}, \texttt{-}, \texttt{abs}, and \texttt{square}.  (If we renamed
\texttt{guess} to \texttt{abs} we would have introduced a bug by \idef{capturing}
the variable \texttt{abs}.  It would have changed from free to bound.)  The
meaning of \texttt{good-enough?} is not independent of the names of its
free variables, however.  It surely depends upon the fact (external to
this definition) that the symbol \texttt{abs} names a procedure for
computing the absolute value of a number.  \texttt{Good-enough?} will
compute a different function if we substitute \texttt{cos} for \texttt{abs} in
its definition.

\subsubsection*{Internal definitions and block structure}


We have one kind of name isolation available to us so far: The formal
parameters of a procedure are local to the body of the procedure.  The
square-root program illustrates another way in which we would like to
control the use of names.  The existing program consists of separate
procedures:

\begin{schemedisplay}
(define (sqrt x)
  (sqrt-iter 1.0 x))
(define (sqrt-iter guess x)
  (if (good-enough? guess x)
      guess
      (sqrt-iter (improve guess x) x)))
(define (good-enough? guess x)
  (< (abs (- (square guess) x)) 0.001))
(define (improve guess x)
  (average guess (/ x guess)))
\end{schemedisplay}

The problem with this program is that the only procedure that is
important to users of \texttt{sqrt} is \texttt{sqrt}.  The other
procedures (\texttt{sqrt-iter}, \texttt{good-enough?}, and
\texttt{improve}) only clutter up their minds.  They may not define
any other procedure called \texttt{good-enough?} as part of another
program to work together with the square-root program, because
\texttt{sqrt} needs it.  The problem is especially severe in the
construction of large systems by many separate programmers.  For
example, in the construction of a large library of numerical
procedures, many numerical functions are computed as successive
approximations and thus might have procedures named
\texttt{good-enough?} and \texttt{improve} as auxiliary procedures.
We would like to localize the subprocedures, hiding them inside
\texttt{sqrt} so that \texttt{sqrt} could coexist with other
successive approximations, each having its own private
\texttt{good-enough?} procedure.  To make this possible, we allow a
procedure to have internal definitions that are local to that
procedure.  For example, in the square-root problem we can write

\begin{schemedisplay}
(define (sqrt x)
  (define (good-enough? guess x)
    (< (abs (- (square guess) x)) 0.001))
  (define (improve guess x)
    (average guess (/ x guess)))
  (define (sqrt-iter guess x)
    (if (good-enough? guess x)
        guess
        (sqrt-iter (improve guess x) x)))
  (sqrt-iter 1.0 x))
\end{schemedisplay}

Such nesting of definitions, called \idef{block structure},
is basically the right solution to the simplest 
name-packaging problem.  But there is a better idea lurking here.  In
addition to internalizing the definitions of the auxiliary procedures,
we can simplify them.  Since \texttt{x} is bound in the definition of
\texttt{sqrt}, the procedures \texttt{good-enough?}, \texttt{improve}, and
\texttt{sqrt-iter}, which are defined internally to \texttt{sqrt}, are in the
scope of \texttt{x}.  Thus, it is not necessary to pass \texttt{x} explicitly to
each of these procedures.  Instead, we allow \texttt{x} to be a free
variable in the internal definitions, as shown below. Then \texttt{x}
gets its value from the argument with which the enclosing
procedure \texttt{sqrt} is called.  This discipline is called \idef{lexical
scoping}.\footnote{Lexical
scoping dictates that free variables in a procedure are taken to refer to
bindings made by enclosing procedure definitions;
that is, they are looked up in
the environment in which the procedure was defined.  We will see how
this works in detail in chapter \ref{chap:3} when we study environments and the
detailed behavior of the interpreter.}

\begin{schemedisplay}
(define (sqrt x)
  (define (good-enough? guess)
    (< (abs (- (square guess) x)) 0.001))
  (define (improve guess)
    (average guess (/ x guess)))
  (define (sqrt-iter guess)
    (if (good-enough? guess)
        guess
        (sqrt-iter (improve guess))))
  (sqrt-iter 1.0))
\end{schemedisplay}

We will use block structure extensively to help us break up large
programs into tractable pieces. \footnote{Embedded definitions must
  come first in a procedure body.  The management is not responsible
  for the consequences of running programs that intertwine definition
  and use.}

The idea of block structure originated with the programming language
Algol 60.  It appears in most advanced programming languages and is an
important tool for helping to organize the construction of large
programs.


% -*- TeX-master: "sicp.tex" -*-
\section{Procedures and the Processes They Generate}
\label{sec:1.2}


We have now considered the elements of programming: We have used
primitive arithmetic operations, we have combined these operations, and
we have abstracted these composite operations by defining them as compound
procedures.  But that is not enough to enable us to say that we know
how to program.  Our situation is analogous to that of someone who has
learned the rules for how the pieces move in chess but knows nothing
of typical openings, tactics, or strategy.  Like the novice chess
player, we don't yet know the common patterns of usage in the domain.
We lack the knowledge of which moves are worth making (which
procedures are worth defining).  We lack the experience to predict the
consequences of making a move (executing a procedure).

The ability to visualize the consequences of the actions under
consideration is crucial to becoming an expert programmer, just as it
is in any synthetic, creative activity.  In becoming an expert
photographer, for example, one must learn how to look at a scene and
know how dark each region will appear on a print for each possible
choice of exposure and development conditions.  Only then can one
reason backward, planning framing, lighting, exposure, and development
to obtain the desired effects.  So it is with programming, where we
are planning the course of action to be taken by a process and where
we control the process by means of a program.  To become experts, we
must learn to visualize the processes generated by various types of
procedures.  Only after we have developed such a skill can we learn
to reliably construct programs that exhibit the desired behavior.

A procedure is a pattern for the \textit{local evolution} of a
computational process.  It specifies how each stage of the process is
built upon the previous stage.  We would like to be able to make
statements about the overall, or \textit{global}, behavior of a
process whose local evolution has been specified by a procedure.  This
is very difficult to do in general, but we can at least try to
describe some typical patterns of process evolution.

In this section we will examine some common ``shapes'' for processes
generated by simple procedures.  We will also investigate the
rates at which these processes consume the important computational
resources of time and space.  The procedures we will consider
are very simple.  Their role is like that played by test patterns in
photography: as oversimplified prototypical patterns, rather than
practical examples in their own right.

\subsection{Linear Recursion and Iteration}
\label{sec:1.2.1}

\begin{figure}
  \centering
  \begin{tikzpicture}
    [every node/.style={anchor=west}]
    \draw (0,0) node (a) {\scheme|(factorial 6)|} ;
    \draw (0,-0.5) node () {\scheme|(* 6 (factorial 5))|} ;
    \draw (0,-1) node () {\scheme|(* 6 (* 5 (factorial 4)))|} ;
    \draw (0,-1.5) node () {\scheme|(* 6 (* 5 (* 4 (factorial 3))))|} ;
    \draw (0,-2) node () {\scheme|(* 6 (* 5 (* 4 (* 3 (factorial 2)))))|} ;
    \draw (0,-2.5) node () {\scheme|(* 6 (* 5 (* 4 (* 3 (* 2 (factorial 1))))))|} ;
    \draw (0,-3) node () {\scheme|(* 6 (* 5 (* 4 (* 3 (* 2 1)))))|} ;
    \draw (0,-3.5) node () {\scheme|(* 6 (* 5 (* 4 (* 3 2))))|} ;
    \draw (0,-4) node () {\scheme|(* 6 (* 5 (* 4 6)))|} ;
    \draw (0,-4.5) node () {\scheme|(* 6 (* 5 24))|} ;
    \draw (0,-5) node () {\scheme|(* 6 120)|} ;
    \draw (0,-5.5) node (b) {\scheme|720|} ;
    \draw[->,red] (a) .. controls +(8cm,-1cm) and +(8cm,1cm) ..  (b) ;
  \end{tikzpicture}
  \caption{A linear recursive process for computing $6!$}
  \label{fig:1.3}
\end{figure}

We begin by considering the factorial function, defined by

\begin{displaymath}
n! = n \cdot (n-1) \cdot (n-2) \cdots 3 \cdot 2 \cdot 1
\end{displaymath}

There are many ways to compute factorials.  One way is to make use of
the observation that $n!$ is equal to $n$ times $(n - 1)!$ for
any positive integer $n$:


\begin{displaymath}
n! = n \cdot \left[ (n-1) \cdot (n-2) \cdots 3 \cdot 2 \cdot 1 \right] = n \cdot (n-1)!
\end{displaymath}

Thus, we can compute $n!$ by computing $(n - 1)!$ and multiplying the
result by $n$.  If we add the stipulation that $1!$ is equal to $1$,
this observation translates directly into a procedure:

\begin{schemedisplay}
(define (factorial n)
  (if (= n 1)
      1
      (* n (factorial (- n 1)))))
\end{schemedisplay}

We can use the substitution model of section \ref{sec:1.1.5} to watch
this procedure in action computing $6!$, as shown in figure
\ref{fig:1.3}.

Now let's take a different perspective on computing factorials.  We
could describe a rule for computing $n!$ by specifying that we first
multiply 1 by 2, then multiply the result by 3, then by 4, and so on
until we reach $n$.  More formally, we maintain a running product,
together with a counter that counts from 1 up to $n$.  We can describe
the computation by saying that the counter and the product
simultaneously change from one step to the next according to the rule

\begin{eqnarray}
  product &  \leftarrow & counter \cdot product \\
  counter  & \leftarrow &  counter  +  1
\end{eqnarray}


and stipulating that $n!$ is the value of the product when
the counter exceeds $n$.

\begin{equation}
  % TODO: Apply caption
  \frac{1}{1 / R_1 + 1 / R_2}
  %\caption{A linear iterative procedures for computing 6!}
  \label{fig:1.4}
\end{equation}

\begin{schemeregion}
Once again, we can recast our description as a procedure for computing
factorials:\footnote{In a real program we would probably use the block
  structure introduced in the last section to hide the definition of
  \scheme|fact-iter|:
\begin{schemedisplay}
(define (factorial n)
  (define (iter product counter)
    (if (> counter n)
        product
        (iter (* counter product)
              (+ counter 1))))
  (iter 1 1))
\end{schemedisplay}

We avoided doing this here so as to minimize the number of things to
think about at once.}
\end{schemeregion}

\begin{schemedisplay}
(define (factorial n)
  (fact-iter 1 1 n))

(define (fact-iter product counter max-count)
  (if (> counter max-count)
      product
      (fact-iter (* counter product)
                 (+ counter 1)
                 max-count)))
\end{schemedisplay}

As before, we can use the substitution model to visualize the process
of computing $6!$, as shown in figure \ref{fig:1.4}.

Compare the two processes.  From one point of view, they seem hardly
different at all.  Both compute the same mathematical function on the
same domain, and each requires a number of steps proportional to $n$
to compute $n!$.  Indeed, both processes even carry out the same
sequence of multiplications, obtaining the same sequence of partial
products.  On the other hand, when we consider the ``shapes'' of the
two processes, we find that they evolve quite differently.

Consider the first process.  The substitution model reveals a shape of
expansion followed by contraction, indicated by the arrow in figure
\ref{fig:1.3}.  The expansion occurs as the process builds up a chain
of \textit{deferred operations} (in this case, a chain of
multiplications).  The contraction occurs as the operations are
actually performed.  This type of process, characterized by a chain of
deferred operations, is called a \textit{recursive process}.  Carrying
out this process requires that the interpreter keep track of the
operations to be performed later on.  In the computation of
$n!$, the length of the chain of deferred multiplications, and
hence the amount of information needed to keep track of it, grows
linearly with $n$ (is proportional to $n$), just like
the number of steps.  Such a process is called a \textit{linear
  recursive process}.

By contrast, the second process does not grow and shrink.  At each
step, all we need to keep track of, for any $n$, are the current
values of the variables \scheme|product|, \scheme|counter|, and
\scheme|max-count|.  We call this an \textit{iterative process}.  In
general, an iterative process is one whose state can be summarized by
a fixed number of \textit{state variables}, together with a fixed rule
that describes how the state variables should be updated as the
process moves from state to state and an (optional) end test that
specifies conditions under which the process should terminate.  In
computing $n!$, the number of steps required grows linearly with $N$.
Such a process is called a \textit{linear iterative process}.

The contrast between the two processes can be seen in another way.  In
the iterative case, the program variables provide a complete
description of the state of the process at any point.  If we stopped
the computation between steps, all we would need to do to resume the
computation is to supply the interpreter with the values of the three
program variables.  Not so with the recursive process.  In this case
there is some additional ``hidden'' information, maintained by the
interpreter and not contained in the program variables, which
indicates ``where the process is'' in negotiating the chain of
deferred operations.  The longer the chain, the more information must
be maintained.\footnote{When we discuss the implementation of
  procedures on register machines in chapter \ref{chap:5}, we will see
  that any iterative process can be realized ``in hardware'' as a
  machine that has a fixed set of registers and no auxiliary memory.
  In contrast, realizing a recursive process requires a machine that
  uses an auxiliary data structure known as a \textit{stack}.}

In contrasting iteration and recursion, we must be careful not to
confuse the notion of a recursive \textit{process} with the notion of
a recursive \textit{procedure}.  When we describe a procedure as
recursive, we are referring to the syntactic fact that the procedure
definition refers (either directly or indirectly) to the procedure
itself.  But when we describe a process as following a pattern that
is, say, linearly recursive, we are speaking about how the process
evolves, not about the syntax of how a procedure is written.  It may
seem disturbing that we refer to a recursive procedure such as
\scheme|fact-iter| as generating an iterative process.  However, the
process really is iterative: Its state is captured completely by its
three state variables, and an interpreter need keep track of only
three variables in order to execute the process.

One reason that the distinction between process and procedure may be
confusing is that most implementations of common languages (including
Ada, Pascal, and C) are designed in such a way that the interpretation
of any recursive procedure consumes an amount of memory that grows
with the number of procedure calls, even when the process described
is, in principle, iterative.  As a consequence, these languages can
describe iterative processes only by resorting to special-purpose
``looping constructs'' such as \scheme|do|, \scheme|repeat|,
\scheme|until|, \scheme|for|, and \scheme|while|.  The implementation
of Scheme we shall consider in chapter 5 does not share this defect.
It will execute an iterative process in constant space, even if the
iterative process is described by a recursive procedure.  An
implementation with this property is called \textit{tail-recursive}.
With a tail-recursive implementation, iteration can be expressed using
the ordinary procedure call mechanism, so that special iteration
constructs are useful only as syntactic sugar.\footnote{Tail recursion
  has long been known as a compiler optimization trick.  A coherent
  semantic basis for tail recursion was provided by Carl Hewitt
  (1977), who explained it in terms of the ``message-passing'' model
  of computation that we shall discuss in chapter 3. Inspired by this,
  Gerald Jay Sussman and Guy Lewis Steele Jr. (see Steele 1975)
  constructed a tail-recursive interpreter for Scheme.  Steele later
  showed how tail recursion is a consequence of the natural way to
  compile procedure calls (Steele 1977).  The IEEE standard for Scheme
  requires that Scheme implementations be tail-recursive.}

\begin{Exercise}
\label{exc:1.9}
Each of the following two procedures defines a method for adding two
positive integers in terms of the procedures \scheme|inc|,
which increments its argument by 1, and \scheme|dec|, which decrements
its argument by 1.

\begin{schemedisplay}
(define (+ a b)
  (if (= a 0)
      b
      (inc (+ (dec a) b))))

(define (+ a b)
  (if (= a 0)
      b
      (+ (dec a) (inc b))))
\end{schemedisplay}

Using the substitution model, illustrate the process generated by each
procedure in evaluating \scheme|(+ 4 5)|.  Are these processes
iterative or recursive?
\end{Exercise}

\begin{Exercise}
\label{exc:1.10}
The following procedure computes a mathematical function called
Ackermann's function.

\begin{schemedisplay}
(define (A x y)
  (cond ((= y 0) 0)
        ((= x 0) (* 2 y))
        ((= y 1) 2)
        (else (A (- x 1)
                 (A x (- y 1))))))
\end{schemedisplay}
What are the values of the following expressions?

\begin{schemedisplay}
(A 1 10)

(A 2 4)

(A 3 3)
\end{schemedisplay}

Consider the following procedures, where \scheme|A| is the procedure  
defined above:
\begin{schemedisplay}
(define (f n) (A 0 n))

(define (g n) (A 1 n))

(define (h n) (A 2 n))

(define (k n) (* 5 n n))
\end{schemedisplay}

Give concise mathematical definitions for the functions computed by
the procedures \scheme|f|, \scheme|g|, and \scheme|h| for positive integer
values of $n$.  For example, \scheme|(k n)| computes $5n^2$
\end{Exercise}

\subsection{Tree Recursion}
\label{sec:1.2.2}



Another common pattern of computation is called \textit{tree
  recursion}.  As an example, consider computing the sequence of
Fibonacci numbers, in which each number is the sum of the preceding
two:

\begin{displaymath}
0, 1, 1, 2, 3, 5, 8, 13, 21, 34
\end{displaymath}

In general, the Fibonacci numbers can be defined by the rule
\begin{displaymath}
  {\rm Fib}(n) =
  \begin{cases}
    0 & {\rm if}~ n = 0 \\
    1 & {\rm if} n = 1 \\
    {\rm Fib}(n-1) + {\rm Fib}(n-2) & {\rm otherwise}
  \end{cases}
\end{displaymath}
We can immediately translate this definition into a recursive
procedure for computing Fibonacci numbers:

\begin{schemedisplay}
(define (fib n)
  (cond ((= n 0) 0)
        ((= n 1) 1)
        (else (+ (fib (- n 1))
                 (fib (- n 2))))))
\end{schemedisplay}

\begin{figure}
  % TODO: Add arrows to diagram (ch1-Z-G-13.gif)
\begin{tikzpicture}
    [level 1/.style={sibling distance=15em},
     level 2/.style={sibling distance=4em}]
    \node {fib 5}
      child { node {fib 4}
        child { node {fib 3} 
          child { node {fib 2}
            child { node {fib 1}
              child { node {1} } }
            child { node {fib 0} 
              child { node {0} } } }
          child { node {fib 1} 
            child { node {1} } } }
        child[missing] {}
        child { node {fib 2}
          child { node {fib 1}
            child { node {1} } }
          child { node {fib 0} 
            child { node {0} } } } }
      child { node {fib 3} 
        child { node {fib 2}
          child { node {fib 1}
            child { node {1} } }
          child { node {fib 0} 
            child { node {0} } } }
        child { node {fib 1} 
          child { node {1} } } } ;
\end{tikzpicture}
\caption{The tree-recursive proccess generated in computing (fib 5).}
\label{fig:1.5}
\end{figure}

Consider the pattern of this computation.  To compute \scheme|(fib 5)|,
we compute \scheme|(fib 4)| and \scheme|(fib 3)|.  To compute \scheme|(fib 4)|,
we compute \scheme|(fib 3)| and \scheme|(fib 2)|.  In general, the evolved
process looks like a tree, as shown in figure \ref{fig:1.5}.
Notice that the branches split into two at each level (except at the
bottom); this reflects the fact that the \scheme|fib| procedure calls
itself twice each time it is invoked.


This procedure is instructive as a prototypical tree recursion, but it
is a terrible way to compute Fibonacci numbers because it does so much
redundant computation.  Notice in figure \ref{fig:1.5} that the entire
computation of \scheme|(fib 3)| -- almost half the work -- is
duplicated.  In fact, it is not hard to show that the number of times
the procedure will compute \scheme|(fib 1)| or \scheme|(fib 0)| (the
number of leaves in the above tree, in general) is precisely ${\rm
  Fib}(n+1)$.  To get an idea of how bad this is, one can show that
the value of ${\rm Fib}(n)$ grows exponentially with $n$.  More
precisely (see exercise \ref{exc:1.13}), ${\rm Fib}(n)$ is the closest
% TODO: Verfy that these equations are in the right order.
integer to $\phi^n$, where $$\phi = \frac{1 + \sqrt 5}{2} \approx
1.6180$$ is the \textit{golden ratio}, which satisfies the
equation $$\phi^2 = \phi+1$$

Thus, the process uses a number of steps that grows exponentially with
the input.  On the other hand, the space required grows only linearly
with the input, because we need keep track only of which nodes are
above us in the tree at any point in the computation.  In general, the
number of steps required by a tree-recursive process will be
proportional to the number of nodes in the tree, while the space
required will be proportional to the maximum depth of the tree.

We can also formulate an iterative process for computing the
Fibonacci numbers.  The idea is to use a pair of integers $a$ and
$b$, initialized to  ${\rm Fib}(1) = 1$ and ${\rm Fib}(0) = 0$,
and to repeatedly apply the simultaneous
transformations
\begin{eqnarray*}
a & \leftarrow & a + b \\
b & \leftarrow & a
\end{eqnarray*}
It is not hard to show that, after applying this transformation $n$
times, $a$ and $b$ will be equal, respectively, to ${\rm Fib}(n+1)$
and ${\rm Fib}(n)$.  Thus, we can compute Fibonacci numbers
iteratively using the procedure

\begin{schemedisplay}
(define (fib n)
  (fib-iter 1 0 n))

(define (fib-iter a b count)
  (if (= count 0)
      b
      (fib-iter (+ a b) a (- count 1))))
\end{schemedisplay}

This second method for computing ${\rm Fib}(n)$ is a linear iteration.
The difference in number of steps required by the two methods -- one
linear in $n$, one growing as fast as ${\rm Fib}(n)$ itself -- is
enormous, even for small inputs.

One should not conclude from this that tree-recursive processes are
useless.  When we consider processes that operate on hierarchically
structured data rather than numbers, we will find that tree recursion
is a natural and powerful tool.\footnote{An example of this was hinted
  at in section \ref{sec:1.1.3}: The interpreter itself evaluates
  expressions using a tree-recursive process} But even in numerical
operations, tree-recursive processes can be useful in helping us to
understand and design programs.  For instance, although the first
\scheme|fib| procedure is much less efficient than the second one, it
is more straightforward, being little more than a translation into
Lisp of the definition of the Fibonacci sequence.  To formulate the
iterative algorithm required noticing that the computation could be
recast as an iteration with three state variables.

\subsubsection*{Example: Counting change}

It takes only a bit of cleverness to come up with the iterative
Fibonacci algorithm.  In contrast, consider the following problem: How
many different ways can we make change of \$1.00, given half-dollars,
quarters, dimes, nickels, and pennies?  More generally, can we write a
procedure to compute the number of ways to change any given amount of
money?

This problem has a simple solution as a recursive procedure.  Suppose
we think of the types of coins available as arranged in some order.
Then the following relation holds:

The number of ways to change amount $a$ using $n$ kinds of coins equals

\begin{itemize}
\item the number of ways to change amount $a$ using all but the first
  kind of coin, plus

\item the number of ways to change amount $a - d$ using all $n$ kinds
  of coins, where $d$ is the denomination of the first kind of coin.
\end{itemize}

To see why this is true, observe that the ways to make change can be
divided into two groups: those that do not use any of the first kind
of coin, and those that do.  Therefore, the total number of ways to
make change for some amount is equal to the number of ways to make
change for the amount without using any of the first kind of coin,
plus the number of ways to make change assuming that we do use the
first kind of coin.  But the latter number is equal to the number of
ways to make change for the amount that remains after using a coin of
the first kind.

Thus, we can recursively reduce the problem of changing a given amount
to the problem of changing smaller amounts using fewer kinds of coins.
Consider this reduction rule carefully, and convince yourself that we
can use it to describe an algorithm if we specify the following
degenerate cases:\footnote{For example, work through in detail how the
  reduction rule applies to the problem of making change for 10 cents
  using pennies and nickels.}

\begin{itemize}
\item If $a$ is exactly 0, we should count that as 1 way to make change.

\item If $a$ is less than 0, we should count that as 0 ways to make change.

\item If $n$ is 0, we should count that as 0 ways to make change.
\end{itemize}

We can easily translate this description into a recursive
procedure:

\begin{schemedisplay}
(define (count-change amount)
  (cc amount 5))
(define (cc amount kinds-of-coins)
  (cond ((= amount 0) 1)
        ((or (< amount 0) (= kinds-of-coins 0)) 0)
        (else (+ (cc amount
                     (- kinds-of-coins 1))
                 (cc (- amount
                        (first-denomination kinds-of-coins))
                     kinds-of-coins)))))
(define (first-denomination kinds-of-coins)
  (cond ((= kinds-of-coins 1) 1)
        ((= kinds-of-coins 2) 5)
        ((= kinds-of-coins 3) 10)
        ((= kinds-of-coins 4) 25)
        ((= kinds-of-coins 5) 50)))
\end{schemedisplay}
(The \scheme|first-denomination| procedure takes as input the number of
kinds of coins available and returns the denomination of the first
kind.  Here we are thinking of the coins as arranged in order from
largest to smallest, but any order would do as well.)  We can now
answer our original question about changing a dollar:

\begin{schemedisplay}
> (count-change 100)
292
\end{schemedisplay}

\scheme|Count-change| generates a tree-recursive process with
redundancies similar to those in our first implementation of
\scheme|fib|.  (It will take quite a while for that 292 to be
computed.)  On the other hand, it is not obvious how to design a
better algorithm for computing the result, and we leave this problem
as a challenge.  The observation that a tree-recursive process may be
highly inefficient but often easy to specify and understand has led
people to propose that one could get the best of both worlds by
designing a ``smart compiler'' that could transform tree-recursive
procedures into more efficient procedures that compute the same
result.\footnote{One approach to coping with redundant computations
  is to arrange matters so that we automatically construct a table of
  values as they are computed.  Each time we are asked to apply the
  procedure to some argument, we first look to see if the value is
  already stored in the table, in which case we avoid performing the
  redundant computation.  This strategy, known as \textit{tabulation}
  or \textit{memoization}, can be implemented in a straightforward
  way.  Tabulation can sometimes be used to transform processes that
  require an exponential number of steps (such as
  \scheme|count-change|) into processes whose space and time
  requirements grow linearly with the input.  See exercise
  \ref{exc:3.27}.}

\begin{Exercise}
\label{exc:1.11}
A function $f$ is defined by the rule that $f(n) = n {\rm ~if~} n < 3
{\rm ~and~} f(n) = f(n-1) + 2f(n-2) + 3f(n-3) {\rm ~if~} n \geq 3$.
Write a procedure that computes $f$ by means of a recursive process.
Write a procedure that computes $f$ by means of an iterative process.
\end{Exercise}

\begin{Exercise}
\label{exc:1.12}
The following pattern of numbers is called
\textit{Pascal's triangle}.

\begin{center}
\tt
\begin{tabular}{c}
  1 \\
  1 1 \\
  1 2 1 \\
1 3 3 1 \\
1 4 6 4 1
\end{tabular} 
\end{center}

The numbers at the edge of the triangle are all 1, and each number
inside the triangle is the sum of the two numbers above it.\footnote{
  The elements of Pascal's triangle are called the \textit{binomial
    coefficients}, because the \textit{n}th row consists of the
  coefficients of the terms in the expansion of $(x+y)^n$.  This
  pattern for computing the coefficients appeared in Blaise Pascal's
  1653 seminal work on probability theory, \textit{Trait\'e du
    triangle arithm\'etique}.  According to Knuth (1973), the same
  pattern appears in the \textit{Szu-yuen Y\"u-chien} (``The Precious
  Mirror of the Four Elements''), published by the Chinese
  mathematician Chu Shih-chieh in 1303, in the works of the
  twelfth-century Persian poet and mathematician Omar Khayyam, and in
  the works of the twelfth-century Hindu mathematician Bh\'ascara
  \'Ach\'arya.}

Write a procedure that computes elements of Pascal's triangle by means
of a recursive process.
\end{Exercise}

\begin{Exercise}
\label{exc:1.13}
Prove that ${\rm Fib}(n)$ is the closest integer to $\frac{\phi^n}{\sqrt 5}$,
where $\phi = \over{1 + \sqrt 5}{2}$  Hint: Let $\psi = \frac{1-\sqrt 5}{2}$.  Use
induction and the definition of the Fibonacci numbers (see
section \ref{sec:1.2.2}) to prove that ${\rm Fib}(n) = \frac{\phi^n - \psi^n}{\sqrt 5}$.
\end{Exercise}

\subsection{Orders of Growth}
\label{sec:1.2.3}

The previous examples illustrate that processes can differ
considerably in the rates at which they consume computational
resources.  One convenient way to describe this difference is to use
the notion of \textit{order of growth} to obtain a gross measure of the
resources required by a process as the inputs become larger.


Let $n$ be a parameter that measures the size of the problem, and let
$R(n)$ be the amount of resources the process requires for a problem
of size $n$.  In our previous examples we took $n$ to be the number
for which a given function is to be computed, but there are other
possibilities.  For instance, if our goal is to compute an
approximation to the square root of a number, we might take $n$ to be
the number of digits accuracy required.  For matrix multiplication we
might take $n$ to be the number of rows in the matrices.  In general
there are a number of properties of the problem with respect to which
it will be desirable to analyze a given process.  Similarly, $R(n)$
might measure the number of internal storage registers used, the
number of elementary machine operations performed, and so on.  In
computers that do only a fixed number of operations at a time, the
time required will be proportional to the number of elementary machine
operations performed.

We say that $R(n)$ has order of growth $\Theta(f(n))$ written $R(n) = \Theta(f(n))$) (pronounced ``theta of $f(n)$''), if there are
positive constants $k_1$ and $k_2$ independent of $n$ such that
$$k_1 f(n) \le R(N) \le k_2 f(n)$$
for any sufficiently large value of $n$.  (In other
words, for large $n$, the value $R(n)$) is sandwiched between $k_1f(n)$
and $k_2f(n)$.)

For instance, with the linear recursive process for computing
factorial described in section \ref{sec:1.2.1} the
number of steps grows proportionally to the input $n$.  Thus, the
steps required for this process grows as $\Theta(n)$.  We also saw
that the space required grows as $\Theta(n)$.  For the iterative
factorial, the number of steps is still $\Theta(n)$ but the space is
$\Theta(1)$ -- that is, constant.\footnote{These statements mask a
great deal of oversimplification.  For instance, if we count process
steps as ``machine operations'' we are making the assumption that the
number of machine operations needed to perform, say, a multiplication
is independent of the size of the numbers to be multiplied, which is
false if the numbers are sufficiently large.  Similar remarks hold for
the estimates of space.  Like the design and description of a process,
the analysis of a process can be carried out at various levels of
abstraction.} The tree-recursive Fibonacci computation requires
$\Theta(\phi^n)$ steps and space $\Theta(n)$, where $\phi$ is the
golden ratio described in section \ref{sec:1.2.2}.


Orders of growth provide only a crude description of the behavior of a
process.  For example, a process requiring $n^2$ steps and a process
requiring $1000n^2$ steps and a process requiring $3n^2 + 10n + 17$
steps all have $\Theta(n^2)$ order of growth.  On the other hand,
order of growth provides a useful indication of how we may expect the
behavior of the process to change as we change the size of the
problem.  For a $\Theta(n)$ (linear) process, doubling the size will
roughly double the amount of resources used.  For an exponential
process, each increment in problem size will multiply the resource
utilization by a constant factor.  In the remainder of section
\ref{sec:1.2} we will examine two algorithms whose order of growth is
logarithmic, so that doubling the problem size increases the resource
requirement by a constant amount.

\begin{Exercise}
\label{exc:1.14}
Draw the tree illustrating the process generated by the \scheme|count-change| procedure of section \ref{sec:1.2.2} in making
change for 11 cents.  What are the orders of growth of the space and
number of steps used by this process as the amount to be changed
increases?
\end{Exercise}

\begin{Exercise}
\label{exc:1.15}
The sine of an angle (specified in radians) can be computed by making
use of the approximation $\sin x \approx x$ if $x$ is sufficiently small, and the trigonometric
identity $$\sin r = 3 \sin \frac{r}{3} - 4 \sin^3 \frac{r}{3}$$
to reduce the size of the argument of $sin$.  (For purposes of
this exercise an angle is considered ``sufficiently small'' if its
magnitude is not greater than 0.1 radians.) These ideas are
incorporated in the following procedures:

\begin{schemedisplay}
(define (cube x) (* x x x))
(define (p x) (- (* 3 x) (* 4 (cube x))))
(define (sine angle)
   (if (not (> (abs angle) 0.1))
       angle
       (p (sine (/ angle 3.0)))))
\end{schemedisplay}
% TODO: use enumerate section
a.  How many times is the procedure \scheme|p| 
applied when \scheme|(sine 12.15)| is evaluated?

b.  What is the order of growth in space and number of steps (as a
function of $a$) used by the process generated by the \scheme|sine|
procedure when \scheme|(sine a)| is evaluated?
\end{Exercise}

\subsection{Exponentiation}
\label{sec:1.2.4}

Consider the problem of computing the exponential of a given number.
We would like a procedure that takes as arguments a base $b$ and a
positive integer exponent $n$ and computes $b^n$.  One way to do this
is via the recursive definition
\begin{eqnarray*}
b^n & = & b \cdot b^{n-1} \\
b^0 & = & 1 
\end{eqnarray*}
which translates readily into the procedure 

\begin{schemedisplay}
(define (expt b n)
  (if (= n 0)
      1
      (* b (expt b (- n 1)))))
\end{schemedisplay}
This is a linear recursive process, which requires $\Theta(n)$ steps
and $\Theta(n)$ space.  Just as with factorial, we can readily
formulate an equivalent linear iteration:

\begin{schemedisplay}
(define (expt b n)
  (expt-iter b n 1))

(define (expt-iter b counter product)
  (if (= counter 0)
      product
      (expt-iter b
                (- counter 1)
                (* b product)))) 
\end{schemedisplay}
This version requires $\Theta(n)$ steps and $\Theta(1)$ space.

We can compute exponentials in fewer steps by using successive
squaring.  For instance, rather than computing $b^8$ as
$$b \cdot (b \cdot (b \cdot (b \cdot (b \cdot (b \cdot (b \cdot b))))))$$
we can compute it using three multiplications:
\begin{eqnarray*}
b^2 & = & b \cdot b \\
b^4 & = & b^2 \cdot b^2 \\
b^8 & = & b^4 \cdot b^4 
\end{eqnarray*}

This method works fine for exponents that are powers of 2.  We can
also take advantage of successive squaring in computing exponentials
in general if we use the rule

\begin{displaymath}
  b^n =
  \begin{cases}
    \big(b^{b/2}\big)^2 & {\rm if~} n {\rm ~is~even} \\
    b \cdot b^{n-1} & {\rm if~} n {\rm ~is~odd}
  \end{cases}
\end{displaymath}
We can express this method as a procedure:


\begin{schemedisplay}
(define (fast-expt b n)
  (cond ((= n 0) 1)
        ((even? n) (square (fast-expt b (/ n 2))))
        (else (* b (fast-expt b (- n 1))))))
\end{schemedisplay}
where the predicate to test whether an integer is even is defined in terms of
the primitive procedure \scheme|remainder| by

\begin{schemedisplay}
(define (even? n)
  (= (remainder n 2) 0))
\end{schemedisplay}
The process evolved by \scheme|fast-expt| grows logarithmically with
$n$ in both space and number of steps.  To see this, observe that
computing $b^{2n}$ using \scheme|fast-expt| requires only one more
multiplication than computing $b^n$.  The size of the exponent we can
compute therefore doubles (approximately) with every new
multiplication we are allowed.  Thus, the number of multiplications
required for an exponent of $n$ grows about as fast as the logarithm
of $n$ to the base 2.  The process has $\Theta(\log n)$
growth.\footnote{More precisely, the number of multiplications
  required is equal to 1 less than the log base 2 of $n$ plus the
  number of ones in the binary representation of $n$.  This total is
  always less than twice the log base 2 of $n$.  The arbitrary
  constants $k_1$ and $k_2$ in the definition of order notation imply
  that, for a logarithmic process, the base to which logarithms are
  taken does not matter, so all such processes are described as
  $\Theta(\log n)$.}

The difference between $\Theta(\log n)$ growth and $\Theta(n)$ growth
becomes striking as $n$ becomes large.  For example,
\scheme|fast-expt| for $n = 1000$ requires only 14
multiplications.\footnote{You may wonder why anyone would care about
  raising numbers to the 1000th power.  See section \ref{sec:1.2.6}.}
It is also possible to use the idea of successive squaring to devise
an iterative algorithm that computes exponentials with a logarithmic
number of steps (see exercise \ref{exc:1.16}), although, as is often
the case with iterative algorithms, this is not written down so
straightforwardly as the recursive algorithm.\footnote{This iterative
  algorithm is ancient.  It appears in the \textit{Chandah-sutra} by
  \'Ach\'arya Pingala, written before 200 \textsc{b.c}.  See Knuth
  1981, section 4.6.3, for a full discussion and analysis of this and
  other methods of exponentiation.}



\begin{Exercise}
\label{exc:1.16}
Design a procedure that evolves an iterative exponentiation process
that uses successive squaring and uses a logarithmic number of steps,
as does \scheme|fast-expt|.  (Hint: Using the observation that
$(b^{n/2})^2 = (b^2)^{n/2}$, keep, along with the exponent $n$ and the
base $b$, an additional state variable $a$, and define the state
transformation in such a way that the product $a~b ^n$ is unchanged
from state to state.  At the beginning of the process $a$ is taken to
be 1, and the answer is given by the value of $a$ at the end of the
process.  In general, the technique of defining an \textit{invariant
  quantity} that remains unchanged from state to state is a powerful
way to think about the design of iterative algorithms.)
\end{Exercise}

\begin{Exercise}
\label{exc:1.17}
The exponentiation algorithms in this section are based on performing
exponentiation by means of repeated multiplication.  In a similar way,
one can perform integer multiplication by means of repeated addition.
The following multiplication procedure (in which it is assumed that
our language can only add, not multiply) is analogous to the \scheme|expt| procedure:

\begin{schemedisplay}
(define (* a b)
  (if (= b 0)
      0
      (+ a (* a (- b 1)))))
\end{schemedisplay}

This algorithm takes a number of steps that is linear in \scheme|b|.
Now suppose we include, together with addition, operations \scheme|double|,
which doubles an integer, and \scheme|halve|, which divides an (even)
integer by 2.  Using these, design a multiplication procedure analogous
to \scheme|fast-expt| that uses a logarithmic number of steps. 
\end{Exercise}

\begin{Exercise}
\label{exc:1.18}
Using the results of exercises \ref{exc:1.16} and \ref{exc:1.17},
devise a procedure that generates an iterative process for multiplying
two integers in terms of adding, doubling, and halving and uses a
logarithmic number of steps.\footnote{This algorithm, which is
  sometimes known as the ``Russian peasant method'' of multiplication,
  is ancient.  Examples of its use are found in the Rhind Papyrus, one
  of the two oldest mathematical documents in existence, written about
  1700 \textsc{b.c}. (and copied from an even older document) by an
  Egyptian scribe named A'h-mose.} 
\end{Exercise}

\begin{Exercise}
\label{exc:1.19}
There is a clever algorithm for computing the Fibonacci numbers in a
logarithmic number of steps.  Recall the transformation of the state
variables $a$ and $b$ in the \scheme|fib-iter| process of section
\ref{sec:1.2.2}: $a \leftarrow a + b$ and $b \leftarrow a$.  Call this
transformation $T$, and observe that applying $T$ over and over again
$n$ times, starting with 1 and 0, produces the pair ${\rm Fib}(n + 1)$
and ${\rm Fib}(n)$.  In other words, the Fibonacci numbers are produced by
applying $T^n$, the $n$th power of the transformation $T$, starting
with the pair $(1,0)$.  Now consider $T$ to be the special case of $p
= 0$ and $q = 1$ in a family of transformations $T_{pq}$, where
$T_{pq}$ transforms the pair $(a,b)$ according to $a \leftarrow bq +
aq + ap$ and $b \leftarrow bp + aq$.  Show that if we apply such a
transformation $T_{pq}$ twice, the effect is the same as using a
single transformation $T_{p'q'}$ of the same form, and compute $p'$
and $q'$ in terms of $p$ and $q$.  This gives us an explicit way to
square these transformations, and thus we can compute $T^n$ using
successive squaring, as in the \scheme|fast-expt| procedure.  Put this
all together to complete the following procedure, which runs in a
logarithmic number of steps:\footnote{This exercise was
suggested to us by Joe Stoy, based on an example in Kaldewaij 1990.}

\begin{schemedisplay}
(define (fib n)
  (fib-iter 1 0 0 1 n))
(define (fib-iter a b p q count)
  (cond ((= count 0) b)
        ((even? count)
         (fib-iter a
                   b
                   <??>      ; compute $p'$
                   <??>      ; compute $q'$
                   (/ count 2)))
        (else (fib-iter (+ (* b q) (* a q) (* a p))
                        (+ (* b p) (* a q))
                        p
                        q
                        (- count 1)))))
\end{schemedisplay}
\end{Exercise}

\subsection{Greatest Common Divisors}
\label{sec:1.2.5}

The greatest common divisor (GCD) of two integers $a$ and $b$ is
defined to be the largest integer that divides both $a$ and
$b$ with no remainder.  For example, the GCD of 16 and 28 is 4.  In chapter 2,
when we investigate how to implement rational-number arithmetic, we
will need to be able to compute GCDs in order to reduce
rational numbers to lowest terms.  (To reduce a rational number to
lowest terms, we must divide both the numerator and the denominator by their
GCD.  For example, 16/28 reduces to 4/7.)  One way to find the
GCD of two integers is to factor them and search for common
factors, but there is a famous algorithm that is much more efficient.

The idea of the algorithm is based on the observation that, if $r$ is
the remainder when $a$ is divided by $b$, then the common divisors of
$a$ and $b$ are precisely the same as the common divisors of $b$ and
$r$.  Thus, we can use the equation $${\rm GCD}(a,b) = {\rm GCD}(b,r)$$
to successively reduce the problem of computing a GCD to the
problem of computing the GCD of smaller and smaller pairs of
integers.  For example,
\begin{eqnarray*}
 {\rm GCD}(206,40) & = & {\rm GCD}(40,6) \\
 & = & {\rm GCD}(6,4) \\
 & = & {\rm GCD}(4,2) \\
 & = & {\rm GCD}(2,0) \\
 & = & 2
\end{eqnarray*}
reduces GCD(206,40) to GCD(2,0), which is 2.  It is possible to show
that starting with any two positive integers and performing repeated
reductions will always eventually produce a pair where the second
number is 0.  Then the GCD is the other number in the pair.  This
method for computing the GCD is known as \textit{Euclid's
  Algorithm}.\footnote{Euclid's Algorithm is so called because it
  appears in Euclid's $Elements$ (Book 7, ca. 300 \textsc{b.c}.).
  According to Knuth (1973), it can be considered the oldest known
  nontrivial algorithm.  The ancient Egyptian method of multiplication
  (exercise \ref{exc:1.18}) is surely older, but, as Knuth explains,
  Euclid's algorithm is the oldest known to have been presented as a
  general algorithm, rather than as a set of illustrative examples.}

It is easy to express Euclid's Algorithm as a procedure:
\begin{schemedisplay}
(define (gcd a b)
  (if (= b 0)
      a
      (gcd b (remainder a b))))
\end{schemedisplay}
This generates an iterative process, whose number of steps grows as
the logarithm of the numbers involved.

The fact that the number of steps required by Euclid's Algorithm has
logarithmic growth bears an interesting relation to the Fibonacci
numbers:

\begin{theorem}[Lam\'e's Theorem]
  If Euclid's Algorithm requires $k$ steps to compute the GCD of some
  pair, then the smaller number in the pair must be greater than or
  equal to the $k$th Fibonacci number.\footnote{This theorem was
    proved in 1845 by Gabriel Lam\'e, a French mathematician and
    engineer known chiefly for his contributions to mathematical
    physics.  To prove the theorem, we consider pairs $(a_{k}
    ,b_{k})$, where $a_k \ge b_k$, for which Euclid's Algorithm
    terminates in $k$ steps.  The proof is based on the claim that, if
    $(a_{k+1}, b_{k+1}) \mapsto (a_k, b_k) \mapsto (a_{k-1},
    b_{k-1})$ are three successive pairs in the reduction process,
    then we must have $b_{k+1} \ge b_k + b_{k-1}$.  To verify the
    claim, consider that a reduction step is defined by applying the
    transformation $a_{k-1} = b_k$, $b_{k-1} = {\rm remainder~of~} a_k
    {\rm ~divided~by~} b_k$.  The second equation means that $a_k =
    qb_k + b_{k-1}$ for some positive integer $q$.  And since $q$ must
    be at least 1 we have $a_k = qb_k + b_{k-1} \ge b_k + b_{k-1}$.
    But in the previous reduction step we have $b_{k+1} = a_k$.
    Therefore, $b_{k+1} = a_k \ge b_k + b_{k-1}$.  This verifies the
    claim.  Now we can prove the theorem by induction on $k$, the
    number of steps that the algorithm requires to terminate.  The
    result is true for $k = 1$, since this merely requires that $b$ be
    at least as large as ${\rm Fib}(1) = 1$.  Now, assume that the
    result is true for all integers less than or equal to $k$ and
    establish the result for $k + 1$.  Let $(a_{k+1}, b_{k+1})
    \mapsto (a_k, b_k) \mapsto (a_{k-1}, b_{k-1})$ be
    successive pairs in the reduction process.  By our induction
    hypotheses, we have $b_{k-1} \ge {\rm Fib}(k - 1)$ and $b_k \ge
    {\rm Fib}(k)$.  Thus, applying the claim we just proved together
    with the definition of the Fibonacci numbers gives $b_{k+1} \ge
    b_k + b_{k-1} \ge {\rm Fib}(k) + {\rm Fib}(k - 1) = {\rm
      Fib}(k+1)$, which completes the proof of Lam\'e's Theorem.}
\end{theorem}



We can use this theorem to get an order-of-growth estimate for Euclid's
Algorithm.  Let $n$ be the smaller of the two inputs to the
procedure.  If the process takes $k$ steps, then we must have 
$n \ge {\rm Fib}(k) \approx \frac{\phi^k}{\sqrt 5}$.  Therefore
the number of steps $k$ grows as the logarithm (to the base
$\phi$) of $n$.  Hence, the order of growth is $\Theta(\log n)$.

\begin{Exercise}
\label{exc:1.20}
The process that a procedure generates is of course dependent on the
rules used by the interpreter.  As an example, consider the iterative
\scheme|gcd| procedure given above.  Suppose we were to interpret this
procedure using normal-order evaluation, as discussed in section
\ref{sec:1.1.5}.  (The normal-order-evaluation rule for \scheme|if| is
described in exercise \ref{exc:1.5}.)  Using the substitution method
(for normal order), illustrate the process generated in evaluating
\scheme|(gcd 206 40)| and indicate the \scheme|remainder| operations
that are actually performed.  How many \scheme|remainder| operations
are actually performed in the normal-order evaluation of \scheme|(gcd
206 40)|?  In the applicative-order evaluation?
\end{Exercise}

\subsection{Example: Testing for Primality}
\label{sec:1.2.6}



This section describes two methods for checking the primality of an
integer $n$, one with order of growth $\Theta(\sqrt n)$, and a
``probabilistic'' algorithm with order of growth $\Theta(\log n)$.
The exercises at the end of this section suggest programming projects
based on these algorithms.

\subsubsection*{Searching for divisors}

Since ancient times, mathematicians have been fascinated by problems
concerning prime numbers, and many people have worked on the problem
of determining ways to test if numbers are prime.  One way
to test if a number is prime is to find the number's divisors.  The
following program finds the smallest integral divisor (greater than 1)
of a given number $n$.  It does this in a straightforward way, by
testing $n$ for divisibility by successive integers starting with 2.

\begin{schemedisplay}
(define (smallest-divisor n)
  (find-divisor n 2))
(define (find-divisor n test-divisor)
  (cond ((> (square test-divisor) n) n)
        ((divides? test-divisor n) test-divisor)
        (else (find-divisor n (+ test-divisor 1)))))
(define (divides? a b)
  (= (remainder b a) 0))
\end{schemedisplay}

We can test whether a number is prime as follows: $n$ is prime if
and only if $n$ is its own smallest divisor.

\begin{schemedisplay}
(define (prime? n)
  (= n (smallest-divisor n)))
\end{schemedisplay}

The end test for \scheme|find-divisor| is based on the fact that if
$n$ is not prime it must have a divisor less than or equal to $\sqrt
n$.\footnote{If $d$ is a divisor of $n$, then so is $n/d$.  But $d$
  and $n/d$ cannot both be greater than $\sqrt n$} This means that the
algorithm need only test divisors between 1 and $\sqrt n$.
Consequently, the number of steps required to identify $n$ as prime
will have order of growth $\Theta(\sqrt n)$.

\subsubsection*{The Fermat test}

The $\Theta(\log n)$ primality test is based on a result from number
theory known as Fermat's Little Theorem.\footnote{Pierre de Fermat
  (1601-1665) is considered to be the founder of modern number theory.
  He obtained many important number-theoretic results, but he usually
  announced just the results, without providing his proofs.  Fermat's
  Little Theorem was stated in a letter he wrote in 1640.  The first
  published proof was given by Euler in 1736 (and an earlier,
  identical proof was discovered in the unpublished manuscripts of
  Leibniz).  The most famous of Fermat's results -- known as Fermat's
  Last Theorem -- was jotted down in 1637 in his copy of the book
  \textit{Arithmetic} (by the third-century Greek mathematician
  Diophantus) with the remark ``I have discovered a truly remarkable
  proof, but this margin is too small to contain it.''  Finding a
  proof of Fermat's Last Theorem became one of the most famous
  challenges in number theory.  A complete solution was finally given
  in 1995 by Andrew Wiles of Princeton University.  }

\begin{theorem}[Fermat's Little Theorem]
  If $n$ is a prime number and $a$ is any positive integer less than
  $n$, then $a$ raised to the $n$th power is congruent to $a$ modulo
  $n$.
\end{theorem}

(Two numbers are said to be \textit{congruent modulo} $n$ if
they both have the same remainder when divided by $n$.  The
remainder of a number $a$ when divided by $n$ is also referred to as
the \textit{remainder of} $a$ \textit{modulo} $n$, or simply as $a$ 
\textit{modulo} $n$.)

If $n$ is not prime, then, in general, most of the numbers $a < n$ will not
satisfy the above relation.  This leads to the following algorithm for
testing primality: Given a number $n$, pick a random number $a < n$ and
compute the remainder of $a^n$ modulo $n$.  If the result is not equal to
$a$, then $n$ is certainly not prime.  If it is $a$, then chances are good
that $n$ is prime.  Now pick another random number $a$ and test it with the
same method.  If it also satisfies the equation, then we can be even more
confident that $n$ is prime.  By trying more and more values of $a$, we can
increase our confidence in the result.  This algorithm is known as the
Fermat test.

To implement the Fermat test, we need a procedure that computes the
exponential of a number modulo another number:

\begin{schemedisplay}
(define (expmod base exp m)
  (cond ((= exp 0) 1)
        ((even? exp)
         (remainder (square (expmod base (/ exp 2) m))
                    m))
        (else
         (remainder (* base (expmod base (- exp 1) m))
                    m))))        
\end{schemedisplay}
This is very similar to the \scheme|fast-expt| procedure of section
\ref{sec:1.2.4}.  It uses successive squaring, so that the number of
steps grows logarithmically with the exponent.\footnote{The reduction
  steps in the cases where the exponent $e$ is greater than 1
  are based on the fact that, for any integers $x$, $y$,
  and $m$, we can find the remainder of $x$ times
  $y$ modulo $m$ by computing separately the remainders
  of $x$ modulo $m$ and $y$ modulo $m$,
  multiplying these, and then taking the remainder of the result
  modulo $m$.  For instance, in the case where $e$ is
  even, we compute the remainder of $b^{e/2}$
  modulo $m$, square this, and take the remainder modulo
  $m$.  This technique is useful because it means we can
  perform our computation without ever having to deal with numbers
  much larger than $m$.  (Compare exercise \ref{exc:1.25})}

The Fermat test is performed by choosing at random a number $a$
between $1$ and $n - 1$ inclusive and checking whether the remainder
modulo $n$ of the $n$th power of $a$ is equal to $a$.  The random
number $a$ is chosen using the procedure \scheme|random|, which we assume is
included as a primitive in Scheme. \scheme|Random| returns a
nonnegative integer less than its integer input.  Hence, to obtain a random
number between $1$ and $n - 1$, we call \scheme|random| with an input of
$n - 1$ and add 1 to the result:

\begin{schemedisplay}
(define (fermat-test n)
  (define (try-it a)
    (= (expmod a n n) a))
  (try-it (+ 1 (random (- n 1)))))
\end{schemedisplay}

The following procedure runs the test a given number of times, as
specified by a parameter.  Its value is true if the test succeeds
every time, and false otherwise.

\begin{schemedisplay}
(define (fast-prime? n times)
  (cond ((= times 0) true)
        ((fermat-test n) (fast-prime? n (- times 1)))
        (else false)))
\end{schemedisplay}

\subsubsection*{Probabilistic methods}

The Fermat test differs in character from most familiar algorithms, in
which one computes an answer that is guaranteed to be correct.  Here,
the answer obtained is only probably correct.  More precisely, if $n$
ever fails the Fermat test, we can be certain that $n$ is not prime.
But the fact that $n$ passes the test, while an extremely strong
indication, is still not a guarantee that $n$ is prime.  What we would
like to say is that for any number $n$, if we perform the test enough
times and find that $n$ always passes the test, then the probability
of error in our primality test can be made as small as we like.

Unfortunately, this assertion is not quite correct.  There do exist
numbers that fool the Fermat test: numbers $n$ that are not prime and
yet have the property that $a^n$ is congruent to $a$ modulo $n$ for
all integers $a < n$.  Such numbers are extremely rare, so the Fermat
test is quite reliable in practice.\footnote{\label{fn:1.47}Numbers
  that fool the Fermat test are called \textit{Carmichael numbers},
  and little is known about them other than that they are extremely
  rare.  There are 255 Carmichael numbers below 100,000,000.  The
  smallest few are 561, 1105, 1729, 2465, 2821, and 6601.  In testing
  primality of very large numbers chosen at random, the chance of
  stumbling upon a value that fools the Fermat test is less than the
  chance that cosmic radiation will cause the computer to make an
  error in carrying out a ``correct'' algorithm.  Considering an
  algorithm to be inadequate for the first reason but not for the
  second illustrates the difference between mathematics and
  engineering.}  There are variations of the Fermat test that cannot
be fooled.  In these tests, as with the Fermat method, one tests the
primality of an integer $n$ by choosing a random integer $a < n$ and
checking some condition that depends upon $n$ and $a$.  (See exercise
\ref{exc:1.28} for an example of such a test.)  On the other hand, in
contrast to the Fermat test, one can prove that, for any $n$, the
condition does not hold for most of the integers $a < n$ unless $n$ is
prime.  Thus, if $n$ passes the test for some random choice of $a$,
the chances are better than even that $n$ is prime.  If $n$ passes the
test for two random choices of $a$, the chances are better than 3 out
of 4 that $n$ is prime. By running the test with more and more
randomly chosen values of $a$ we can make the probability of error as
small as we like.

The existence of tests for which one can prove that the chance of
error becomes arbitrarily small has sparked interest in algorithms of
this type, which have come to be known as \idef{probabilistic
  algorithms}.  There is a great deal of research activity in this
area, and probabilistic algorithms have been fruitfully applied to
many fields.\footnote{One of the most striking applications of
  probabilistic prime testing has been to the field of cryptography.
  Although it is now computationally infeasible to factor an arbitrary
  200-digit number, the primality of such a number can be checked in a
  few seconds with the Fermat test.  This fact forms the basis of a
  technique for constructing ``unbreakable codes'' suggested by
  Rivest, Shamir, and Adleman (1977).  The resulting \textit{RSA
    algorithm} has become a widely used technique for enhancing the
  security of electronic communications.  Because of this and related
  developments, the study of prime numbers, once considered the
  epitome of a topic in ``pure'' mathematics to be studied only for
  its own sake, now turns out to have important practical applications
  to cryptography, electronic funds transfer, and information
  retrieval.}

\begin{Exercise}
\label{exc:1.21}
Use the \scheme|smallest-divisor| procedure to find the smallest divisor
of each of the following numbers: 199, 1999, 19999.
\end{Exercise}

\begin{Exercise}
\label{exc:1.22}
Most Lisp implementations include a primitive called \scheme|runtime|
that returns an integer that specifies the amount of time the system
has been running (measured, for example, in microseconds).  The
following \scheme|timed-prime-test| procedure, when called with an
integer $n$, prints $n$ and checks to see if $n$ is prime.  If $n$ is
prime, the procedure prints three asterisks followed by the amount of time
used in performing the test.

\begin{schemedisplay}
(define (timed-prime-test n)
  (newline)
  (display n)
  (start-prime-test n (runtime)))
(define (start-prime-test n start-time)
  (if (prime? n)
      (report-prime (- (runtime) start-time))))
(define (report-prime elapsed-time)
  (display " *** ")
  (display elapsed-time))
\end{schemedisplay}
Using this procedure, write a procedure \scheme|search-for-primes| that
checks the primality of consecutive odd integers in a specified range.
Use your procedure to find the three smallest primes larger than 1000;
larger than 10,000; larger than 100,000; larger than 1,000,000.  Note
the time needed to test each prime.  Since the testing algorithm has
order of growth of $\Theta(\sqrt n)$, you should expect that testing
for primes around 10,000 should take about $\sqrt( 10$ times as long
as testing for primes around 1000.  Do your timing data bear this out?
How well do the data for 100,000 and 1,000,000 support the $\sqrt n$
prediction?  Is your result compatible with the notion that programs
on your machine run in time proportional to the number of steps
required for the computation?
\end{Exercise}

\begin{Exercise}
\label{exc:1.23}
The \scheme|smallest-divisor| procedure shown at the start of this section
does lots of needless testing: After it checks to see if the
number is divisible by 2 there is no point in checking to see if
it is divisible by any larger even numbers.  This suggests that the
values used for \scheme|test-divisor| should not be 2, 3, 4, 5, 6,
\scheme|...|, but rather 2, 3, 5, 7, 9, \scheme|...|.  To implement this
change, define a procedure \scheme|next| that returns 3 if its input is
equal to 2 and otherwise returns its input plus 2.  Modify the \scheme|smallest-divisor| procedure to use \scheme|(next test-divisor)| instead
of \scheme|(+ test-divisor 1)|.  With \scheme|timed-prime-test|
incorporating this modified version of \scheme|smallest-divisor|, run the
test for each of the 12 primes found in
exercise \ref{exc:1.22}.  Since this modification halves the
number of test steps, you should expect it to run about twice as fast.
Is this expectation confirmed?  If not, what is the observed ratio of
the speeds of the two algorithms, and how do you explain the fact that
it is different from 2?
\end{Exercise}

\begin{Exercise}
\label{exc:1.24}
Modify the \scheme|timed-prime-test| procedure of
exercise \ref{exc:1.22} to use \scheme|fast-prime?| (the
Fermat method), and test each of the 12 primes you found in that
exercise.  Since the Fermat test has $\Theta(\log n)$ growth, how
would you expect the time to test primes near 1,000,000 to compare
with the time needed to test primes near 1000?  Do your data bear this
out?  Can you explain any discrepancy you find?
\end{Exercise}


\begin{Exercise}
\label{exc:1.25}
 Alyssa P. Hacker complains that we went to a lot of extra work in
writing \scheme|expmod|.  After all, she says, since we already know how
to compute exponentials, we could have simply written

\begin{schemedisplay}
(define (expmod base exp m)
  (remainder (fast-expt base exp) m))
\end{schemedisplay}
Is she correct?  Would this procedure serve as well for our fast prime
tester?  Explain.
\end{Exercise}


\begin{Exercise}
\label{exc:1.26}
Louis Reasoner is having great difficulty doing exercise
\ref{exc:1.24}.  His \scheme|fast-prime?| test seems to run more
slowly than his \scheme|prime?| test.  Louis calls his friend Eva Lu
Ator over to help.  When they examine Louis's code, they find that he
has rewritten the \scheme|expmod| procedure to use an explicit
multiplication, rather than calling \scheme|square|:

\begin{schemedisplay}
(define (expmod base exp m)
  (cond ((= exp 0) 1)
        ((even? exp)
         (remainder (* (expmod base (/ exp 2) m)
                       (expmod base (/ exp 2) m))
                    m))
        (else
         (remainder (* base (expmod base (- exp 1) m))
                    m))))
\end{schemedisplay}
``I don't see what difference that could make,'' says Louis.  ``I
do.''  says Eva.  ``By writing the procedure like that, you have
transformed the $\Theta(\log n)$ process into a $\Theta(n)$ process.''
Explain.
\end{Exercise}

\begin{Exercise}
\label{exc:1.27}
Demonstrate that the Carmichael numbers listed in footnote \ref{fn:1.47}
really do fool the Fermat test.  That is, write a procedure that takes
an integer $n$ and tests whether $a^n$ is congruent to $a$ modulo $n$
for every $a < n$, and try your procedure on the given Carmichael
numbers.
\end{Exercise}

\begin{Exercise}
\label{exc:1.28}
One variant of the Fermat test that cannot be fooled is called the
\textit{Miller-Rabin test} (Miller 1976; Rabin 1980).  This starts from
an alternate form of Fermat's Little Theorem, which states that if $n$
is a prime number and $a$ is any positive integer less than $n$, then
$a$ raised to the $n - 1$)st power is congruent to $1$ modulo $n$.  To test
the primality of a number $n$ by the Miller-Rabin test, we pick a
random number $a < n$ and raise $a$ to the $(n - 1)$st power modulo $n$
using the \scheme|expmod| procedure.  However, whenever we perform the
squaring step in \scheme|expmod|, we check to see if we have discovered a
``nontrivial square root of 1 modulo $n$,'' that is, a number not
equal to $1$ or $n - 1$ whose square is equal to 1 modulo $n$.  It is
possible to prove that if such a nontrivial square root of 1 exists,
then $n$ is not prime.  It is also possible to prove that if $n$ is an
odd number that is not prime, then, for at least half the numbers
$a < n$, computing $a^{n-1}$ in this way will reveal a nontrivial
square root of 1 modulo $n$.  (This is why the Miller-Rabin test
cannot be fooled.)  Modify the \scheme|expmod| procedure to signal if it
discovers a nontrivial square root of 1, and use this to implement
the Miller-Rabin test with a procedure analogous to \scheme|fermat-test|.
Check your procedure by testing various known primes and non-primes.
Hint: One convenient way to make \scheme|expmod| signal is to have it
return 0.
\end{Exercise}
% -*- TeX-master: "sicp.tex" -*-

\section{Formulating Abstractions with Higher-Order Procedures}
\label{sec:1.3}




We have seen that procedures are, in effect, abstractions that describe
compound operations on numbers independent of the particular numbers.
For example, when we


\begin{schemedisplay}
(define (cube x) (* x x x))
\end{schemedisplay}
we are not talking about the cube of a particular number, but rather
about a method for obtaining the cube of any number.  Of course we
could get along without ever defining this procedure, by
always writing expressions such as

\begin{schemedisplay}
(* 3 3 3)
(* x x x)
(* y y y)        
\end{schemedisplay}
and never mentioning \scheme|cube| explicitly.  This would place us at
a serious disadvantage, forcing us to work always at the level of the
particular operations that happen to be primitives in the language
(multiplication, in this case) rather than in terms of higher-level
operations.  Our programs would be able to compute cubes, but our
language would lack the ability to express the concept of cubing.  One
of the things we should demand from a powerful programming language is
the ability to build abstractions by assigning names to common
patterns and then to work in terms of the abstractions directly.
Procedures provide this ability.  This is why all but the most
primitive programming languages include mechanisms for defining
procedures.

Yet even in numerical processing we will be severely limited in our
ability to create abstractions if we are restricted to procedures
whose parameters must be numbers.  Often the same programming pattern
will be used with a number of different procedures.  To express such
patterns as concepts, we will need to construct procedures that can
accept procedures as arguments or return procedures as values.
Procedures that manipulate procedures are called \textit{higher-order
  procedures}.  This section shows how higher-order procedures can
serve as powerful abstraction mechanisms, vastly increasing the
expressive power of our language.


\subsection{Procedures as Arguments}
\label{sec:1.3.1}

Consider the following three procedures.  The first computes the sum
of the integers from \scheme|a| through \scheme|b|:

\begin{schemedisplay}
(define (sum-integers a b)
  (if (> a b)
      0
      (+ a (sum-integers (+ a 1) b))))
\end{schemedisplay}
The second computes the sum of the cubes of the integers in the given range:

\begin{schemedisplay}
(define (sum-cubes a b)
  (if (> a b)
      0
      (+ (cube a) (sum-cubes (+ a 1) b))))
\end{schemedisplay}
The third computes the sum of a sequence of terms in the
series

\begin{displaymath}
\frac{1}{3 \cdot 5} +
\frac{1}{5 \cdot 7} +
\frac{1}{9 \cdot 11} +
\cdots
\end{displaymath}
which converges to $\pi / 8$ (very slowly):\footnote{This series,
  usually written in the equivalent form $\frac{\pi}{4} = 1 - (1/3) +
  (1/5) - (1/7) + \cdots$, is due to Leibniz.  We'll see how to use
  this as the basis for some fancy numerical tricks in section
  \ref{sec:3.5.3}.}

\begin{schemedisplay}
(define (pi-sum a b)
  (if (> a b)
      0
      (+ (/ 1.0 (* a (+ a 2))) (pi-sum (+ a 4) b))))
\end{schemedisplay}

These three procedures clearly share a common underlying pattern.
They are for the most part identical, differing only in the name of
the procedure, the function of \scheme|a| used to compute the term to be added,
and the function that provides the next value of \scheme|a|.  We could generate
each of the procedures by filling in slots in the same template:

\begin{schemedisplay}
(define (<name> a b)
  (if (> a b)
      0
      (+ (<term> a)
         (<name> (<next> a) b))))
\end{schemedisplay}

The presence of such a common pattern is strong evidence that there is
a useful abstraction waiting to be brought to the surface.  Indeed,
mathematicians long ago identified the abstraction of
\textit{summation of a series} and invented ``sigma
notation,'' for example$$\sum^b_{n=a} f(n) = f(a) + \cdots + f(b)$$
to express this concept.  The power of sigma notation is that it
allows mathematicians to deal with the concept of summation
itself rather than only with particular sums -- for example, to
formulate general results about sums that are independent of the
particular series being summed.

Similarly, as program designers, we would like our language to
be powerful enough so that we can write a procedure that expresses the
concept of summation itself rather than only procedures
that compute particular sums.  We can do so readily in our
procedural language by taking the common template shown above and
transforming the ``slots'' into formal parameters:

\begin{schemedisplay}
(define (sum term a next b)
  (if (> a b)
      0
      (+ (term a)
         (sum term (next a) next b))))
\end{schemedisplay}
Notice that \scheme|sum| takes as its arguments the lower and upper bounds
\scheme|a| and \scheme|b| together with the procedures \scheme|term| and \scheme|next|.
We can use \scheme|sum| just as we would any procedure.  For example, we can
use it (along with a procedure \scheme|inc| that increments its argument by 1)
to define \scheme|sum-cubes|:

\begin{schemedisplay}
(define (inc n) (+ n 1))
(define (sum-cubes a b)
  (sum cube a inc b))
\end{schemedisplay}
Using this, we can compute the sum of the cubes of the integers from 1
to 10:


\begin{schemedisplay}
(sum-cubes 1 10)
<i>3025</i>
\end{schemedisplay}
With the aid of an identity procedure to compute the term, we can define
\scheme|sum-integers| in terms of \scheme|sum|:


\begin{schemedisplay}
(define (identity x) x)

(define (sum-integers a b)
  (sum identity a inc b))
\end{schemedisplay}
Then we can add up the integers from 1 to 10:


\begin{schemedisplay}
(sum-integers 1 10)
<i>55</i>
\end{schemedisplay}
We can also define \scheme|pi-sum| in the same way:\footnote{Notice
  that we have used block structure (section \ref{sec:1.1.8}) to embed
  the definitions of \scheme|pi-next| and \scheme|pi-term| within
  \scheme|pi-sum|, since these procedures are unlikely to be useful
  for any other purpose.  We will see how to get rid of them
  altogether in section \ref{sec:1.3.2}.}


\begin{schemedisplay}
(define (pi-sum a b)
  (define (pi-term x)
    (/ 1.0 (* x (+ x 2))))
  (define (pi-next x)
    (+ x 4))
  (sum pi-term a pi-next b))
\end{schemedisplay}
Using these procedures, we can compute an approximation to 
$\pi$:


\begin{schemedisplay}
> (* 8 (pi-sum 1 1000))
3.139592655589783
\end{schemedisplay}


Once we have \scheme|sum|, we can use it as a building block in
formulating further concepts.  For instance, the definite integral of a
function $f$ between the limits $a$ and $b$ can be approximated
numerically using the formula
\begin{displaymath}
  \int^b_a f = \left[
    f\left(a + \frac{dx}{2}\right) +
    f\left(a + dx + \frac{dx}{2}\right) +
    f\left(a + 2dx + \frac{dx}{2}\right) +
    \cdots
  \right]
  dx
\end{displaymath}
for small values of $dx$.  We can express this directly as a
procedure:

\begin{schemedisplay}
(define (integral f a b dx)
  (define (add-dx x) (+ x dx))
  (* (sum f (+ a (/ dx 2.0)) add-dx b)
     dx))

> (integral cube 0 1 0.01)
.24998750000000042
> (integral cube 0 1 0.001)
.249999875000001
\end{schemedisplay}
(The exact value of the integral of \scheme|cube| between 0 and 1 is 1/4.)

\begin{Exercise}
\label{exc:1.29}
Simpson's Rule is a more accurate method of numerical integration than
the method illustrated above.  Using Simpson's Rule, the integral of a
function $f$ between $a$ and $b$ is approximated as
\begin{displaymath}
  \frac{h}{3}[
    y_0 + 4y_1 + 2y_2 + 4y_3 + 2y_4 + \cdots + 2y_{n-2} + 4y_{n-1} + y_n
  ]
\end{displaymath}

where $h = \frac{b-a}{n}$, for some even integer $n$, and $y_k = f(a +
kh)$.  (Increasing $n$ increases the accuracy of the approximation.)
Define a procedure that takes as arguments $f$, $a$, $b$, and $n$ and
returns the value of the integral, computed using Simpson's Rule.  Use
your procedure to integrate \scheme|cube| between 0 and 1 (with $n =
100$ and $n = 1000$), and compare the results to those of the
\scheme|integral| procedure shown above.
\end{Exercise}

\begin{Exercise}
\label{exc:1.30}
 The \scheme|sum| procedure above generates a linear recursion.  The
procedure can be rewritten so that the sum is performed iteratively.
Show how to do this by filling in the missing expressions in the
following definition:

\begin{schemedisplay}
(define (sum term a next b)
  (define (iter a result)
    (if <??>
        <??>
        (iter <??> <??>)))
  (iter <??> <??>))
\end{schemedisplay}
\end{Exercise}

\begin{Exercise}
\label{exc:1.31}
% TODO: Use enumerate
a.  The \scheme|sum| procedure is only the simplest of a vast number
of similar abstractions that can be captured as higher-order
procedures.\footnote{The intent of exercises
  \ref{exc:1.31}-\ref{exc:1.33} is to demonstrate the expressive power
  that is attained by using an appropriate abstraction to consolidate
  many seemingly disparate operations.  However, though accumulation
  and filtering are elegant ideas, our hands are somewhat tied in
  using them at this point since we do not yet have data structures to
  provide suitable means of combination for these abstractions.  We
  will return to these ideas in section \ref{sec:2.2.3} when we show
  how to use \textit{sequences} as interfaces for combining filters
  and accumulators to build even more powerful abstractions.  We will
  see there how these methods really come into their own as a powerful
  and elegant approach to designing programs.} Write an analogous
procedure called \scheme|product| that returns the product of the
values of a function at points over a given range.  Show how to define
\scheme|factorial| in terms of \scheme|product|.  Also use
\scheme|product| to compute approximations to $\pi$ using the
formula\footnote{This formula was discovered by the seventeenth-century
English mathematician John Wallis.}
\begin{displaymath}
  \frac{\pi}{4} = 
  \frac{2 \cdot 4 \cdot 4 \cdot 6 \cdot 6 \cdot 8 \cdots} 
       {3 \cdot 3 \cdot 5 \cdot 5 \cdot 7 \cdot 7 \cdots}
\end{displaymath}

b.  If your \scheme|product|
procedure generates a recursive process, write one that generates
an iterative process.
If it generates an iterative process, write one that generates
a recursive process.
\end{Exercise}

\begin{Exercise}
\label{exc:1.32}
a. Show that \scheme|sum| and \scheme|product| (exercise
\ref{exc:1.31}) are both special cases of a still more general notion
called \scheme|accumulate| that combines a collection of terms, using
some general accumulation function:

\begin{schemedisplay}
(accumulate combiner null-value term a next b)
\end{schemedisplay}
\scheme|Accumulate| takes as arguments the same term and range
specifications as \scheme|sum| and \scheme|product|, together with a \scheme|combiner|
procedure (of two arguments) that specifies how the current
term is to be combined with the accumulation of the preceding terms
and a \scheme|null-value| that specifies what base value to use
when the terms run out.  Write \scheme|accumulate|
and show how \scheme|sum| and \scheme|product| can both
be defined as simple calls to \scheme|accumulate|.

b. If your \scheme|accumulate|
procedure generates a recursive process, write one that generates
an iterative process.
If it generates an iterative process, write one that generates
a recursive process.
\end{Exercise}


\begin{Exercise}
\label{exc:1.33}
You can obtain an even more general version of \scheme|accumulate|
(exercise \ref{exc:1.32}) by introducing the notion of a
\textit{filter} on the terms to be combined.  That is, combine only
those terms derived from values in the range that satisfy a specified
condition.  The resulting \scheme|filtered-accumulate| abstraction
takes the same arguments as accumulate, together with an additional
predicate of one argument that specifies the filter.  Write
\scheme|filtered-accumulate| as a procedure.  Show how to express the
following using \scheme|filtered-accumulate|:

a. the sum of the squares of the prime numbers in the interval $a$ to
$b$ (assuming that you have a \scheme|prime?| predicate already written)

b. the product of all the positive integers less than $n$
that are relatively prime to $n$ (i.e., all positive integers
$i < n {\rm ~such~that~GCD}(i,n) = 1$).
\end{Exercise}


\subsection{Constructing Procedures Using {\tt Lambda}}
\label{sec:1.3.2}




In using \scheme|sum| as in section \ref{sec:1.3.1},
it seems terribly awkward to have to define trivial procedures such as
\scheme|pi-term| and \scheme|pi-next| just so we can use them as arguments to
our higher-order procedure.  Rather than define \scheme|pi-next| and \scheme|pi-term|, it would be more convenient
to have a way to directly specify ``the procedure that returns its
input incremented by 4'' and ``the procedure that returns the
reciprocal of its input times its input plus 2.''  We can do this by
introducing the special form \scheme|lambda|, which creates procedures.
Using \scheme|lambda| we can describe what we want as

\begin{schemedisplay}
(lambda (x) (+ x 4))
\end{schemedisplay}
and 

\begin{schemedisplay}
(lambda (x) (/ 1.0 (* x (+ x 2))))
\end{schemedisplay}
Then our \scheme|pi-sum| procedure can be expressed without defining any
auxiliary procedures as

\begin{schemedisplay}
(define (pi-sum a b)
  (sum (lambda (x) (/ 1.0 (* x (+ x 2))))
       a
       (lambda (x) (+ x 4))
       b))
\end{schemedisplay}

Again using \scheme|lambda|, we can write the \scheme|integral| procedure
without having to define the auxiliary procedure \scheme|add-dx|:

\begin{schemedisplay}
(define (integral f a b dx)
  (* (sum f
          (+ a (/ dx 2.0))
          (lambda (x) (+ x dx))
          b)
     dx))
\end{schemedisplay}

In general, \scheme|lambda| is used to create procedures in the same way as
\scheme|define|, except that no name is specified for the procedure:

\begin{schemedisplay}
(lambda (<formal-parameters>) <body>)
\end{schemedisplay}
The resulting procedure is just as much a procedure as one that is
created using \scheme|define|.  The only difference is that it has not
been associated with any name in the environment.  In fact,

\begin{schemedisplay}
(define (plus4 x) (+ x 4))
\end{schemedisplay}
is equivalent to

\begin{schemedisplay}
(define plus4 (lambda (x) (+ x 4)))
\end{schemedisplay}
We can read a \scheme|lambda| expression as follows:

\begin{tabular}{ccccc}

\scheme|(lambda| & \scheme|(x)| & \scheme|(+| & \scheme|x| & \scheme|4))| \\
$\uparrow$ & $\uparrow$ & $\uparrow$ & $\uparrow$ & $\uparrow$ \\
\rm the procedure & of an argument \scheme|x| & that adds & \scheme|x| & and \scheme|4|
\end{tabular}

Like any expression that has a procedure as its value, a
\scheme|lambda| expression can be used as the operator in a
combination such as

\begin{schemedisplay}
> ((lambda (x y z) (+ x y (square z))) 1 2 3)
12
\end{schemedisplay}
or, more generally, in any context where we would normally use a
procedure name.\footnote{It would be clearer and less intimidating to
  people learning Lisp if a name more obvious than \scheme|lambda|,
  such as \scheme|make-procedure|, were used.  But the convention is
  firmly entrenched.  The notation is adopted from the $\lambda$
  calculus, a mathematical formalism introduced by the mathematical
  logician Alonzo Church (1941).  Church developed the $\lambda$
  calculus to provide a rigorous foundation for studying the notions
  of function and function application.  The $\lambda$ calculus has
  become a basic tool for mathematical investigations of the semantics
  of programming languages.}


\subsubsection*{Using \scheme|let| to create local variables}

Another use of \scheme|lambda| is in creating local variables.
We often need local variables in our procedures other than those that have
been bound as formal parameters.  For example, suppose we wish to
compute the function

\begin{displaymath}
  f(x,y) = x(1+xy)^2 + y(1-y) + (1+xy)(1-y)
\end{displaymath}

which we could also express as
\begin{eqnarray*}
  a & = & 1 + xy \\
  b & = & 1-y \\
  f(x,y) & = & xa^2 + yb + ab
\end{eqnarray*}
In writing a procedure to compute $f$, we would like to include as
local variables not only $x$ and $y$ but also the names of
intermediate quantities like $a$ and $b$.  One way to accomplish this
is to use an auxiliary procedure to bind the local variables:


\begin{schemedisplay}
(define (f x y)
  (define (f-helper a b)
    (+ (* x (square a))
       (* y b)
       (* a b)))
  (f-helper (+ 1 (* x y)) 
            (- 1 y)))
\end{schemedisplay}

Of course, we could use a \scheme|lambda| expression to specify an
anonymous procedure for binding our local variables.  The body of
\scheme|f| then becomes a single call to that procedure:


\begin{schemedisplay}
(define (f x y)
  ((lambda (a b)
     (+ (* x (square a))
        (* y b)
        (* a b)))
   (+ 1 (* x y))
   (- 1 y)))
\end{schemedisplay}
This construct is so useful that there is a special form called
\scheme|let| to make its use more convenient.  Using \scheme|let|, the
\scheme|f| procedure could be written as


\begin{schemedisplay}
(define (f x y)
  (let ((a (+ 1 (* x y)))
        (b (- 1 y)))
    (+ (* x (square a))
       (* y b)
       (* a b))))
\end{schemedisplay}
The general form of a \scheme|let| expression is


\begin{schemedisplay}
(let ((<var_1> <exp_1>)
      (<var_2> <exp_2>)
      \vdots 
      (<var_n> <exp_n>))
   <body>)
\end{schemedisplay}
which can be thought of as saying

\begin{tabular}{lll}
let & \slot{var_1} have the value \slot{exp_1} & \hfil\\
& \slot{var_2} have the value \slot{exp_2} \\
& \vdots \\
& \slot{var_n} have the value \slot{exp_n} \\
in & \slot{body}
\end{tabular}

The first part of the \scheme|let| expression is a list of
name-expression pairs.  When the \scheme|let| is evaluated, each name
is associated with the value of the corresponding expression.  The
body of the \scheme|let| is evaluated with these names bound as local
variables.  The way this happens is that the \scheme|let| expression
is interpreted as an alternate syntax for

% TODO: format this correctly
\begin{schemedisplay}
((lambda (<$var_1$> ... <$var_n$>)
    <body>)
 <$exp_1$>
 \vdots
 <$exp_n$>)
\end{schemedisplay}
No new mechanism is required in the interpreter in order to
provide local variables.  A \scheme|let| expression is simply syntactic sugar for
the underlying \scheme|lambda| application.

We can see from this equivalence that the scope of a variable
specified by a \scheme|let| expression is the body of the
\scheme|let|.  This implies that:

\begin{itemize}
\item \scheme|Let| allows one to
bind variables as locally as possible to where they
are to be used.  For example, if the value of \scheme|x| is 5,
the value of the expression

\begin{schemedisplay}
(+ (let ((x 3))
     (+ x (* x 10)))
   x)
\end{schemedisplay}

is 38.  Here, the \scheme|x| in the body of the \scheme|let| is 3,
so the value of the \scheme|let| expression is 33.  On the other hand, the
\scheme|x| that is the second argument to the outermost \scheme|+| is still 5.

\item The variables' values are computed outside the \scheme|let|.
This matters when the expressions that
provide the values for the local variables depend upon
variables having the same names as the local variables themselves.
For example, if the value of \scheme|x| is 2, the expression

\begin{schemedisplay}
(let ((x 3)
      (y (+ x 2)))
  (* x y))
\end{schemedisplay}
will have the value 12 because, inside the body of the \scheme|let|,
\scheme|x| will be 3 and \scheme|y| will be 4 (which is the
outer \scheme|x| plus 2).
\end{itemize}

Sometimes we can use internal definitions to get the same effect as
with \scheme|let|.  For example, we could have defined the procedure
\scheme|f| above as
\begin{schemedisplay}
(define (f x y)
  (define a (+ 1 (* x y)))
  (define b (- 1 y))
  (+ (* x (square a))
     (* y b)
     (* a b)))
\end{schemedisplay}
We prefer, however, to use \scheme|let| in situations like this and to
use internal \scheme|define| only for internal
procedures.\footnote{Understanding internal definitions well enough to
  be sure a program means what we intend it to mean requires a more
  elaborate model of the evaluation process than we have presented in
  this chapter.  The subtleties do not arise with internal definitions
  of procedures, however.  We will return to this issue in section
  \ref{sec:4.1.6}, after we learn more about evaluation.}

\begin{Exercise}
\label{exc:1.34}
Suppose we define the procedure

\begin{schemedisplay}
(define (f g)
  (g 2))
\end{schemedisplay}
Then we have

\begin{schemedisplay}
> (f square)
4

> (f (lambda (z) (* z (+ z 1))))
6
\end{schemedisplay}
What happens if we (perversely) ask the interpreter to evaluate the
combination \scheme|(f f)|?  Explain.
\end{Exercise}


\subsection{Procedures as General Methods}
\label{sec:1.3.3}


We introduced compound procedures in section \ref{sec:1.1.4} as a
mechanism for abstracting patterns of numerical operations so as to
make them independent of the particular numbers involved.  With
higher-order procedures, such as the \scheme|integral| procedure of
section \ref{sec:1.3.1}, we began to see a more powerful kind of
abstraction: procedures used to express general methods of
computation, independent of the particular functions involved.  In
this section we discuss two more elaborate examples -- general methods
for finding zeros and fixed points of functions -- and show how these
methods can be expressed directly as procedures.


\subsubsection*{Finding roots of equations by the half-interval method}


The \textit{half-interval method} is a simple but powerful technique
for finding roots of an equation $f(x) = 0$, where $f$ is a continuous
function.  The idea is that, if we are given points $a$ and $b$ such
that $f(a) < 0 < f(b)$, then $f$ must have at least one zero between
$a$ and $b$.  To locate a zero, let $x$ be the average of $a$ and $b$
and compute $f(x)$.  If $f(x) > 0$, then $f$ must have a zero between
$a$ and $x$.  If $f(x) < 0$ then $f$ must have a zero between $x$ and
$b$.  Continuing in this way, we can identify smaller and smaller
intervals on which $f$ must have a zero.  When we reach a point where
the interval is small enough, the process stops.  Since the interval
of uncertainty is reduced by half at each step of the process, the
number of steps required grows as $\Theta(\log
\frac{L}{T})$, where $L$ is the length of the
original interval and $T$ is the error tolerance (that is, the
size of the interval we will consider ``small enough'').  Here is a
procedure that implements this strategy:


\begin{schemedisplay}
(define (search f neg-point pos-point)
  (let ((midpoint (average neg-point pos-point)))
    (if (close-enough? neg-point pos-point)
        midpoint
        (let ((test-value (f midpoint)))
          (cond ((positive? test-value)
                 (search f neg-point midpoint))
                ((negative? test-value)
                 (search f midpoint pos-point))
                (else midpoint))))))
\end{schemedisplay}

We assume that we are initially given the function $f$ together with
points at which its values are negative and positive.  We first
compute the midpoint of the two given points.  Next we check to see if
the given interval is small enough, and if so we simply return the
midpoint as our answer.  Otherwise, we compute as a test value the
value of $f$ at the midpoint.  If the test value is positive, then we
continue the process with a new interval running from the original
negative point to the midpoint.  If the test value is negative, we
continue with the interval from the midpoint to the positive point.
Finally, there is the possibility that the test value is 0, in which
case the midpoint is itself the root we are searching for.


To test whether the endpoints are ``close enough'' we can use a
procedure similar to the one used in section \ref{sec:1.1.7} for
computing square roots:\footnote{We have used 0.001 as a
  representative ``small'' number to indicate a tolerance for the
  acceptable error in a calculation.  The appropriate tolerance for a
  real calculation depends upon the problem to be solved and the
  limitations of the computer and the algorithm.  This is often a very
  subtle consideration, requiring help from a numerical analyst or
  some other kind of magician.}

\begin{schemedisplay}
(define (close-enough? x y)
  (< (abs (- x y)) 0.001))
\end{schemedisplay}

\scheme|Search| is awkward to use directly, because we can
accidentally give it points at which $f$'s values do not have the
required sign, in which case we get a wrong answer.  Instead we will
use \scheme|search| via the following procedure, which checks to see
which of the endpoints has a negative function value and which has a
positive value, and calls the \scheme|search| procedure accordingly.
If the function has the same sign on the two given points, the
half-interval method cannot be used, in which case the procedure
signals an error.\footnote{This can be accomplished using
  \scheme|error|, which takes as arguments a number of items that are
  printed as error messages.}

\begin{schemedisplay}
(define (half-interval-method f a b)
  (let ((a-value (f a))
        (b-value (f b)))
    (cond ((and (negative? a-value) (positive? b-value))
           (search f a b))
          ((and (negative? b-value) (positive? a-value))
           (search f b a))
          (else
           (error "Values are not of opposite sign" a b)))))
\end{schemedisplay}

The following example uses the half-interval method to approximate $\pi$
as the root between 2 and 4 of $\sin x = 0$:

\begin{schemedisplay}
> (half-interval-method sin 2.0 4.0)
3.14111328125
\end{schemedisplay}

Here is another example, using the half-interval method
to search for a root of the equation $x^3 - 2x - 3 = 0$
between 1 and 2:

\begin{schemedisplay}
> (half-interval-method (lambda (x) (- (* x x x) (* 2 x) 3))
                        1.0
                        2.0)
1.89306640625
\end{schemedisplay}


\subsubsection*{Finding fixed points of functions}


A number $x$ is called a \textit{fixed point} of a function $f$ if $x$
satisfies the equation $f(x) = x$.  For some functions $f$ we
can locate a fixed point by beginning with an initial guess and
applying $f$ repeatedly,

\[
f(x), f(f(x)), f(f(f(x))), \ldots
\]

until the value does not change very much.  Using this idea, we can
devise a procedure \scheme|fixed-point| that takes as inputs a function
and an initial guess and produces an approximation to a fixed point of
the function.  We apply the function repeatedly until we find two
successive values whose difference is less than some prescribed
tolerance:


\begin{schemedisplay}
(define tolerance 0.00001)
(define (fixed-point f first-guess)
  (define (close-enough? v1 v2)
    (< (abs (- v1 v2)) tolerance))
  (define (try guess)
    (let ((next (f guess)))
      (if (close-enough? guess next)
          next
          (try next))))
  (try first-guess))
\end{schemedisplay}
For example, we can use this method to approximate the fixed point of
the cosine function, starting with 1 as an initial
approximation:\footnote{Try this during a boring lecture: Set your
  calculator to radians mode and then repeatedly press the
  \scheme|cos| button until you obtain the fixed point.}


\begin{schemedisplay}
> (fixed-point cos 1.0)
.7390822985224023
\end{schemedisplay}
Similarly, we can find a solution to the equation
$y = \sin y + \cos y$: 

\begin{schemedisplay}
> (fixed-point (lambda (y) (+ (sin y) (cos y)))
               1.0)
1.2587315962971173
\end{schemedisplay}

The fixed-point process is reminiscent of the process we used for
finding square roots in section \ref{sec:1.1.7}.  Both are based on
the idea of repeatedly improving a guess until the result satisfies
some criterion.  In fact, we can readily formulate the square-root
computation as a fixed-point search.  Computing the square root of
some number $x$ requires finding a $y$ such that $y^2 = x$.  Putting
this equation into the equivalent form $y = \frac{x}{y}$, we recognize
that we are looking for a fixed point of the
function\footnote{$\mapsto$ (pronounced ``maps to'') is the
  mathematician's way of writing \scheme|lambda|.  $y \mapsto x/y$
  means \scheme|(lambda(y) (/ x y))|, that is, the function whose
  value at $y$ is $x/y$.} $y \mapsto x/y$, and we can therefore try to
compute square roots as

\begin{schemedisplay}
(define (sqrt x)
  (fixed-point (lambda (y) (/ x y))
               1.0))
\end{schemedisplay}

Unfortunately, this fixed-point search does not converge.  Consider an
initial guess $y_1$.  The next guess is $y_2 = x/y_1$ and the next
guess is $y_3 = x/y_2 = x/(x/y_1) = y_1$.  This results in an infinite
loop in which the two guesses $y_1$ and $y_2$ repeat over and over,
oscillating about the answer.

One way to control such oscillations is to prevent the guesses from
changing so much.  Since the answer is always between our guess $y$
and $x/y$, we can make a new guess that is not as far from $y$ as
$x/y$ by averaging $y$ with $x/y$, so that the next
guess after $y$ is $\frac{1}{2}\big(y + \frac{x}{y}\big)$
instead of $\frac{x}{y}$.  The process of making such a
sequence of guesses is simply the process of looking for a fixed point
of $y \mapsto \frac{1}{2}\big(y + \frac{x}{y}\big)$:

\begin{schemedisplay}
(define (sqrt x)
  (fixed-point (lambda (y) (average y (/ x y)))
               1.0))
\end{schemedisplay}
(Note that $y = \frac{1}{2}\big(y + \frac{x}{y}\big)$ is a simple
transformation of the equation $y = \frac{x}{y}$; to derive it, add
$y$ to both sides of the equation and divide by 2.)

With this modification, the square-root procedure works.  In fact, if
we unravel the definitions, we can see that the sequence of
approximations to the square root generated here is precisely the same
as the one generated by our original square-root procedure of section
\ref{sec:1.1.7}.  This approach of averaging successive approximations
to a solution, a technique we that we call \textit{average damping},
often aids the convergence of fixed-point searches.

\begin{Exercise}
\label{exc:1.35}
Show that the golden ratio $\phi$ (section \ref{sec:1.2.2}) is a fixed
point of the transformation $x \mapsto 1 + \frac{1}{x}$, and use this
fact to compute $\phi$ by means of the \scheme|fixed-point| procedure.
\end{Exercise}

\begin{Exercise}
\label{exc:1.36}
Modify \scheme|fixed-point| so that it prints the sequence of
approximations it generates, using the \scheme|newline| and
\scheme|display| primitives shown in exercise \ref{exc:1.22}.  Then
find a solution to $x^x = 1000$ by finding a fixed point of $x \mapsto
\frac{\log 1000}{\log x}$.  (Use Scheme's primitive \scheme|log|
procedure, which computes natural logarithms.)  Compare the number of
steps this takes with and without average damping.  (Note that you
cannot start \scheme|fixed-point| with a guess of 1, as this would
cause division by \scheme|log|(1) = 0.)
\end{Exercise}


\begin{Exercise}
\label{exc:1.37}
a. An infinite \textit{continued fraction} is an expression of the form
\[
% TODO: Format this better (without decreasing size)
f = \frac{N_1}{D_1 + 
  \frac{N_2}{D_2 +
    \frac{N_3}{D_3 +
      \cdots}}}
\]    
As an example, one can show that the infinite continued fraction
expansion with the $N_i$ and the $D_i$ all equal to 1 produces
$\frac{1}{\phi}$ , where $\phi$ is the golden ratio (described in
section \ref{sec:1.2.2}).  One way to approximate an infinite
continued fraction is to truncate the expansion after a given number
of terms.  Such a truncation -- a so-called \textit{$k$-term finite
  continued fraction} -- has the form 

\[
% TODO: Format this better (without decreasing size)
\frac{N_1}{D_1 +
  \frac{N_2}{\ddots + 
    \frac{N_k}{D_k}}}
\]
Suppose that \scheme|n| and \scheme|d| are procedures of one argument
(the term index $i$) that return the $N_i$ and $D_i$ of the terms of
the continued fraction.  Define a procedure \scheme|cont-frac| such
that evaluating \scheme|(cont-frac n d k)| computes the value of the
\textit{k}-term finite continued fraction.  Check your procedure by
approximating $\frac{1}{\phi}$ using
\begin{schemedisplay}
(cont-frac (lambda (i) 1.0)
           (lambda (i) 1.0)
           k)
\end{schemedisplay}
for successive values of \scheme|k|.  How large must you make \scheme|k|
in order to get an approximation that is accurate to 4 decimal places?

b. If your \scheme|cont-frac| procedure generates a recursive process,
write one that generates an iterative process.  If it generates an
iterative process, write one that generates a recursive process.
\end{Exercise}


\begin{Exercise}
\label{exc:1.38}
In 1737, the Swiss mathematician Leonhard Euler published a memoir
\textit{De Fractionibus Continuis}, which included a continued
fraction expansion for $e - 2$, where $e$ is the base of the natural
logarithms.  In this fraction, the $N_i$ are all 1, and the $D_1$ are
successively $1, 2, 1, 1, 4, 1, 1, 6, 1, 1, 8, \ldots$.  Write a
program that uses your \scheme|cont-frac| procedure from exercise
\ref{exc:1.37} to approximate $e$, based on Euler's expansion.
\end{Exercise}

\begin{Exercise}
\label{exc:1.39}
A continued fraction representation of the tangent function was
published in 1770 by the German mathematician J.H. Lambert:
\begin{displaymath}
  % TODO: format this without decreasing size
  \tan x = \frac{x}{1 - 
    \frac{x^2}{3 -
      \frac{x^2}{5 - 
        \ddots}}}
\end{displaymath}
where $x$ is in radians.  Define a procedure \scheme|(tan-cf x k)|
that computes an approximation to the tangent function based on
Lambert's formula.  \scheme|K| specifies the number of terms to
compute, as in exercise \ref{exc:1.37}.
\end{Exercise}


\subsection{Procedures as Returned Values}
\label{sec:1.3.4}



The above examples demonstrate how the ability to pass procedures as
arguments significantly enhances the expressive power of our
programming language.  We can achieve even more expressive power by
creating procedures whose returned values are themselves procedures.

We can illustrate this idea by looking again at the fixed-point
example described at the end of section \ref{sec:1.3.3}.  We
formulated a new version of the square-root procedure as a fixed-point
search, starting with the observation that $\sqrt x$ is a fixed-point
of the function $y \mapsto x/y$.  Then we used average damping to make
the approximations converge.  Average damping is a useful general
technique in itself.  Namely, given a function $f$, we consider the
function whose value at $x$ is equal to the average of $x$ and $f(x)$.


We can express the idea of average damping by means of the
following procedure:

\begin{schemedisplay}
(define (average-damp f)
  (lambda (x) (average x (f x))))
\end{schemedisplay}
\scheme|Average-damp| is a procedure that takes as its argument a
procedure \scheme|f| and returns as its value a procedure (produced by
the \scheme|lambda|) that, when applied to a number \scheme|x|,
produces the average of \scheme|x| and \scheme|(f x)|.  For example,
applying \scheme|average-damp| to the \scheme|square| procedure
produces a procedure whose value at some number $x$ is the average of
$x$ and $x^2$.  Applying this resulting procedure to 10 returns the
average of 10 and 100, or 55:\footnote{Observe that this is a
  combination whose operator is itself a combination.  Exercise
  \ref{exc:1.4} already demonstrated the ability to form such
  combinations, but that was only a toy example.  Here we begin to see
  the real need for such combinations -- when applying a procedure
  that is obtained as the value returned by a higher-order procedure.}

\begin{schemedisplay}
> ((average-damp square) 10)
55
\end{schemedisplay}

Using \scheme|average-damp|, we can reformulate the square-root procedure
as follows:

\begin{schemedisplay}
(define (sqrt x)
  (fixed-point (average-damp (lambda (y) (/ x y)))
               1.0))
\end{schemedisplay}
Notice how this formulation makes explicit the three ideas in the
method: fixed-point search, average damping, and the function $y
\mapsto x/y$.  It is instructive to compare this formulation of the
square-root method with the original version given in section
\ref{sec:1.1.7}.  Bear in mind that these procedures express the same
process, and notice how much clearer the idea becomes when we express
the process in terms of these abstractions.  In general, there are
many ways to formulate a process as a procedure.  Experienced
programmers know how to choose procedural formulations that are
particularly perspicuous, and where useful elements of the process are
exposed as separate entities that can be reused in other applications.
As a simple example of reuse, notice that the cube root of $x$ is a
fixed point of the function $y \mapsto \frac{x}{y^2}$, so we can
immediately generalize our square-root procedure to one that extracts
cube roots:\footnote{See exercise \ref{exc:1.45} for a further
  generalization.}


\begin{schemedisplay}
(define (cube-root x)
  (fixed-point (average-damp (lambda (y) (/ x (square y))))
               1.0))
\end{schemedisplay}



\subsubsection*{Newton's method}

When we first introduced the square-root procedure, in
section \ref{sec:1.1.7}, we mentioned that this was a special case of
\textit{Newton's method}.  
If $x \mapsto g(x)$ is a differentiable function, then a solution of
the equation $g(x) = 0$ is a fixed point of the function $x \mapsto f(x)$
where
\begin{displaymath}
  f(x) = x - \frac{g(x)}{Dg(x)}
\end{displaymath}
and $Dg(x)$ is the derivative of $g$ evaluated at $x$.  Newton's
method is the use of the fixed-point method we saw above to
approximate a solution of the equation by finding a fixed point of the
function $f$.\footnote{Elementary calculus books usually describe
  Newton's method in terms of the sequence of approximations $x_{n+1}
  = x_n - \frac{g(x_n)}{Dg(x_n)}$.  Having language for talking about
  processes and using the idea of fixed points simplifies the
  description of the method.} For many functions $f$ and for
sufficiently good initial guesses for $x$, Newton's method converges
very rapidly to a solution of $g(x) = 0$.\footnote{Newton's method
  does not always converge to an answer, but it can be shown that in
  favorable cases each iteration doubles the number-of-digits accuracy
  of the approximation to the solution.  In such cases, Newton's
  method will converge much more rapidly than the half-interval
  method.}

In order to implement Newton's method as a procedure, we must first
express the idea of derivative.  Note that ``derivative,'' like
average damping, is something that transforms a function into another
function.  For instance, the derivative of the function $\textit{x}
\mapsto x^3$ is the function $x \mapsto 3x^2$.  In general, if $g$ is
a function and $dx$ is a small number, then the derivative $Dg$ of $g$
is the function whose value at any number $x$ is given (in the limit
of small $dx$) by \[Dg(x) = \frac{g(x+dx) - g(x)}{dx}\] Thus, we can
express the idea of derivative (taking $dx$ to be, say, 0.00001) as
the procedure

\begin{schemedisplay}
(define (deriv g)
  (lambda (x)
    (/ (- (g (+ x dx)) (g x))
       dx)))
\end{schemedisplay}
along with the definition

\begin{schemedisplay}
(define dx 0.00001)
\end{schemedisplay}

Like \scheme|average-damp|, \scheme|deriv| is a procedure that takes a
procedure as argument and returns a procedure as value.  For example,
to approximate the derivative of $x \mapsto x^3$ at 5 (whose exact
value is 75) we can evaluate

\begin{schemedisplay}
> (define (cube x) (* x x x))
> ((deriv cube) 5)
75.00014999664018
\end{schemedisplay}

With the aid of \scheme|deriv|, we can express Newton's method as a
fixed-point process:

\begin{schemedisplay}
(define (newton-transform g)
  (lambda (x)
    (- x (/ (g x) ((deriv g) x)))))
(define (newtons-method g guess)
  (fixed-point (newton-transform g) guess))
\end{schemedisplay}
The \scheme|newton-transform| procedure expresses the formula at the
beginning of this section, and \scheme|newtons-method| is readily
defined in terms of this.  It takes as arguments a procedure that
computes the function for which we want to find a zero, together with
an initial guess.  For instance, to find the square root of $x$, we
can use Newton's method to find a zero of the function $y \mapsto y^2
- x$ starting with an initial guess of 1.\footnote{For finding square
  roots, Newton's method converges rapidly to the correct solution
  from any starting point.} This provides yet another form of the
square-root procedure:

\begin{schemedisplay}
(define (sqrt x)
  (newtons-method (lambda (y) (- (square y) x))
                  1.0))
\end{schemedisplay}

\subsubsection*{Abstractions and first-class procedures}

We've seen two ways to express the square-root
computation as an instance of a more general method, once as a fixed-point
search and once using Newton's method.  Since Newton's method
was itself expressed as a fixed-point process,
we actually saw two ways to compute square roots as fixed points.
Each method begins with a function and finds a fixed
point of some transformation of the function.  We can express this
general idea itself as a procedure:

\begin{schemedisplay}
(define (fixed-point-of-transform g transform guess)
  (fixed-point (transform g) guess))
\end{schemedisplay}

This very general procedure takes as its arguments a procedure
\scheme|g| that computes some function, a procedure that transforms
\scheme|g|, and an initial guess.  The returned result is a fixed
point of the transformed function.

Using this abstraction, we can recast the first square-root
computation from this section (where we look for a fixed point of the
average-damped version of $y \mapsto x/y$) as an instance of this
general method:

\begin{schemedisplay}
(define (sqrt x)
  (fixed-point-of-transform (lambda (y) (/ x y))
                            average-damp
                            1.0))
\end{schemedisplay}

Similarly, we can express the second square-root computation from this
section (an instance of Newton's method that finds a fixed point of
the Newton transform of $y \mapsto y^2 - x$) as

\begin{schemedisplay}
(define (sqrt x)
  (fixed-point-of-transform (lambda (y) (- (square y) x))
                            newton-transform
                            1.0))
\end{schemedisplay}

We began section \ref{sec:1.3} with the observation that compound
procedures are a crucial abstraction mechanism, because they permit us
to express general methods of computing as explicit elements in our
programming language.  Now we've seen how higher-order procedures
permit us to manipulate these general methods to create further
abstractions.

As programmers, we should be alert to opportunities to identify the
underlying abstractions in our programs and to build upon them and
generalize them to create more powerful abstractions.  This is not to
say that one should always write programs in the most abstract way
possible; expert programmers know how to choose the level of
abstraction appropriate to their task.  But it is important to be able
to think in terms of these abstractions, so that we can be ready to
apply them in new contexts.  The significance of higher-order
procedures is that they enable us to represent these abstractions
explicitly as elements in our programming language, so that they can
be handled just like other computational elements.

In general, programming languages impose restrictions on the ways in
which computational elements can be manipulated.  Elements with the
fewest restrictions are said to have \textit{first-class} status.
Some of the ``rights and privileges'' of first-class elements
are:\footnote{The notion of first-class status of programming-language
  elements is due to the British computer scientist Christopher
  Strachey (1916-1975).}
\begin{itemize}
\item They may be named by variables.
\item They may be passed as arguments to procedures.
\item They may be returned as the results of procedures.
\item They may be included in data structures.\footnote{We'll see
examples of this after we introduce data structures in chapter \ref{chap:2}.}
\end{itemize}
Lisp, unlike other common programming languages, awards procedures
full first-class status.  This poses challenges for efficient
implementation, but the resulting gain in expressive power is
enormous.\footnote{The major implementation cost of first-class
  procedures is that allowing procedures to be returned as values
  requires reserving storage for a procedure's free variables even
  while the procedure is not executing.  In the Scheme implementation
  we will study in section \ref{sec:4.1}, these variables are stored
  in the procedure's environment.}

\begin{Exercise}
\label{exc:1.40}
Define a procedure \scheme|cubic| that can be used together with the
\scheme|newtons-method| procedure in expressions of the form


\begin{schemedisplay}
(newtons-method (cubic a b c) 1)
\end{schemedisplay}
to approximate zeros of the cubic $x+3 + ax^2 + bc + c$.
\end{Exercise}

\begin{Exercise}
\label{exc:1.41}
Define a procedure \scheme|double| that takes a procedure of one
argument as argument and returns a procedure that applies the original
procedure twice.  For example, if \scheme|inc| is a procedure that
adds 1 to its argument, then \scheme|(double inc)| should be a
procedure that adds 2.  What value is returned by

\begin{schemedisplay}
(((double (double double)) inc) 5)
\end{schemedisplay}
\end{Exercise}

\begin{Exercise}
\label{exc:1.42}
Let $f$ and $g$ be two one-argument functions.  The
\textit{composition} $f$ after $g$ is defined to be the function $x
\mapsto f(g(x))$.  Define a procedure \scheme|compose| that implements
composition.  For example, if \scheme|inc| is a procedure that adds 1
to its argument,

\begin{schemedisplay}
> ((compose square inc) 6)
49
\end{schemedisplay}
\end{Exercise}


\begin{Exercise}
\label{exc:1.43}
If $f$ is a numerical function and $n$ is a positive integer, then we
can form the $n$th repeated application of $f$, which is defined to be
the function whose value at $x$ is $f(f(\ldots(f(x))\ldots))$.  For
example, if $f$ is the function $x \mapsto x + 1$, then the $n$th
repeated application of $f$ is the function $x \mapsto x + n$.  If $f$
is the operation of squaring a number, then the $n$th repeated
application of $f$ is the function that raises its argument to the
$2^n$th power.  Write a procedure that takes as inputs a
procedure that computes $f$ and a positive integer $n$ and returns the
procedure that computes the $n$th repeated application of $f$.  Your
procedure should be able to be used as follows:


\begin{schemedisplay}
> ((repeated square 2) 5)
625
\end{schemedisplay}
Hint: You may find it convenient to use \scheme|compose| from
exercise \ref{exc:1.42}.
\end{Exercise}


\begin{Exercise}
\label{exc:1.44}
The idea of \textit{smoothing} a function is an important concept in
signal processing.  If $f$ is a function and $dx$ is some small
number, then the smoothed version of $f$ is the function whose value
at a point $x$ is the average of $f(x - dx)$, $f(x)$, and $f(x + dx)$.
Write a procedure \scheme|smooth| that takes as input a procedure that
computes $f$ and returns a procedure that computes the smoothed $f$.
It is sometimes valuable to repeatedly smooth a function (that is,
smooth the smoothed function, and so on) to obtained the
\textit{$n$-fold smoothed function}.  Show how to generate the
$n$-fold smoothed function of any given function using \scheme|smooth|
and \scheme|repeated| from exercise \ref{exc:1.43}.
\end{Exercise}

\begin{Exercise}
\label{exc:1.45}
We saw in section \ref{sec:1.3.3} that attempting to compute square
roots by naively finding a fixed point of $y \mapsto x/y$ does not
converge, and that this can be fixed by average damping.  The same
method works for finding cube roots as fixed points of the
average-damped $y \mapsto x/y^2$.  Unfortunately, the process does not
work for fourth roots -- a single average damp is not enough to make a
fixed-point search for $y \mapsto x/y^3$ converge.  On the other hand,
if we average damp twice (i.e., use the average damp of the average
damp of $y \mapsto x/y^3$) the fixed-point search does converge.  Do
some experiments to determine how many average damps are required to
compute $n$th roots as a fixed-point search based upon repeated
average damping of $y \mapsto x/y^{n-1}$.  Use this to implement a
simple procedure for computing $n$th roots using \scheme|fixed-point|,
\scheme|average-damp|, and the \scheme|repeated| procedure of exercise
\ref{exc:1.43}.  Assume that any arithmetic operations you need are
available as primitives.
\end{Exercise}


\begin{Exercise}
\label{exc:1.46}
Several of the numerical methods described in this chapter are
instances of an extremely general computational strategy known as
\textit{iterative improvement}.  Iterative improvement says that, to
compute something, we start with an initial guess for the answer, test
if the guess is good enough, and otherwise improve the guess and
continue the process using the improved guess as the new guess.  Write
a procedure \scheme|iterative-improve| that takes two procedures as
arguments: a method for telling whether a guess is good enough and a
method for improving a guess.  \scheme|Iterative-improve| should
return as its value a procedure that takes a guess as argument and
keeps improving the guess until it is good enough.  Rewrite the
\scheme|sqrt| procedure of section \ref{sec:1.1.7} and the
\scheme|fixed-point| procedure of section \ref{sec:1.3.3} in terms of
\scheme|iterative-improve|.
\end{Exercise}

% -*- TeX-master: "sicp.tex" -*-

\chapter{Building Abstractions with Data}
\label{chap:2}

\epigraph{
We now come to the decisive step of mathematical abstraction: we
forget about what the symbols stand for. \scheme|...|[The mathematician]
need not be idle; there are many operations which he may carry out
with these symbols, without ever having to look at the things they
stand for.}{Hermann Weyl, \textit{The Mathematical Way of Thinking}}

We concentrated in chapter 1 on computational processes and on the
role of procedures in program design.  We saw how to use primitive
data (numbers) and primitive operations (arithmetic operations), how
to combine procedures to form compound procedures through composition,
conditionals, and the use of parameters, and how to abstract
procedures by using \scheme|define|.  We saw that a procedure can be
regarded as a pattern for the local evolution of a process, and we
classified, reasoned about, and performed simple algorithmic analyses
of some common patterns for processes as embodied in procedures.  We
also saw that higher-order procedures enhance the power of our
language by enabling us to manipulate, and thereby to reason in terms
of, general methods of computation.  This is much of the essence of
programming.

In this chapter we are going to look at more complex data.  All the
procedures in chapter 1 operate on simple numerical data, and simple
data are not sufficient for many of the problems we wish to address
using computation.  Programs are typically designed to model complex
phenomena, and more often than not one must construct computational
objects that have several parts in order to model real-world phenomena
that have several aspects.  Thus, whereas our focus in chapter 1 was
on building abstractions by combining procedures to form compound
procedures, we turn in this chapter to another key aspect of any
programming language: the means it provides for building abstractions
by combining data objects to form \textit{compound data}.

Why do we want compound data in a programming language?  For the same
reasons that we want compound procedures: to elevate the conceptual
level at which we can design our programs, to increase the modularity
of our designs, and to enhance the expressive power of our language.
Just as the ability to define procedures enables us to deal with
processes at a higher conceptual level than that of the primitive
operations of the language, the ability to construct compound data
objects enables us to deal with data at a higher conceptual level than
that of the primitive data objects of the language.

Consider the task of designing a system to perform arithmetic with
rational numbers.  We could imagine an operation \scheme|add-rat| that
takes two rational numbers and produces their sum.  In terms of simple
data, a rational number can be thought of as two integers: a numerator
and a denominator.  Thus, we could design a program in which each
rational number would be represented by two integers (a numerator and
a denominator) and where \scheme|add-rat| would be implemented by two
procedures (one producing the numerator of the sum and one producing
the denominator).  But this would be awkward, because we would then
need to explicitly keep track of which numerators corresponded to
which denominators.  In a system intended to perform many operations
on many rational numbers, such bookkeeping details would clutter the
programs substantially, to say nothing of what they would do to our
minds.  It would be much better if we could ``glue together'' a
numerator and denominator to form a pair -- a \textit{compound data
  object} -- that our programs could manipulate in a way that would be
consistent with regarding a rational number as a single conceptual
unit.

The use of compound data also enables us to increase the modularity of
our programs.  If we can manipulate rational numbers directly as
objects in their own right, then we can separate the part of our
program that deals with rational numbers per se from the details of
how rational numbers may be represented as pairs of integers.  The
general technique of isolating the parts of a program that deal with
how data objects are represented from the parts of a program that deal
with how data objects are used is a powerful design methodology called
\textit{data abstraction}.  We will see how data abstraction makes
programs much easier to design, maintain, and modify.

The use of compound data leads to a real increase in the expressive
power of our programming language.  Consider the idea of forming a
``linear combination'' $ax + by$.  We might like to write a procedure
that would accept $a$, $b$, $x$, and $y$ as arguments and return the
value of $ax + by$.  This presents no difficulty if the arguments are
to be numbers, because we can readily define the procedure

\begin{schemedisplay}
(define (linear-combination a b x y) 
  (+ (* a x) (* b y)))
\end{schemedisplay}
But suppose we are not concerned only with numbers.  Suppose we would
like to express, in procedural terms, the idea that one can form
linear combinations whenever addition and multiplication are
defined -- for rational numbers, complex numbers, polynomials, or
whatever.  We could express this as a procedure of the form

\begin{schemedisplay}
(define (linear-combination a b x y)     
  (add (mul a x) (mul b y))) 
\end{schemedisplay}
where \scheme|add| and \scheme|mul| are not the primitive procedures
\scheme|+| and \scheme|*| but rather more complex things that will
perform the appropriate operations for whatever kinds of data we pass
in as the arguments \scheme|a|, \scheme|b|, \scheme|x|, and
\scheme|y|. The key point is that the only thing
\scheme|linear-combination| should need to know about \scheme|a|,
\scheme|b|, \scheme|x|, and \scheme|y| is that the procedures
\scheme|add| and \scheme|mul| will perform the appropriate
manipulations.  From the perspective of the procedure
\scheme|linear-combination|, it is irrelevant what \scheme|a|,
\scheme|b|, \scheme|x|, and \scheme|y| are and even more irrelevant
how they might happen to be represented in terms of more primitive
data.  This same example shows why it is important that our
programming language provide the ability to manipulate compound
objects directly: Without this, there is no way for a procedure such
as \scheme|linear-combination| to pass its arguments along to
\scheme|add| and \scheme|mul| without having to know their detailed
structure.\footnote{The ability to directly manipulate procedures
  provides an analogous increase in the expressive power of a
  programming language.  For example, in section \ref{sec:1.3.1} we
  introduced the \scheme|sum| procedure, which takes a procedure
  \scheme|term| as an argument and computes the sum of the values of
  \scheme|term| over some specified interval.  In order to define
  \scheme|sum|, it is crucial that we be able to speak of a procedure
  such as \scheme|term| as an entity in its own right, without regard
  for how \scheme|term| might be expressed with more primitive
  operations.  Indeed, if we did not have the notion of ``a
  procedure,'' it is doubtful that we would ever even think of the
  possibility of defining an operation such as \scheme|sum|.
  Moreover, insofar as performing the summation is concerned, the
  details of how \scheme|term| may be constructed from more primitive
  operations are irrelevant.  }

We begin this chapter by implementing the rational-number arithmetic
system mentioned above.  This will form the background for our
discussion of compound data and data abstraction.  As with compound
procedures, the main issue to be addressed is that of abstraction as a
technique for coping with complexity, and we will see how data
abstraction enables us to erect suitable \textit{abstraction barriers}
between different parts of a program.

We will see that the key to forming compound data is that a
programming language should provide some kind of ``glue'' so that data
objects can be combined to form more complex data objects.  There are
many possible kinds of glue.  Indeed, we will discover how to form
compound data using no special ``data'' operations at all, only
procedures.  This will further blur the distinction between
``procedure'' and ``data,'' which was already becoming tenuous toward
the end of chapter 1.  We will also explore some conventional
techniques for representing sequences and trees.  One key idea in
dealing with compound data is the notion of \textit{closure} -- that the
glue we use for combining data objects should allow us to combine not
only primitive data objects, but compound data objects as well.
Another key idea is that compound data objects can serve as \textit{conventional interfaces} for combining program modules in
mix-and-match ways.  We illustrate some of these ideas by presenting a
simple graphics language that exploits closure.

We will then augment the representational power of our language by
introducing \textit{symbolic expressions} -- data whose elementary parts
can be arbitrary symbols rather than only numbers.  We explore various
alternatives for representing sets of objects.  We will find that,
just as a given numerical function can be computed by many different
computational processes, there are many ways in which a given data
structure can be represented in terms of simpler objects, and the
choice of representation can have significant impact on the time and
space requirements of processes that manipulate the data.  We will
investigate these ideas in the context of symbolic differentiation,
the representation of sets, and the encoding of information.

Next we will take up the problem of working with data that may be
represented differently by different parts of a program.  This leads
to the need to implement \textit{generic operations}, which must handle
many different types of data.  Maintaining modularity in the
presence of generic operations requires more powerful abstraction
barriers than can be erected with simple data abstraction alone.  In
particular, we introduce \textit{data-directed programming} as a
technique that allows individual data representations to be designed
in isolation and then combined \textit{additively} (i.e., without
modification).  To illustrate the power of this approach to system
design, we close the chapter by applying what we have learned to the
implementation of a package for performing symbolic arithmetic on
polynomials, in which the coefficients of the polynomials can be
integers, rational numbers, complex numbers, and even other
polynomials.


% -*- TeX-master: "sicp.tex" -*-


\section{Introduction to Data Abstraction}
\label{sec:2.1}

In section \ref{sec:1.1.8}, we noted
that a procedure used as an element in creating a more complex
procedure could be regarded not only as a collection of particular
operations but also as a procedural abstraction.  That is, the details
of how the procedure was implemented could be suppressed, and the
particular procedure itself could be replaced by any other procedure
with the same overall behavior.  In other words, we could make an
abstraction that would separate the way the procedure would be used
from the details of how the procedure would be implemented in terms of
more primitive procedures.  The analogous notion for compound data is
called \textit{data abstraction}.  Data abstraction is a methodology that
enables us to isolate how a compound data object is used from the
details of how it is constructed from more primitive data objects.

The basic idea of data abstraction is to structure the programs that
are to use compound data objects so that they operate on ``abstract
data.'' That is, our programs should use data in such a way as to make
no assumptions about the data that are not strictly necessary for
performing the task at hand.  At the same time, a ``concrete'' data
representation is defined independent of the programs that use
the data.  The interface between these two parts of our system will be
a set of procedures, called \textit{selectors} and \textit{constructors},
that implement the abstract data in terms of the concrete
representation.  To illustrate this technique, we will consider how to
design a set of procedures for manipulating rational numbers.


\subsection{Example: Arithmetic Operations for Rational Numbers}
\label{sec:2.1.1}



Suppose we want to do arithmetic with rational numbers.  We want to be
able to add, subtract, multiply, and divide them and to test whether
two rational numbers are equal.

Let us begin by assuming that we already have a way of constructing a
rational number from a numerator and a denominator.  We also assume
that, given a rational number, we have a way of extracting (or
selecting) its numerator and its denominator.  Let us further assume
that the constructor and selectors are available as procedures:

\begin{itemize}
\item \scheme|(make-rat <n> <d>)| returns the rational number whose
  numerator is the integer \scheme|<n>| and whose denominator is the
  integer \scheme|<d>|.

\item \scheme|(numer <x>)| returns the numerator of the rational
  number \scheme|<x>|.

\item \scheme|(denom <x>)| returns the denominator of the rational
  number \scheme|<x>|.
\end{itemize}

We are using here a powerful strategy of synthesis: \textit{wishful thinking}.
We haven't yet said how a rational number is represented, or how the
procedures \scheme|numer|, \scheme|denom|, and \scheme|make-rat| should be
implemented.  Even so, if we did have these three procedures, we could
then add, subtract, multiply, divide, and test equality by using the
following relations:

\begin{eqnarray*}
  \frac{n_1}{d_1} + \frac{n_2}{d_2} & = & \frac{n_1d_2 + n_2d_1}{d_1d_2} \\
  \frac{n_1}{d_1} - \frac{n_2}{d_2} & = & \frac{n_1d_2 - n_2d_1}{d_1d_2} \\
  \frac{n_1}{d_1} \cdot \frac{n_2}{d_2} & = & \frac{n_1n_2}{d_1d_2} \\
  \frac{n_1/d1}{n_2/d_2} & = & \frac{n_1d_2}{d_1n_2} \\
  \frac{n_1}{d_1} & =& \frac{n_2}{d_2} \quad \text{if and only if} \quad n_1d_2 = n_2d_1 \\
\end{eqnarray*}

We can express these rules as procedures:

\begin{schemedisplay}
(define (add-rat x y)
  (make-rat (+ (* (numer x) (denom y))
               (* (numer y) (denom x)))
            (* (denom x) (denom y))))
(define (sub-rat x y)
  (make-rat (- (* (numer x) (denom y))
               (* (numer y) (denom x)))
            (* (denom x) (denom y))))
(define (mul-rat x y)
  (make-rat (* (numer x) (numer y))
            (* (denom x) (denom y))))
(define (div-rat x y)
  (make-rat (* (numer x) (denom y))
            (* (denom x) (numer y))))
(define (equal-rat? x y)
  (= (* (numer x) (denom y))
     (* (numer y) (denom x))))
\end{schemedisplay}

Now we have the operations on rational numbers defined in terms of the
selector and constructor procedures
\scheme|numer|, \scheme|denom|, and \scheme|make-rat|.
But we haven't yet defined these.
What we need is some way to glue together a numerator and a
denominator to form a rational
number.


\subsubsection*{Pairs}

To enable us to implement the concrete level of our data abstraction,
our language provides a compound structure called a \textit{pair},
which can be constructed with the primitive procedure \scheme|cons|.
This procedure takes two arguments and returns a compound data object
that contains the two arguments as parts.  Given a pair, we can
extract the parts using the primitive procedures \scheme|car| and
\scheme|cdr|.\footnote{The name \scheme|cons| stands for
  ``construct.''  The names \scheme|car| and \scheme|cdr| derive from
  the original implementation of Lisp on the IBM 704.  That machine
  had an addressing scheme that allowed one to reference the
  ``address'' and ``decrement'' parts of a memory location.
  \scheme|Car| stands for ``Contents of Address part of Register'' and
  \scheme|cdr| (pronounced ``could-er'') stands for ``Contents of
  Decrement part of Register.''} Thus, we can use \scheme|cons|,
\scheme|car|, and \scheme|cdr| as follows:

\begin{schemedisplay}
> (define x (cons 1 2))

> (car x)
1

> (cdr x)
2
\end{schemedisplay}

Notice that a pair is a data object that can be given a name and
manipulated, just like a primitive data object.  Moreover, \scheme|cons|
can be used to form pairs whose elements are pairs, and so on:


\begin{schemedisplay}
> (define x (cons 1 2))

> (define y (cons 3 4))

> (define z (cons x y))

> (car (car z))
1

> (car (cdr z))
3
\end{schemedisplay}
In section \ref{sec:2.2} we will see how this ability to
combine pairs means that pairs can be used as general-purpose building
blocks to create all sorts of complex data structures.  The single
compound-data primitive \textit{pair}, implemented by the procedures \scheme|cons|, \scheme|car|, and \scheme|cdr|, is the only glue we need.  Data
objects constructed from pairs are called \textit{list-structured} data.


\subsubsection*{Representing rational numbers}

\begin{schemeregion}
Pairs offer a natural way to complete the rational-number system.
Simply represent a rational number as a pair of two integers: a
numerator and a denominator.  Then \scheme|make-rat|, \scheme|numer|, and
\scheme|denom| are readily implemented as follows:\footnote{Another way to define the selectors and constructor is
\begin{schemedisplay}
(define make-rat cons)
(define numer car)
(define denom cdr)
\end{schemedisplay}
The first definition associates the name \scheme|make-rat| with the
value of the expression \scheme|cons|, which is the primitive
procedure that constructs pairs.  Thus \scheme|make-rat| and
\scheme|cons| are names for the same primitive constructor.

Defining selectors and constructors in this way is efficient: Instead
of \scheme|make-rat| \textit{calling} \scheme|cons|, \scheme|make-rat|
\textit{is} \scheme|cons|, so there is only one procedure called, not
two, when \scheme|make-rat| is called.  On the other hand, doing this
defeats debugging aids that trace procedure calls or put breakpoints
on procedure calls: You may want to watch \scheme|make-rat| being
called, but you certainly don't want to watch every call to
\scheme|cons|.

We have chosen not to use this style of definition in this book.}
\end{schemeregion}


\begin{schemedisplay}
(define (make-rat n d) (cons n d))

(define (numer x) (car x))

(define (denom x) (cdr x))
\end{schemedisplay}
Also, in order to display the results of our computations, we can
print rational numbers by printing the numerator, a slash, and the
denominator:\footnote{\scheme|Display| is the Scheme primitive for
  printing data.  The Scheme primitive \scheme|newline| starts a new
  line for printing.  Neither of these procedures returns a useful
  value, so in the uses of \scheme|print-rat| below, we show only what
  \scheme|print-rat| prints, not what the interpreter prints as the
  value returned by \scheme|print-rat|.}

\begin{schemedisplay}
(define (print-rat x)
  (newline)
  (display (numer x))
  (display "/")
  (display (denom x)))
\end{schemedisplay}
Now we can try our rational-number procedures:

\begin{schemedisplay}
> (define one-half (make-rat 1 2))

> (print-rat one-half)
1/2

> (define one-third (make-rat 1 3))
> (print-rat (add-rat one-half one-third))
5/6

> (print-rat (mul-rat one-half one-third))
1/6

> (print-rat (add-rat one-third one-third))
6/9
\end{schemedisplay}

As the final example shows, our rational-number implementation does
not reduce rational numbers to lowest terms.  We can remedy this by
changing \scheme|make-rat|. If we have a \scheme|gcd| procedure like the one
in section \ref{sec:1.2.5} that produces the greatest common divisor of two
integers, we can use \scheme|gcd| to reduce the numerator and the
denominator to lowest terms before constructing the pair:

\begin{schemedisplay}
(define (make-rat n d)
  (let ((g (gcd n d)))
    (cons (/ n g) (/ d g))))
\end{schemedisplay}
Now we have

\begin{schemedisplay}
(print-rat (add-rat one-third one-third))
2/3
\end{schemedisplay}
as desired.  This modification was accomplished by changing the
constructor \scheme|make-rat| without changing any of the procedures
(such as \scheme|add-rat| and \scheme|mul-rat|)
that implement the actual operations.

\begin{Exercise}
\label{exc:2.1}
Define a better version of \scheme|make-rat| that
handles both positive and negative arguments.  \scheme|Make-rat| should
normalize the sign so that if the rational number is positive, both
the numerator and denominator are positive, and if the rational number
is negative, only the numerator is negative.
\end{Exercise}


\subsection{Abstraction Barriers}
\label{sec:2.1.2}

Before continuing with more examples of compound data and data
abstraction, let us consider some of the issues raised by the
rational-number example.  We defined the rational-number operations in
terms of a constructor \scheme|make-rat| and selectors \scheme|numer| and
\scheme|denom|.  In general, the underlying idea of data abstraction is
to identify for each type of data object a basic set of operations in
terms of which all manipulations of data objects of that type will be
expressed, and then to use only those operations in manipulating the
data.

We can envision the structure of the rational-number system as
shown in figure \ref{fig:2.1}.  The
horizontal lines represent \textit{abstraction barriers} that isolate
different ``levels'' of the system.  At each level, the barrier
separates the programs (above) that use the data abstraction from the
programs (below) that implement the data abstraction.  Programs that
use rational numbers manipulate them solely in terms of the procedures
supplied ``for public use'' by the rational-number package: \scheme|add-rat|, \scheme|sub-rat|, \scheme|mul-rat|, \scheme|div-rat|, and \scheme|equal-rat?|. These, in turn, are implemented solely in terms of the
constructor and selectors \scheme|make-rat|, \scheme|numer|, and \scheme|denom|, which themselves are implemented in terms of pairs.  The
details of how pairs are implemented are irrelevant to the rest of the
rational-number package so long as pairs can be manipulated by the use
of \scheme|cons|, \scheme|car|, and \scheme|cdr|.  In effect, procedures at
each level are the interfaces that define the abstraction barriers and
connect the different levels.

\begin{figure}
\centering
\begin{tikzpicture}
  [red]
  \draw (0,0) -- (10,0) -- (10,10) -- (0,10) -- cycle ;
  \draw (0,0) -- (10,10);
  \draw (10,0) -- (0,10);
\end{tikzpicture}
\caption{Data-abstraction barriers in the rational-number package.}
\label{fig:2.1}
\end{figure}

This simple idea has many advantages.  One advantage is that it makes
programs much easier to maintain and to modify.  Any complex data
structure can be represented in a variety of ways with the primitive
data structures provided by a programming language.  Of course, the
choice of representation influences the programs that operate on it;
thus, if the representation were to be changed at some later time, all
such programs might have to be modified accordingly.  This task could
be time-consuming and expensive in the case of large programs unless
the dependence on the representation were to be confined by design to
a very few program modules.

For example, an alternate way to address the problem of reducing rational
numbers to lowest terms is to perform the reduction whenever we
access the parts of a rational number, rather than when we construct
it.  This leads to different constructor and selector procedures:

\begin{schemedisplay}
(define (make-rat n d)
  (cons n d))
(define (numer x)
  (let ((g (gcd (car x) (cdr x))))
    (/ (car x) g)))
(define (denom x)
  (let ((g (gcd (car x) (cdr x))))
    (/ (cdr x) g)))
\end{schemedisplay}
The difference between this implementation and the previous one lies
in when we compute the \scheme|gcd|.  If in our typical use of
rational numbers we access the numerators and denominators of the same
rational numbers many times, it would be preferable to compute the
\scheme|gcd| when the rational numbers are constructed.  If not, we
may be better off waiting until access time to compute the
\scheme|gcd|.  In any case, when we change from one representation to
the other, the procedures \scheme|add-rat|, \scheme|sub-rat|, and so
on do not have to be modified at all.


Constraining the dependence on the representation to a few interface
procedures helps us design programs as well as modify them, because it
allows us to maintain the flexibility to consider alternate
implementations.  To continue with our simple example, suppose we are
designing a rational-number package and we can't decide initially
whether to perform the \scheme|gcd| at construction time or at
selection time.  The data-abstraction methodology gives us a way to
defer that decision without losing the ability to make progress on the
rest of the system.

\begin{Exercise}
\label{exc:2.2}
Consider the problem of representing line segments in a plane.  Each
segment is represented as a pair of points: a starting point and an
ending point.  Define a constructor \scheme|make-segment| and
selectors \scheme|start-segment| and \scheme|end-segment| that define
the representation of segments in terms of points.  Furthermore, a
point can be represented as a pair of numbers: the $x$ coordinate and
the $y$ coordinate.  Accordingly, specify a constructor
\scheme|make-point| and selectors \scheme|x-point| and
\scheme|y-point| that define this representation.  Finally, using your
selectors and constructors, define a procedure
\scheme|midpoint-segment| that takes a line segment as argument and
returns its midpoint (the point whose coordinates are the average of
the coordinates of the endpoints).  To try your procedures, you'll
need a way to print points:

\begin{schemedisplay}
(define (print-point p)
  (newline)
  (display "(")
  (display (x-point p))
  (display ",")
  (display (y-point p))
  (display ")"))
\end{schemedisplay}
\end{Exercise}

\begin{Exercise}
\label{exc:2.3}
Implement a representation for rectangles in a plane.  (Hint: You may
want to make use of exercise \ref{exc:2.2}.)  In terms of your
constructors and selectors, create procedures that compute the
perimeter and the area of a given rectangle.  Now implement a
different representation for rectangles.  Can you design your system
with suitable abstraction barriers, so that the same perimeter and
area procedures will work using either representation?
\end{Exercise}




\subsection{What Is Meant by Data?}
\label{sec:2.1.3}



We began the rational-number implementation in
section \ref{sec:2.1.1} by implementing the rational-number
operations \scheme|add-rat|, \scheme|sub-rat|, and so on in terms of three
unspecified procedures: \scheme|make-rat|, \scheme|numer|, and \scheme|denom|.
At that point, we could think of the operations as being defined in
terms of data objects -- numerators, denominators, and rational
numbers -- whose behavior was specified by the latter three procedures.

But exactly what is meant by \textit{data}?  It is not enough to say
``whatever is implemented by the given selectors and constructors.''
Clearly, not every arbitrary set of three procedures can serve as an
appropriate basis for the rational-number implementation.  We need to
guarantee that, if we construct a rational number \scheme|x| from a
pair of integers \scheme|n| and \scheme|d|, then extracting the
\scheme|numer| and the \scheme|denom| of \scheme|x| and dividing them
should yield the same result as dividing \scheme|n| by \scheme|d|.  In
other words, \scheme|make-rat|, \scheme|numer|, and \scheme|denom|
must satisfy the condition that, for any integer \scheme|n| and any
non-zero integer \scheme|d|, if \scheme|x| is (\scheme|make-rat n d|),
then \[ \frac{\text{(numer x)}}{\text{(denom x)}} = \frac{n}{d} \]

In fact, this is the only condition \scheme|make-rat|, \scheme|numer|,
and \scheme|denom| must fulfill in order to form a suitable basis for
a rational-number representation.  In general, we can think of data as
defined by some collection of selectors and constructors, together
with specified conditions that these procedures must fulfill in order
to be a valid representation.\footnote{Surprisingly, this idea is very
  difficult to formulate rigorously. There are two approaches to
  giving such a formulation.  One, pioneered by C. A. R. Hoare (1972),
  is known as the method of \textit{abstract models}.  It formalizes
  the ``procedures plus conditions'' specification as outlined in the
  rational-number example above.  Note that the condition on the
  rational-number representation was stated in terms of facts about
  integers (equality and division).  In general, abstract models
  define new kinds of data objects in terms of previously defined
  types of data objects.  Assertions about data objects can therefore
  be checked by reducing them to assertions about previously defined
  data objects.  Another approach, introduced by Zilles at MIT, by
  Goguen, Thatcher, Wagner, and Wright at IBM (see Thatcher, Wagner,
  and Wright 1978), and by Guttag at Toronto (see Guttag 1977), is
  called \textit{algebraic specification}.  It regards the
  ``procedures'' as elements of an abstract algebraic system whose
  behavior is specified by axioms that correspond to our
  ``conditions,'' and uses the techniques of abstract algebra to check
  assertions about data objects.  Both methods are surveyed in the
  paper by Liskov and Zilles (1975).}


This point of view can serve to define not only ``high-level'' data
objects, such as rational numbers, but lower-level objects as well.
Consider the notion of a pair, which we used in order to define our
rational numbers.  We never actually said what a pair was, only that
the language supplied procedures \scheme|cons|, \scheme|car|, and
\scheme|cdr| for operating on pairs.  But the only thing we need to
know about these three operations is that if we glue two objects
together using \scheme|cons| we can retrieve the objects using
\scheme|car| and \scheme|cdr|.  That is, the operations satisfy the
condition that, for any objects \scheme|x| and \scheme|y|, if
\scheme|z| is \scheme|(cons x y)| then \scheme|(car z)| is \scheme|x|
and \scheme|(cdr z)| is \scheme|y|.  Indeed, we mentioned that these
three procedures are included as primitives in our language.  However,
any triple of procedures that satisfies the above condition can be
used as the basis for implementing pairs.  This point is illustrated
strikingly by the fact that we could implement \scheme|cons|,
\scheme|car|, and \scheme|cdr| without using any data structures at
all but only using procedures.  Here are the definitions:

\begin{schemedisplay}
(define (cons x y)
  (define (dispatch m)
    (cond ((= m 0) x)
          ((= m 1) y)
          (else (error "Argument not 0 or 1 -- CONS" m))))
  dispatch)

(define (car z) (z 0))

(define (cdr z) (z 1))
\end{schemedisplay}
This use of procedures corresponds to nothing like our intuitive
notion of what data should be.  Nevertheless, all we need to do to
show that this is a valid way to represent pairs is to verify that
these procedures satisfy the condition given above.

The subtle point to notice is that the value returned by \scheme|(cons
x y)| is a procedure -- namely the internally defined procedure
\scheme|dispatch|, which takes one argument and returns either
\scheme|x| or \scheme|y| depending on whether the argument is 0 or 1.
Correspondingly, \scheme|(car z)| is defined to apply \scheme|z| to 0.
Hence, if \scheme|z| is the procedure formed by \scheme|(cons x y)|,
then \scheme|z| applied to 0 will yield \scheme|x|. Thus, we have
shown that \scheme|(car (cons x y))| yields \scheme|x|, as desired.
Similarly, \scheme|(cdr (cons x y))| applies the procedure returned by
\scheme|(cons x y)| to 1, which returns \scheme|y|.  Therefore, this
procedural implementation of pairs is a valid implementation, and if
we access pairs using only \scheme|cons|, \scheme|car|, and
\scheme|cdr| we cannot distinguish this implementation from one that
uses ``real'' data structures.

The point of exhibiting the procedural representation of pairs is not
that our language works this way (Scheme, and Lisp systems in general,
implement pairs directly, for efficiency reasons) but that it could
work this way.  The procedural representation, although obscure, is a
perfectly adequate way to represent pairs, since it fulfills the only
conditions that pairs need to fulfill.  This example also demonstrates
that the ability to manipulate procedures as objects automatically
provides the ability to represent compound data.  This may seem a
curiosity now, but procedural representations of data will play a
central role in our programming repertoire.  This style of programming
is often called \textit{message passing}, and we will be using it as a
basic tool in chapter 3 when we address the issues of modeling and
simulation.

\begin{Exercise}
\label{exc:2.4}
Here is an alternative procedural representation of pairs.  For this
representation, verify that \scheme|(car (cons x y))| yields \scheme|x| for
any objects \scheme|x| and \scheme|y|.

\begin{schemedisplay}
(define (cons x y)
  (lambda (m) (m x y)))

(define (car z)
  (z (lambda (p q) p)))
\end{schemedisplay}
What is the corresponding definition of \scheme|cdr|? (Hint: To verify
that this works, make use of the substitution model of
section \ref{sec:1.1.5}.)



\begin{Exercise}
\label{exc:2.5}
Show that we can represent pairs of nonnegative integers using only
numbers and arithmetic operations if we represent the pair $a$ and $b$
as the integer that is the product $2^a3^b$.  Give the corresponding
definitions of the procedures \scheme|cons|, \scheme|car|, and
\scheme|cdr|.
\end{Exercise}

\begin{Exercise}
\label{exc:2.6}
In case representing pairs as procedures wasn't mind-boggling enough,
consider that, in a language that can manipulate procedures, we can
get by without numbers (at least insofar as nonnegative integers are
concerned) by implementing 0 and the operation of adding 1 as

\begin{schemedisplay}
(define zero (lambda (f) (lambda (x) x)))

(define (add-1 n)
  (lambda (f) (lambda (x) (f ((n f) x)))))
\end{schemedisplay}
This representation is known as \textit{Church numerals}, after its
inventor, Alonzo Church, the logician who invented the $\lambda$
calculus.

Define \scheme|one| and \scheme|two| directly (not in terms of \scheme|zero|
and \scheme|add-1|).  (Hint: Use substitution to evaluate \scheme|(add-1 zero)|).
Give a direct definition of the addition procedure \scheme|+| (not in
terms of repeated application of \scheme|add-1|).
\end{Exercise}


\subsection{Extended Exercise: Interval Arithmetic}
\label{sec:2.1.4}


Alyssa P. Hacker is designing a system to help people solve
engineering problems.  One feature she wants to provide in her system
is the ability to manipulate inexact quantities (such as measured
parameters of physical devices) with known precision, so that when
computations are done with such approximate quantities the results
will be numbers of known precision.

Electrical engineers will be using Alyssa's system to compute
electrical quantities.  It is sometimes necessary for them to compute
the value of a parallel equivalent resistance $R_p$ of two resistors
$R_1$ and $R_2$ using the formula \[ R_p = \frac{1}{\frac{1}{R_1} + \frac{1}{R_2}} \]

Resistance values are usually known only up to some tolerance
guaranteed by the manufacturer of the resistor.  For example, if you
buy a resistor labeled ``6.8 ohms with 10\% tolerance'' you can only
be sure that the resistor has a resistance between 6.8 - 0.68 = 6.12 and
6.8 + 0.68 = 7.48 ohms.  Thus, if you have a 6.8-ohm 10\% resistor in
parallel with a 4.7-ohm 5\% resistor, the resistance of the
combination can range from about 2.58 ohms (if the two resistors are
at the lower bounds) to about 2.97 ohms (if the two resistors are at
the upper bounds).

Alyssa's idea is to implement ``interval arithmetic'' as a set of
arithmetic operations for combining ``intervals'' (objects
that represent the range of possible values of an inexact quantity).
The result of adding, subtracting, multiplying, or dividing two
intervals is itself an interval, representing the range of the
result.

Alyssa postulates the existence of an abstract object called an
``interval'' that has two endpoints: a lower bound and an upper bound.
She also presumes that, given the endpoints of an interval, she can
construct the interval using the data constructor \scheme|make-interval|.
Alyssa first writes a procedure for adding two intervals.  She
reasons that the minimum value the sum could be is the sum of the two
lower bounds and the maximum value it could be is the sum of the two
upper bounds:

\begin{schemedisplay}
(define (add-interval x y)
  (make-interval (+ (lower-bound x) (lower-bound y))
                 (+ (upper-bound x) (upper-bound y))))
\end{schemedisplay}
Alyssa also works out the product of two intervals by finding the
minimum and the maximum of the products of the bounds and using them
as the bounds of the resulting interval.  (\scheme|Min| and \scheme|max| are
primitives that find the minimum or maximum of any number of
arguments.)

\begin{schemedisplay}
(define (mul-interval x y)
  (let ((p1 (* (lower-bound x) (lower-bound y)))
        (p2 (* (lower-bound x) (upper-bound y)))
        (p3 (* (upper-bound x) (lower-bound y)))
        (p4 (* (upper-bound x) (upper-bound y))))
    (make-interval (min p1 p2 p3 p4)
                   (max p1 p2 p3 p4))))
\end{schemedisplay}
To divide two intervals, Alyssa multiplies the first by the reciprocal of
the second.  Note that the bounds of the reciprocal interval are
the reciprocal of the upper bound and the reciprocal of the lower bound, in
that order.

\begin{schemedisplay}
(define (div-interval x y)
  (mul-interval x 
                (make-interval (/ 1.0 (upper-bound y))
                               (/ 1.0 (lower-bound y)))))
\end{schemedisplay}

\begin{Exercise}
\label{exc:2.7}
Alyssa's program is incomplete because she has not specified the
implementation of the interval abstraction.  Here is a definition of
the interval constructor:

\begin{schemedisplay}
(define (make-interval a b) (cons a b))
\end{schemedisplay}
Define selectors \scheme|upper-bound| and \scheme|lower-bound| to complete
the implementation.
\end{Exercise}

\begin{Exercise}
\label{exc:2.8}
Using reasoning analogous to Alyssa's, describe how the difference
of two intervals may be computed.  Define a corresponding subtraction
procedure, called \scheme|sub-interval|.
\end{Exercise}

\begin{Exercise}
\label{exc:2.9}
The \textit{width} of an interval is half of the difference between its
upper and lower bounds.  The width is a measure of the uncertainty of
the number specified by the interval.  For some arithmetic operations
the width of the result of combining two intervals is a function only
of the widths of the argument intervals, whereas for others the width
of the combination is not a function of the widths of the argument
intervals.  Show that the width of the sum (or difference) of two
intervals is a function only of the widths of the intervals being
added (or subtracted).  Give examples to show that this is not true
for multiplication or division.
\end{Exercise}

\begin{Exercise}
\label{exc:2.10}
Ben Bitdiddle, an expert systems programmer, looks over Alyssa's
shoulder and comments that it is not clear what it means to
divide by an interval that spans zero.  Modify Alyssa's code to
check for this condition and to signal an error if it occurs.
\end{Exercise}

\begin{Exercise}
\label{exc:2.11}
In passing, Ben also cryptically comments: ``By testing the signs of
the endpoints of the intervals, it is possible to break \scheme|mul-interval| into nine cases, only one of which requires more than
two multiplications.''  Rewrite this procedure using Ben's
suggestion.

After debugging her program, Alyssa shows it to a potential user,
who complains that her program solves the wrong problem.  He
wants a program that can deal with numbers represented as a center
value and an additive tolerance; for example, he wants to work with
intervals such as $3.5 \pm 0.15$ rather than [3.35, 3.65].  Alyssa 
returns to her desk and fixes this problem by supplying an alternate
constructor and alternate selectors:

\begin{schemedisplay}
(define (make-center-width c w)
  (make-interval (- c w) (+ c w)))
(define (center i)
  (/ (+ (lower-bound i) (upper-bound i)) 2))
(define (width i)
  (/ (- (upper-bound i) (lower-bound i)) 2))
\end{schemedisplay}

Unfortunately, most of Alyssa's users are engineers.  Real engineering
situations usually involve measurements with only a small uncertainty,
measured as the ratio of the width of the interval to the midpoint of
the interval.  Engineers usually specify percentage tolerances on the
parameters of devices, as in the resistor specifications given
earlier.
\end{Exercise}

\begin{Exercise}
\label{exc:2.12}
Define a constructor \scheme|make-center-percent| that takes a center and
a percentage tolerance and produces the desired interval.  You must
also define a selector \scheme|percent| that produces the
percentage tolerance for a given interval.  The \scheme|center| selector
is the same as the one shown above.
\end{Exercise}

\begin{Exercise}
\label{exc:2.13}
Show that under the assumption of small percentage tolerances there is
a simple formula for the approximate percentage tolerance of the
product of two intervals in terms of the tolerances of the factors.
You may simplify the problem by assuming that all numbers are
positive.
\end{Exercise}

After considerable work, Alyssa P. Hacker delivers her finished
system.  Several years later, after she has forgotten all about it, she
gets a frenzied call from an irate user,  Lem E. Tweakit.
It seems that Lem has
noticed that the formula for parallel resistors can be written in two
algebraically equivalent ways:  \[\frac{R_1R_2}{R_1 + R_2}\] and \[\frac{1}{1/R_1 + 1/R_2}\]

He has written the following two programs, each of which computes the
parallel-resistors formula differently:

\begin{schemedisplay}
(define (par1 r1 r2)
  (div-interval (mul-interval r1 r2)
                (add-interval r1 r2)))
(define (par2 r1 r2)
  (let ((one (make-interval 1 1))) 
    (div-interval one
                  (add-interval (div-interval one r1)
                                (div-interval one r2)))))
\end{schemedisplay}
Lem complains that Alyssa's program gives different answers for
the two ways of computing. This is a serious complaint.
\end{Exercise}
\begin{Exercise}
\label{exc:2.14}
Demonstrate that Lem is right. Investigate the behavior of the system
on a variety of arithmetic expressions. Make some intervals \textit{A}
and \textit{B}, and use them in computing the expressions
\textit{A}/\textit{A} and \textit{A}/\textit{B}.  You will get the
most insight by using intervals whose width is a small percentage of
the center value. Examine the results of the computation in
center-percent form (see exercise \ref{exc:2.12}).
\end{Exercise}

\begin{Exercise}
\label{exc:2.15}
Eva Lu Ator, another user, has also noticed the different intervals
computed by different but algebraically equivalent expressions. She
says that a formula to compute with intervals using Alyssa's system
will produce tighter error bounds if it can be written in such a form
that no variable that represents an uncertain number is repeated.
Thus, she says, \scheme|par2| is a ``better'' program for parallel
resistances than \scheme|par1|.  Is she right?  Why?
\end{Exercise}

\begin{Exercise}
\label{exc:2.16}
Explain, in general, why equivalent algebraic expressions may lead to
different answers.  Can you devise an interval-arithmetic package that
does not have this shortcoming, or is this task impossible?  (Warning:
This problem is very difficult.)
\end{Exercise}

% -*- TeX-master: "sicp.tex" -*-

\section{Hierarchical Data and the Closure Property}
\label{sec:2.2}

As we have seen, pairs provide a primitive ``glue'' that we can use to
construct compound data objects.  Figure \ref{fig:2.2} shows a
standard way to visualize a pair -- in this case, the pair formed by
\scheme|(cons 1 2)|.  In this representation, which is called
\textit{box-and-pointer notation}, each object is shown as a
\textit{pointer} to a box.  The box for a primitive object contains a
representation of the object.  For example, the box for a number
contains a numeral.  The box for a pair is actually a double box, the
left part containing (a pointer to) the \scheme|car| of the pair and
the right part containing the \scheme|cdr|.


We have already seen that \scheme|cons| can be used to combine not
only numbers but pairs as well.  (You made use of this fact, or
should have, in doing exercises \ref{exc:2.2}
and \ref{exc:2.3}.)  As a consequence, pairs provide a universal
building block from which we can construct all sorts of data
structures.  Figure \ref{fig:2.3} shows two ways
to use pairs to combine the numbers 1, 2, 3, and 4.

\begin{figure}
\centering
\placeholder
\caption{Box-and-pointer representation of \texttt{(cons 1 2)}}
\label{fig:2.2}
\end{figure}

\begin{figure}
\centering
\placeholder
\caption{Two ways to combine 1, 2, 3, and 4 using pairs}
\label{fig:2.3}
\end{figure}

The ability to create pairs whose elements are pairs is the essence of
list structure's importance as a representational tool.  We refer to
this ability as the \textit{closure property} of \scheme|cons|.  In
general, an operation for combining data objects satisfies the closure
property if the results of combining things with that operation can
themselves be combined using the same operation.\footnote{6} Closure
is the key to power in any means of combination because it permits us
to create \textit{hierarchical} structures -- structures made up of
parts, which themselves are made up of parts, and so on.

From the outset of chapter 1, we've made essential use of closure in
dealing with procedures, because all but the very simplest programs
rely on the fact that the elements of a combination can themselves be
combinations.  In this section, we take up the consequences of closure
for compound data.  We describe some conventional techniques for using
pairs to represent sequences and trees, and we exhibit a graphics
language that illustrates closure in a vivid way.\footnote{7}


\subsection{Representing Sequences}
\label{sec:2.2.1}



\begin{figure}
  \centering
  % TODO: create the diagram
  \caption{The sequence 1, 2, 3, 4 represented as a chain of pairs.}
  \label{fig:2.4}
\end{figure}

One of the useful structures we can build with pairs is a
\textit{sequence} -- an ordered collection of data objects.  There
are, of course, many ways to represent sequences in terms of pairs.
One particularly straightforward representation is illustrated in
figure \ref{fig:2.4}, where the sequence 1, 2, 3, 4 is represented as
a chain of pairs.  The \scheme|car| of each pair is the corresponding
item in the chain, and the \scheme|cdr| of the pair is the next pair
in the chain.  The \scheme|cdr| of the final pair signals the end of
the sequence by pointing to a distinguished value that is not a pair,
represented in box-and-pointer diagrams as a diagonal line and in
programs as the value of the variable \scheme|nil|.  The entire
sequence is constructed by nested \scheme|cons| operations:

\begin{schemedisplay}
(cons 1
      (cons 2
            (cons 3
                  (cons 4 nil))))
\end{schemedisplay}


Such a sequence of pairs, formed by nested \scheme|cons|es, is called
a \textit{list}, and Scheme provides a primitive called \scheme|list|
to help in constructing lists.\footnote{8} The above sequence could be
produced by \scheme|(list 1 2 3 4)|.  In general,


\begin{schemedisplay}
(list <a_1> <a_2> ... <a_n>)
\end{schemedisplay}
is equivalent to


\begin{schemedisplay}
(cons <a_1> (cons <a_2> (cons \scheme|...| (cons <a_n> nil) ...)))
\end{schemedisplay}
Lisp systems conventionally print lists by printing the sequence of
elements, enclosed in parentheses.  Thus, the data object in
figure \ref{fig:2.4} is printed as \scheme|(1 2 3 4)|:


\begin{schemedisplay}
> (define one-through-four (list 1 2 3 4))

> one-through-four
(1 2 3 4)
\end{schemedisplay}
Be careful not to confuse the expression \scheme|(list 1 2 3 4)| with the
list \scheme|(1 2 3 4)|, which is the result obtained when the expression
is evaluated.  Attempting to evaluate the expression \scheme|(1 2 3 4)| will
signal an error when the interpreter tries to apply the procedure \scheme|1| to arguments \scheme|2|, \scheme|3|, and \scheme|4|.


We can think of \scheme|car| as selecting the first item in the list,
and of \scheme|cdr| as selecting the sublist consisting of all but the
first item.  Nested applications of \scheme|car| and \scheme|cdr| can
be used to extract the second, third, and subsequent items in the
list.\footnote{9} The constructor \scheme|cons| makes a list like the
original one, but with an additional item at the beginning.


\begin{schemedisplay}
> (car one-through-four)
1

> (cdr one-through-four)
(2 3 4)
> (car (cdr one-through-four))
2

> (cons 10 one-through-four)
(10 1 2 3 4)

> (cons 5 one-through-four)
(5 1 2 3 4)
\end{schemedisplay}
The value of \scheme|nil|, used to terminate the chain of pairs, can
be thought of as a sequence of no elements, the \textit{empty list}.
The word \textit{nil} is a contraction of the Latin word
\textit{nihil}, which means ``nothing.''\footnote{10}


\subsubsection*{List operations}


The use of pairs to represent sequences of elements as lists is
accompanied by conventional programming techniques for manipulating
lists by successively ``\scheme|cdr|ing down'' the lists.  For example,
the procedure \scheme|list-ref| takes as arguments a list and a number
\textit{n} and returns the \textit{n}th item of the list.  It is customary to
number the elements of the list beginning with 0.  The method for
computing \scheme|list-ref| is the following:

\begin{itemize}
\item For $n = 0$, \scheme|list-ref| should return the \scheme|car| of
  the list.

\item Otherwise, \scheme|list-ref| should return the $(n - 1)$st item
  of the \scheme|cdr| of the list.
\end{itemize}

\begin{schemedisplay}
(define (list-ref items n)
  (if (= n 0)
      (car items)
      (list-ref (cdr items) (- n 1))))
(define squares (list 1 4 9 16 25))

> (list-ref squares 3)
16
\end{schemedisplay}

Often we \scheme|cdr| down the whole list.  To aid in this, Scheme
includes a primitive predicate \scheme|null?|, which tests whether its
argument is the empty list.  The procedure \scheme|length|, which
returns the number of items in a list, illustrates this typical
pattern of use:

\begin{schemedisplay}
(define (length items)
  (if (null? items)
      0
      (+ 1 (length (cdr items)))))
(define odds (list 1 3 5 7))

> (length odds)
4
\end{schemedisplay}
The \scheme|length| procedure implements a simple recursive plan. The
reduction step is:

\begin{itemize}
\item The \scheme|length| of any list is 1 plus the \scheme|length| of
  the \scheme|cdr| of the list.
\end{itemize}

This is applied successively until we reach the base case:

\begin{itemize}
\item The \scheme|length| of the empty list is 0.
\end{itemize}

We could also compute \scheme|length| in an iterative style:

\begin{schemedisplay}
(define (length items)
  (define (length-iter a count)
    (if (null? a)
        count
        (length-iter (cdr a) (+ 1 count))))
  (length-iter items 0))
\end{schemedisplay}

Another conventional programming technique is to ``\scheme|cons| up'' an
answer list while \scheme|cdr|ing down a list, as in the procedure \scheme|append|, which takes two lists as arguments and combines their
elements to make a new list:

\begin{schemedisplay}
> (append squares odds)
(1 4 9 16 25 1 3 5 7)

> (append odds squares)
(1 3 5 7 1 4 9 16 25)
\end{schemedisplay}
\scheme|Append| is also implemented using a recursive plan.  To
\scheme|append| lists \scheme|list1| and \scheme|list2|, do the
following:

\begin{itemize}
\item If \scheme|list1| is the empty list, then the result is just
  \scheme|list2|.

\item Otherwise, \scheme|append| the \scheme|cdr| of \scheme|list1|
  and \scheme|list2|, and \scheme|cons| the \scheme|car| of
  \scheme|list1| onto the result:
\end{itemize}

\begin{schemedisplay}
(define (append list1 list2)
  (if (null? list1)
      list2
      (cons (car list1) (append (cdr list1) list2))))
\end{schemedisplay}

\begin{Exercise}
\label{exc:2.17}
Define a procedure \scheme|last-pair| that returns the list that contains only
the last element of a given (nonempty) list:

\begin{schemedisplay}
> (last-pair (list 23 72 149 34))
(34)
\end{schemedisplay}
\end{Exercise}

\begin{Exercise}
\label{exc:2.18}
Define a procedure \scheme|reverse| that takes a list as argument and
returns a list of the same elements in reverse order:

\begin{schemedisplay}
> (reverse (list 1 4 9 16 25))
(25 16 9 4 1)
\end{schemedisplay}
\end{Exercise}

\begin{Exercise}
\label{exc:2.19}
Consider the change-counting program of section \ref{sec:1.2.2}.  It
would be nice to be able to easily change the currency used by the
program, so that we could compute the number of ways to change a
British pound, for example.  As the program is written, the knowledge
of the currency is distributed partly into the procedure
\scheme|first-denomination| and partly into the procedure
\scheme|count-change| (which knows that there are five kinds of
U.S. coins).  It would be nicer to be able to supply a list of coins
to be used for making change.

We want to rewrite the procedure \scheme|cc| so that its second
argument is a list of the values of the coins to use rather than an
integer specifying which coins to use.  We could then have lists that
defined each kind of currency:

\begin{schemedisplay}
(define us-coins (list 50 25 10 5 1))
(define uk-coins (list 100 50 20 10 5 2 1 0.5))
\end{schemedisplay}
We could then call \scheme|cc| as follows:

\begin{schemedisplay}
> (cc 100 us-coins)
292
\end{schemedisplay}
To do this will require changing the program \scheme|cc| somewhat.  It
will still have the same form, but it will access its second argument
differently, as follows:

\begin{schemedisplay}
(define (cc amount coin-values)
  (cond ((= amount 0) 1)
        ((or (< amount 0) (no-more? coin-values)) 0)
        (else
         (+ (cc amount
                (except-first-denomination coin-values))
            (cc (- amount
                   (first-denomination coin-values))
                coin-values)))))
\end{schemedisplay}
Define the procedures \scheme|first-denomination|,
\scheme|except-first-denomination|, and \scheme|no-more?| in terms of
primitive operations on list structures.  Does the order of the list
\scheme|coin-values| affect the answer produced by \scheme|cc|?  Why
or why not?
\end{Exercise}

\begin{Exercise}
\label{exc:2.20}
The procedures \scheme|+|, \scheme|*|, and \scheme|list| take
arbitrary numbers of arguments. One way to define such procedures is
to use \scheme|define| with \textit{dotted-tail notation}.  In a
procedure definition, a parameter list that has a dot before the last
parameter name indicates that, when the procedure is called, the
initial parameters (if any) will have as values the initial arguments,
as usual, but the final parameter's value will be a \textit{list} of
any remaining arguments.  For instance, given the definition

\begin{schemedisplay}
(define (f x y . z) \textit{<body>})
\end{schemedisplay}
the procedure \scheme|f| can be called with two or more arguments.
If we evaluate

\begin{schemedisplay}
(f 1 2 3 4 5 6)
\end{schemedisplay}
then in the body of \scheme|f|, \scheme|x| will be 1, \scheme|y| will be
2, and \scheme|z| will be the list \scheme|(3 4 5 6)|.
Given the definition

\begin{schemedisplay}
(define (g . w) \textit{<body>})
\end{schemedisplay}
the procedure \scheme|g| can be called with zero or more arguments.
If we evaluate

\begin{schemedisplay}
(g 1 2 3 4 5 6)
\end{schemedisplay}
then in the body of \scheme|g|, \scheme|w| will be the
list \scheme|(1 2 3 4 5 6)|.\footnote{11}

Use this notation
to write a procedure \scheme|same-parity| that takes one or more integers
and returns a list of all the arguments that have the same even-odd
parity as the first argument.  For example,
\begin{schemedisplay}
> (same-parity 1 2 3 4 5 6 7)
(1 3 5 7)

> (same-parity 2 3 4 5 6 7)
(2 4 6)
\end{schemedisplay}
\end{Exercise}


\subsubsection*{Mapping over lists}


One extremely useful operation is to apply some transformation
to each element in a list and generate the list of results.
For instance, the following procedure scales each number in a list by
a given factor:

\begin{schemedisplay}
(define (scale-list items factor)
  (if (null? items)
      nil
      (cons (* (car items) factor)
            (scale-list (cdr items) factor))))
> (scale-list (list 1 2 3 4 5) 10)
(10 20 30 40 50)
\end{schemedisplay}

We can abstract this general idea and capture it as a common pattern
expressed as a higher-order procedure, just as in section
\ref{sec:1.3}.  The higher-order procedure here is called
\scheme|map|.  \scheme|Map| takes as arguments a procedure of one
argument and a list, and returns a list of the results produced by
applying the procedure to each element in the list:\footnote{12}

\begin{schemedisplay}
(define (map proc items)
  (if (null? items)
      nil
      (cons (proc (car items))
            (map proc (cdr items)))))

> (map abs (list -10 2.5 -11.6 17))
(10 2.5 11.6 17)

> (map (lambda (x) (* x x))
       (list 1 2 3 4))
(1 4 9 16)
\end{schemedisplay}
Now we can give a new definition of \scheme|scale-list| in terms of \scheme|map|:
\begin{schemedisplay}
(define (scale-list items factor)
  (map (lambda (x) (* x factor))
       items))
\end{schemedisplay}


\scheme|Map| is an important construct, not only because it captures a
common pattern, but because it establishes a higher level of
abstraction in dealing with lists.  In the original definition of
\scheme|scale-list|, the recursive structure of the program draws
attention to the element-by-element processing of the list.  Defining
\scheme|scale-list| in terms of \scheme|map| suppresses that level of
detail and emphasizes that scaling transforms a list of elements to a
list of results.  The difference between the two definitions is not
that the computer is performing a different process (it isn't) but
that we think about the process differently.  In effect, \scheme|map|
helps establish an abstraction barrier that isolates the
implementation of procedures that transform lists from the details of
how the elements of the list are extracted and combined.  Like the
barriers shown in figure \ref{fig:2.1}, this abstraction gives us the
flexibility to change the low-level details of how sequences are
implemented, while preserving the conceptual framework of operations
that transform sequences to sequences.  Section \ref{sec:2.2.3}
expands on this use of sequences as a framework for organizing
programs.

\begin{Exercise}
\label{exc:2.21}
The procedure \scheme|square-list| takes a list of numbers as argument
and returns a list of the squares of those numbers.

\begin{schemedisplay}
> (square-list (list 1 2 3 4))
(1 4 9 16)
\end{schemedisplay}

Here are two different definitions of \scheme|square-list|.  Complete
both of them by filling in the missing expressions:

\begin{schemedisplay}
(define (square-list items)
  (if (null? items)
      nil
      (cons <\textit{??}> <\textit{??}>)))
(define (square-list items)
  (map <\textit{??}> <\textit{??}>))
\end{schemedisplay}
\end{Exercise}

\begin{Exercise}
\label{exc:2.22}
Louis Reasoner tries to rewrite the first \scheme|square-list| procedure of
exercise \ref{exc:2.21} so that it evolves an iterative
process:

\begin{schemedisplay}
(define (square-list items)
  (define (iter things answer)
    (if (null? things)
        answer
        (iter (cdr things) 
              (cons (square (car things))
                    answer))))
  (iter items nil))
\end{schemedisplay}
Unfortunately, defining \scheme|square-list| this way produces the
answer list in the reverse order of the one desired.  Why?

Louis then tries to fix his bug by interchanging the arguments to
\scheme|cons|:

\begin{schemedisplay}
(define (square-list items)
  (define (iter things answer)
    (if (null? things)
        answer
        (iter (cdr things)
              (cons answer
                    (square (car things))))))
  (iter items nil))
\end{schemedisplay}
This doesn't work either.  Explain.
\end{Exercise}

\begin{Exercise}
\label{exc:2.23}
The procedure \scheme|for-each| is similar to \scheme|map|.  It takes
as arguments a procedure and a list of elements.  However, rather than
forming a list of the results, \scheme|for-each| just applies the
procedure to each of the elements in turn, from left to right.  The
values returned by applying the procedure to the elements are not used
at all -- \scheme|for-each| is used with procedures that perform an
action, such as printing.  For example,
\begin{schemedisplay}
(for-each (lambda (x) (newline) (display x))
          (list 57 321 88))
<i>57</i>
<i>321</i>
<i>88</i>
\end{schemedisplay}
The value returned by the call to \scheme|for-each| (not illustrated
above) can be something arbitrary, such as true.  Give an
implementation of \scheme|for-each|.
\end{Exercise}

\subsection{Hierarchical Structures}
\label{sec:2.2.2}

The representation of sequences in terms of lists generalizes
naturally to represent sequences whose elements may
themselves be sequences.  For example, we can regard the object
\scheme|((1 2) 3 4)| constructed by

\begin{schemedisplay}
(cons (list 1 2) (list 3 4))
\end{schemedisplay}
as a list of three items, the first of which is itself a list,
\scheme|(1 2)|.  Indeed, this is suggested by the form in which the
result is printed by the interpreter.  Figure \ref{fig:2.5} shows the
representation of this structure in terms of pairs.

\begin{schemeregion}
  \begin{figure}
    \centering
    % TODO: fill in this diagram
    \caption{Structure formed by \scheme|(cons (list 1 2) (list 3 4))|}
    \label{fig:2.5}
  \end{figure}
\end{schemeregion}

Another way to think of sequences whose elements are sequences is as
\textit{trees}.  The elements of the sequence are the branches of the
tree, and elements that are themselves sequences are subtrees.  Figure
\ref{fig:2.6} shows the structure in figure \ref{fig:2.5} viewed as a
tree.

\begin{figure}
  \centering
  % TODO: fill in this diagram
  \caption{The list structure in figure \ref{fig:2.5} viewed as a tree.}
  \label{fig:2.6}
\end{figure}

Recursion is a natural tool for dealing with tree structures, since we
can often reduce operations on trees to operations on their branches,
which reduce in turn to operations on the branches of the branches,
and so on, until we reach the leaves of the tree.  As an example,
compare the \scheme|length| procedure of section \ref{sec:2.2.1} with
the \scheme|count-leaves| procedure, which returns the total number of
leaves of a tree:


\begin{schemedisplay}
(define x (cons (list 1 2) (list 3 4)))

> (length x)
3
> (count-leaves x)
4

> (list x x)
(((1 2) 3 4) ((1 2) 3 4))

> (length (list x x))
2

> (count-leaves (list x x))
8
\end{schemedisplay}

To implement \scheme|count-leaves|, recall the recursive plan for computing
\scheme|length|:

\begin{itemize}
\item \scheme|Length| of a list \scheme|x| is 1 plus \scheme|length|
  of the \scheme|cdr| of \scheme|x|.

\item \scheme|Length| of the empty list is 0.
\end{itemize}

\scheme|Count-leaves| is similar.  The value for the empty list is the same:

\begin{itemize}
\item \scheme|Count-leaves| of the empty list is 0.
\end{itemize}

But in the reduction step, where we strip off the \scheme|car| of the
list, we must take into account that the \scheme|car| may itself be a
tree whose leaves we need to count.  Thus, the appropriate reduction
step is


\begin{itemize}
\item \scheme|Count-leaves| of a tree \scheme|x| is
  \scheme|count-leaves| of the \scheme|car| of \scheme|x| plus
  \scheme|count-leaves| of the \scheme|cdr| of \scheme|x|.
\end{itemize}

Finally, by taking \scheme|car|s we reach actual leaves, so we need
another base case:


\begin{itemize}
\item \scheme|Count-leaves| of a leaf is 1.
\end{itemize}

To aid in writing recursive procedures on trees, Scheme provides the
primitive predicate \scheme|pair?|, which tests whether its argument
is a pair.  Here is the complete procedure:\footnote{13}


\begin{schemedisplay}
(define (count-leaves x)
  (cond ((null? x) 0)  
        ((not (pair? x)) 1)
        (else (+ (count-leaves (car x))
                 (count-leaves (cdr x))))))
\end{schemedisplay}

\begin{Exercise}
\label{exc:2.24}
Suppose we evaluate the expression \scheme|(list 1 (list 2 (list 3 4)))|.
Give the result printed by the interpreter, the corresponding
box-and-pointer structure, and the interpretation of this as a tree
(as in figure \ref{fig:2.6}).
\end{Exercise}

\begin{Exercise}
\label{exc:2.25}
Give combinations of \scheme|car|s and \scheme|cdr|s that will pick 7 from
each of the following lists:

\begin{schemedisplay}
(1 3 (5 7) 9)

((7))

(1 (2 (3 (4 (5 (6 7))))))
\end{schemedisplay}
\end{Exercise}

\begin{Exercise}
\label{exc:2.26}
Suppose we define \scheme|x| and \scheme|y| to be two lists:

\begin{schemedisplay}
(define x (list 1 2 3))
(define y (list 4 5 6))
\end{schemedisplay}
What result is printed by the interpreter in response to evaluating
each of the following expressions:

\begin{schemedisplay}
(append x y)

(cons x y)

(list x y)
\end{schemedisplay}
\end{Exercise}

\begin{Exercise}
\label{exc:2.27}
Modify your \scheme|reverse| procedure of exercise \ref{exc:2.18} to
produce a \scheme|deep-reverse| procedure that takes a list as argument
and returns as its value the list with its elements reversed and with
all sublists deep-reversed as well.  For example,

\begin{schemedisplay}
(define x (list (list 1 2) (list 3 4)))

x
<i>((1 2) (3 4))</i>

(reverse x)
<i>((3 4) (1 2))</i>

(deep-reverse x)
<i>((4 3) (2 1))</i>
\end{schemedisplay}
\end{Exercise}

\begin{Exercise}
\label{exc:2.28}
Write a procedure \scheme|fringe| that takes as argument a tree
(represented as a list) and returns a list whose elements are all the
leaves of the tree arranged in left-to-right order.  For example,

\begin{schemedisplay}
(define x (list (list 1 2) (list 3 4)))

(fringe x)
<i>(1 2 3 4)</i>

(fringe (list x x))
<i>(1 2 3 4 1 2 3 4)</i>
\end{schemedisplay}
\end{Exercise}

\begin{Exercise}
\label{exc:2.29}
A binary mobile consists of two branches, a left branch and a right
branch.  Each branch is a rod of a certain length, from which hangs
either a weight or another binary mobile.  We can represent a binary
mobile using compound data by constructing it from two branches (for
example, using \scheme|list|):


\begin{schemedisplay}
(define (make-mobile left right)
  (list left right))
\end{schemedisplay}
A branch is constructed from a \scheme|length| (which must be a number)
together with a \scheme|structure|, which may be either a number
(representing a simple weight) or another mobile:


\begin{schemedisplay}
(define (make-branch length structure)
  (list length structure))
\end{schemedisplay}

% TODO: use enumerate
a.  Write the corresponding selectors \scheme|left-branch| and
\scheme|right-branch|, which return the branches of a mobile, and
\scheme|branch-length| and \scheme|branch-structure|, which return 
the components of a branch.

b.  Using your selectors, define a procedure \scheme|total-weight| 
that returns the total weight of a mobile.

c.  A mobile is said to be \textit{balanced} if the torque applied
by its top-left branch is equal to that applied by its top-right
branch (that is, if the length of the left rod multiplied by the
weight hanging from that rod is equal to the corresponding product for
the right side) and if each of the submobiles hanging off its branches
is balanced. Design a predicate that tests whether a binary mobile is
balanced.

d.  Suppose we change the representation of mobiles so that the
constructors are

\begin{schemedisplay}
(define (make-mobile left right)
  (cons left right))
(define (make-branch length structure)
  (cons length structure))
\end{schemedisplay}
How much do you need to change your programs to convert to the new
representation?
\end{Exercise}

\subsubsection*{Mapping over trees}

Just as \scheme|map| is a powerful abstraction for dealing with
sequences, \scheme|map| together with recursion is a powerful
abstraction for dealing with trees.  For instance, the
\scheme|scale-tree| procedure, analogous to \scheme|scale-list| of
section \ref{sec:2.2.1}, takes as arguments a numeric factor and a
tree whose leaves are numbers.  It returns a tree of the same shape,
where each number is multiplied by the factor.  The recursive plan for
\scheme|scale-tree| is similar to the one for \scheme|count-leaves|:

\begin{schemedisplay}
> (define (scale-tree tree factor)
    (cond ((null? tree) nil)
          ((not (pair? tree)) (* tree factor))
          (else (cons (scale-tree (car tree) factor)
                      (scale-tree (cdr tree) factor)))))
> (scale-tree (list 1 (list 2 (list 3 4) 5) (list 6 7))
              10)
(10 (20 (30 40) 50) (60 70))
\end{schemedisplay}

Another way to implement \scheme|scale-tree| is to regard the tree as
a sequence of sub-trees and use \scheme|map|.  We map over the
sequence, scaling each sub-tree in turn, and return the list of
results.  In the base case, where the tree is a leaf, we simply
multiply by the factor:

\begin{schemedisplay}
(define (scale-tree tree factor)
  (map (lambda (sub-tree)
         (if (pair? sub-tree)
             (scale-tree sub-tree factor)
             (* sub-tree factor)))
       tree))
\end{schemedisplay}
Many tree operations can be implemented by similar combinations of
sequence operations and recursion.


\label{exc:2.30}
Define a procedure \scheme|square-tree| analogous to the
\scheme|square-list| procedure of exercise \ref{exc:2.21}.  That is,
\scheme|square-list| should behave as follows:

\begin{schemedisplay}
> (square-tree
   (list 1
         (list 2 (list 3 4) 5)
         (list 6 7)))
(1 (4 (9 16) 25) (36 49))
\end{schemedisplay}
Define \scheme|square-tree| both directly (i.e., without using any
higher-order procedures) and also by using \scheme|map| and recursion.
\end{Exercise}

\begin{Exercise}
\label{exc:2.31}
Abstract your answer to exercise \ref{exc:2.30} to produce a procedure
\scheme|tree-map| with the property that \scheme|square-tree| could be
defined as

\begin{schemedisplay}
(define (square-tree tree) (tree-map square tree))
\end{schemedisplay}
\end{Exercise}

\begin{Exercise}
\label{exc:2.32}
We can represent a set as a list of distinct elements, and we can
represent the set of all subsets of the set as a list of lists.  For
example, if the set is \scheme|(1 2 3)|, then the set of all subsets
is \scheme|(() (3) (2) (2 3) (1) (1 3) (1 2) (1 2 3))|.  Complete the
following definition of a procedure that generates the set of subsets
of a set and give a clear explanation of why it works:
\begin{schemedisplay}
(define (subsets s)
  (if (null? s)
      (list nil)
      (let ((rest (subsets (cdr s))))
        (append rest (map <??> rest)))))
\end{schemedisplay}
\end{Exercise}

\subsection{Sequences as Conventional Interfaces}
\label{sec:2.2.3}

In working with compound data, we've stressed how data abstraction
permits us to design programs without becoming enmeshed in the details
of data representations, and how abstraction preserves for us the
flexibility to experiment with alternative representations.  In this
section, we introduce another powerful design principle for working
with data structures -- the use of \textit{conventional interfaces}.

In section \ref{sec:1.3} we saw how program abstractions, implemented
as higher-order procedures, can capture common patterns in programs
that deal with numerical data.  Our ability to formulate analogous
operations for working with compound data depends crucially on the
style in which we manipulate our data structures.  Consider, for
example, the following procedure, analogous to the
\scheme|count-leaves| procedure of section \ref{sec:2.2.2}, which
takes a tree as argument and computes the sum of the squares of the
leaves that are odd:
\begin{schemedisplay}
(define (sum-odd-squares tree)
  (cond ((null? tree) 0)  
        ((not (pair? tree))
         (if (odd? tree) (square tree) 0))
        (else (+ (sum-odd-squares (car tree))
                 (sum-odd-squares (cdr tree))))))
\end{schemedisplay}

On the surface, this procedure is very different from the following
one, which constructs a list of all the even Fibonacci numbers
 $\text{Fib}(k)$, where $k$ is less than or equal to a given integer $n$:
\begin{schemedisplay}
(define (even-fibs n)
  (define (next k)
    (if (> k n)
        nil
        (let ((f (fib k)))
          (if (even? f)
              (cons f (next (+ k 1)))
              (next (+ k 1))))))
  (next 0))
\end{schemedisplay}

Despite the fact that these two procedures are structurally very
different, a more abstract description of the two computations reveals
a great deal of similarity.  The first program
\begin{itemize}
\item enumerates the leaves of a tree;
\item filters them, selecting the odd ones;
\item squares each of the selected ones; and
\item accumulates the results using \scheme|+|, starting with 0.
\end{itemize}
The second program
\begin{itemize}
\item enumerates the integers from 0 to \textit{n};
\item computes the Fibonacci number for each integer;
\item filters them, selecting the even ones; and
\item accumulates the results using \scheme|cons|, starting with the
  empty list.
\end{itemize}

A signal-processing engineer would find it natural to conceptualize
these processes in terms of signals flowing through a cascade of
stages, each of which implements part of the program plan, as shown in
figure \ref{fig:2.7}.  In \scheme|sum-odd-squares|, we begin with an
\textit{enumerator}, which generates a ``signal'' consisting of the
leaves of a given tree.  This signal is passed through a
\textit{filter}, which eliminates all but the odd elements.  The
resulting signal is in turn passed through a \textit{map}, which is a
``transducer'' that applies the \scheme|square| procedure to each
element.  The output of the map is then fed to an
\textit{accumulator}, which combines the elements using \scheme|+|,
starting from an initial 0.  The plan for \scheme|even-fibs| is
analogous.

\begin{schemeregion}
  \begin{figure}
    % TODO
    \centering
    \caption{The signal-flow plans for the procedures
      \scheme|sum-odd-squares| (top) and \scheme|even-fibs| (bottom)
      reveal the commonality between the two programs.}
    \label{fig:2.7}
  \end{figure}
\end{schemeregion}

Unfortunately, the two procedure definitions above fail to exhibit
this signal-flow structure.  For instance, if we examine the
\scheme|sum-odd-squares| procedure, we find that the enumeration is
implemented partly by the \scheme|null?| and \scheme|pair?| tests and
partly by the tree-recursive structure of the procedure.  Similarly,
the accumulation is found partly in the tests and partly in the
addition used in the recursion.  In general, there are no distinct
parts of either procedure that correspond to the elements in the
signal-flow description.  Our two procedures decompose the
computations in a different way, spreading the enumeration over the
program and mingling it with the map, the filter, and the
accumulation.  If we could organize our programs to make the
signal-flow structure manifest in the procedures we write, this would
increase the conceptual clarity of the resulting code.

\subsubsection*{Sequence Operations}

The key to organizing programs so as to more clearly reflect the
signal-flow structure is to concentrate on the ``signals'' that flow
from one stage in the process to the next.  If we represent these
signals as lists, then we can use list operations to implement the
processing at each of the stages.  For instance, we can implement the
mapping stages of the signal-flow diagrams using the \scheme|map|
procedure from section \ref{sec:2.2.1}:

\begin{schemedisplay}
> (map square (list 1 2 3 4 5))
(1 4 9 16 25)
\end{schemedisplay}

Filtering a sequence to select only those elements that satisfy a
given predicate is accomplished by

\begin{schemedisplay}
(define (filter predicate sequence)
  (cond ((null? sequence) nil)
        ((predicate (car sequence))
         (cons (car sequence)
               (filter predicate (cdr sequence))))
        (else (filter predicate (cdr sequence)))))
\end{schemedisplay}
For example,
\begin{schemedisplay}
> (filter odd? (list 1 2 3 4 5))
(1 3 5)
\end{schemedisplay}

Accumulations can be implemented by
\begin{schemedisplay}
> (define (accumulate op initial sequence)
    (if (null? sequence)
        initial
        (op (car sequence)
            (accumulate op initial (cdr sequence)))))
> (accumulate + 0 (list 1 2 3 4 5))
15
> (accumulate * 1 (list 1 2 3 4 5))
120
> (accumulate cons nil (list 1 2 3 4 5))
(1 2 3 4 5)
\end{schemedisplay}

All that remains to implement signal-flow diagrams is to enumerate the
sequence of elements to be processed.  For \scheme|even-fibs|, we need
to generate the sequence of integers in a given range, which we can do
as follows:
\begin{schemedisplay}
> (define (enumerate-interval low high)
    (if (> low high)
        nil
        (cons low (enumerate-interval (+ low 1) high))))
> (enumerate-interval 2 7)
(2 3 4 5 6 7)
\end{schemedisplay}

To enumerate the leaves of a tree, we can use\footnote{14}
\begin{schemedisplay}
> (define (enumerate-tree tree)
    (cond ((null? tree) nil)
          ((not (pair? tree)) (list tree))
          (else (append (enumerate-tree (car tree))
                        (enumerate-tree (cdr tree))))))
> (enumerate-tree (list 1 (list 2 (list 3 4)) 5))
(1 2 3 4 5)
\end{schemedisplay}

Now we can reformulate \scheme|sum-odd-squares| and \scheme|even-fibs|
as in the signal-flow diagrams.  For \scheme|sum-odd-squares|, we
enumerate the sequence of leaves of the tree, filter this to keep only
the odd numbers in the sequence, square each element, and sum the
results:

\begin{schemedisplay}
(define (sum-odd-squares tree)
  (accumulate +
              0
              (map square
                   (filter odd?
                           (enumerate-tree tree)))))
\end{schemedisplay}
For \scheme|even-fibs|, we enumerate the integers from 0 to $n$,
generate the Fibonacci number for each of these integers, filter the
resulting sequence to keep only the even elements, and accumulate the
results into a list:

\begin{schemedisplay}
(define (even-fibs n)
  (accumulate cons
              nil
              (filter even?
                      (map fib
                           (enumerate-interval 0 n)))))
\end{schemedisplay}


The value of expressing programs as sequence operations is that this
helps us make program designs that are modular, that is, designs that
are constructed by combining relatively independent pieces.  We can
encourage modular design by providing a library of standard components
together with a conventional interface for connecting the components
in flexible ways.

Modular construction is a powerful strategy for
controlling complexity in engineering design.  In real
signal-processing applications, for example, designers regularly build
systems by cascading elements selected from standardized families of
filters and transducers.  Similarly, sequence operations provide a
library of standard program elements that we can mix and match.  For
instance, we can reuse pieces from the \scheme|sum-odd-squares| and \scheme|even-fibs| procedures in a program that constructs a list of the
squares of the first $n+1$ Fibonacci numbers:

\begin{schemedisplay}
> (define (list-fib-squares n)
    (accumulate cons
                nil
                (map square
                     (map fib
                          (enumerate-interval 0 n)))))
> (list-fib-squares 10)
(0 1 1 4 9 25 64 169 441 1156 3025)
\end{schemedisplay}

We can rearrange the pieces and use them in computing the product of
the odd integers in a sequence:
\begin{schemedisplay}
> (define (product-of-squares-of-odd-elements sequence)
    (accumulate *
                1
                (map square
                     (filter odd? sequence))))
> (product-of-squares-of-odd-elements (list 1 2 3 4 5))
225
\end{schemedisplay}

We can also formulate conventional data-processing applications in
terms of sequence operations.  Suppose we have a sequence of personnel
records and we want to find the salary of the highest-paid programmer.
Assume that we have a selector \scheme|salary| that returns the salary of
a record, and a predicate \scheme|programmer?| that tests if a record is
for a programmer.  Then we can write
\begin{schemedisplay}
(define (salary-of-highest-paid-programmer records)
  (accumulate max
              0
              (map salary
                   (filter programmer? records))))
\end{schemedisplay}
These examples give just a hint of the vast range of operations that
can be expressed as sequence operations.\footnote{15}

Sequences, implemented here as lists, serve as a conventional
interface that permits us to combine processing modules.
Additionally, when we uniformly represent structures as sequences, we
have localized the data-structure dependencies in our programs to a
small number of sequence operations.  By changing these, we can
experiment with alternative representations of sequences, while
leaving the overall design of our programs intact.  We will exploit
this capability in section \ref{sec:3.5}, when we generalize the
sequence-processing paradigm to admit infinite sequences.

\begin{Exercise}
\label{exc:2.33}
Fill in the missing expressions to complete the following definitions
of some basic list-manipulation operations as accumulations:
\begin{schemedisplay}
(define (map p sequence)
  (accumulate (lambda (x y) <??>) nil sequence))
(define (append seq1 seq2)
  (accumulate cons <??> <??>))
(define (length sequence)
  (accumulate <??> 0 sequence))
\end{schemedisplay}


\begin{Exercise}
\label{exc:2.34}
Evaluating a polynomial in $x$ at a given value of $x$ can be
formulated as an accumulation.  We evaluate the polynomial 
\[a_nx^n + a_{n-1}x^{n-1} + \cdots + a_1x + a_0 \] using a well-known
algorithm called \textit{Horner's rule}, which structures the
computation as \[ (\cdots (a_nx + a_{n-1})x+ \cdots + a_1)x + a_0 \]
In other words, we start with $a_n$, multiply by $x$, add $a_{n-1}$,
multiply by $x$, and so on, until we reach $a_0$.\footnote{16} Fill in
the following template to produce a procedure that evaluates a
polynomial using Horner's rule.  Assume that the coefficients of the
polynomial are arranged in a sequence, from $a_0$ through $a_n$.
\begin{schemedisplay}
(define (horner-eval x coefficient-sequence)
  (accumulate (lambda (this-coeff higher-terms) <??>)
              0
              coefficient-sequence))
\end{schemedisplay}
For example, to compute $1 + 3x + 5x^3 + x^5$ at $x = 2$ you would evaluate
\begin{schemedisplay}
(horner-eval 2 (list 1 3 0 5 0 1))
\end{schemedisplay}
\end{Exercise}

\begin{Exercise}
\label{exc:2.35}
Redefine \scheme|count-leaves| from section \ref{sec:2.2.2} as an
accumulation:
\begin{schemedisplay}
(define (count-leaves t)
  (accumulate <??> <??> (map <??> <??>)))
\end{schemedisplay}
\end{Exercise}

\begin{Exercise}
\label{exc:2.36}
The procedure \scheme|accumulate-n| is similar to \scheme|accumulate|
except that it takes as its third argument a sequence of sequences,
which are all assumed to have the same number of elements.  It applies
the designated accumulation procedure to combine all the first
elements of the sequences, all the second elements of the sequences,
and so on, and returns a sequence of the results.  For instance, if
\scheme|s| is a sequence containing four sequences, \scheme|((1 2 3)
(4 5 6) (7 8 9) (10 11 12)),| then the value of \scheme|(accumulate-n
+ 0 s)| should be the sequence \scheme|(22 26 30)|.  Fill in the
missing expressions in the following definition of
\scheme|accumulate-n|:
\begin{schemedisplay}
(define (accumulate-n op init seqs)
  (if (null? (car seqs))
      nil
      (cons (accumulate op init <??>)
            (accumulate-n op init <??>))))
\end{schemedisplay}
\end{Exercise}

\begin{Exercise}
\label{exc:2.37}
Suppose we represent vectors $v = (v_i)$ as sequences of numbers, and
matrices $m = (m_{ij})$ as sequences of vectors (the rows of the matrix).
For example, the matrix
\begin{displaymath}
  \left[
    \begin{array}{cccc}
      1 & 2 & 3 & 4 \\
      4 & 5 & 6 & 6 \\
      6 & 7 & 8 & 9 \\
    \end{array}
  \right]
\end{displaymath}

is represented as the sequence \scheme|((1 2 3 4) (4 5 6 6) (6 7 8 9))|.
With this representation, we can use sequence operations to concisely
express the basic matrix and vector operations.  These operations
(which are described in any book on matrix algebra) are the following:

\begin{schemeregion}
  \begin{tabular}{ll}
    \scheme|(dot-product v w)| & returns the sum $\sum_i v_iw_i$ \\
    \scheme|(matrix-*-vector m v)| & returns the vector $t$, where $t_i = \sum_j m_{ij}v_j$ \\
    \scheme|(matrix-*-matrix m n)| & returns the matrix $p$, where $p_i_j = \sum_k m_i_kn_k_j$ \\
    \scheme|(transpose m)| & returns the matrix $n$ , where $n_i_j = m_j_i$ \\
    
  \end{tabular}
\end{schemeregion}


We can define the dot product as\footnote{17}
\begin{schemedisplay}
(define (dot-product v w)
  (accumulate + 0 (map * v w)))
\end{schemedisplay}
Fill in the missing expressions in the following procedures for
computing the other matrix operations.  (The procedure \scheme|accumulate-n| is
defined in exercise \ref{exc:2.36}.)
\begin{schemedisplay}
(define (matrix-*-vector m v)
  (map <??> m))
(define (transpose mat)
  (accumulate-n <??> <??> mat))
(define (matrix-*-matrix m n)
  (let ((cols (transpose n)))
    (map <??> m)))
\end{schemedisplay}
\end{Exercise}

\begin{Exercise}
\label{exc:2.38}
The \scheme|accumulate| procedure is also known as
\scheme|fold-right|, because it combines the first element of the
sequence with the result of combining all the elements to the right.
There is also a \scheme|fold-left|, which is similar to
\scheme|fold-right|, except that it combines elements working in the
opposite direction:
\begin{schemedisplay}
(define (fold-left op initial sequence)
  (define (iter result rest)
    (if (null? rest)
        result
        (iter (op result (car rest))
              (cdr rest))))
  (iter initial sequence))
\end{schemedisplay}
What are the values of
\begin{schemedisplay}
(fold-right / 1 (list 1 2 3))
(fold-left / 1 (list 1 2 3))
(fold-right list nil (list 1 2 3))
(fold-left list nil (list 1 2 3))
\end{schemedisplay}
Give a property that \scheme|op| should satisfy to guarantee that \scheme|fold-right| and \scheme|fold-left| will produce the same values for any
sequence.
\end{Exercise}

\begin{Exercise}
\label{exc:2.39}

Complete the following definitions of \scheme|reverse|
(exercise \ref{exc:2.18}) in terms of \scheme|fold-right| and \scheme|fold-left| from exercise \ref{exc:2.38}:
\begin{schemedisplay}
(define (reverse sequence)
  (fold-right (lambda (x y) <\textit{??}>) nil sequence))
(define (reverse sequence)
  (fold-left (lambda (x y) <\textit{??}>) nil sequence))
\end{schemedisplay}
\end{Exercise}

\subsubsection*{Nested Mappings}

We can extend the sequence paradigm to include many computations that
are commonly expressed using nested loops.\footnote{18} Consider this
problem: Given a positive integer $n$, find all ordered pairs of
distinct positive integers $i$ and $j$, where $ 1 \le j < i \le n$,
such that $i + j$ is prime.  For example, if $n$ is 6, then the pairs
are the following: 

\begin{tabular}{c|ccccccc}
  $i$ & 2 & 3 & 4 & 4 & 5 & 6 & 6 \\
  $j$ & 1 & 2 & 1 & 3 & 2 & 1 & 5 \\
  \hline{}
  $ i + j$ & 3 & 5 & 5 & 7 & 7 & 7 & 11
\end{tabular}

A natural way to organize this computation is to generate the sequence
of all ordered pairs of positive integers less than or equal to $n$,
filter to select those pairs whose sum is prime, and then, for each
pair $(i,j)$ that passes through the filter, produce the triple
$(i,j,i+j)$.

Here is a way to generate the sequence of pairs: For each integer $i
\le n$, enumerate the integers $j < i$, and for each such $i$ and $j$
generate the pair $(i,j)$.  In terms of sequence operations, we map
along the sequence \scheme|(enumerate-interval 1 n)|.  For each $i$ in
this sequence, we map along the sequence \scheme|(enumerate-interval 1
(- i 1))|.  For each $j$ in this latter sequence, we generate the pair
\scheme|(list i j)|.  This gives us a sequence of pairs for each $i$.
Combining all the sequences for all the $j$ (by accumulating with
\scheme|append|) produces the required sequence of pairs:\footnote{19}
\begin{schemedisplay}
(accumulate append
            nil
            (map (lambda (i)
                   (map (lambda (j) (list i j))
                        (enumerate-interval 1 (- i 1))))
                 (enumerate-interval 1 n)))
\end{schemedisplay}
The combination of mapping and accumulating with \scheme|append| is so
common in this sort of program that we will isolate it as a separate
procedure:
\begin{schemedisplay}
(define (flatmap proc seq)
  (accumulate append nil (map proc seq)))
\end{schemedisplay}
Now filter this sequence of pairs to find those whose sum is
prime. The filter predicate is called for each element of the
sequence; its argument is a pair and it must extract the integers from
the pair.  Thus, the predicate to apply to each element in the
sequence is
\begin{schemedisplay}
(define (prime-sum? pair)
  (prime? (+ (car pair) (cadr pair))))
\end{schemedisplay}
Finally, generate the sequence of results by mapping over the filtered
pairs using the following procedure, which constructs a triple
consisting of the two elements of the pair along with their sum:
\begin{schemedisplay}
(define (make-pair-sum pair)
  (list (car pair) (cadr pair) (+ (car pair) (cadr pair))))
\end{schemedisplay}
Combining all these steps yields the complete procedure:
\begin{schemedisplay}
(define (prime-sum-pairs n)
  (map make-pair-sum
       (filter prime-sum?
               (flatmap
                (lambda (i)
                  (map (lambda (j) (list i j))
                       (enumerate-interval 1 (- i 1))))
                (enumerate-interval 1 n)))))
\end{schemedisplay}

Nested mappings are also useful for sequences other than those that
enumerate intervals.  Suppose we wish to generate all the permutations
of a set $S$; that is, all the ways of ordering the items in the set.
For instance, the permutations of \{1,2,3\} are \{1,2,3\}, \{ 1,3,2\},
\{2,1,3\}, \{ 2,3,1\}, \{ 3,1,2\}, and \{ 3,2,1\}.  Here is a plan for
generating the permutations of $S$: For each item $x$ in $S$,
recursively generate the sequence of permutations of $S
-x$,\footnote{20} and adjoin $x$ to the front of each one.  This
yields, for each $x$ in $S$, the sequence of permutations of $S$ that
begin with $x$.  Combining these sequences for all $x$ gives all the
permutations of $S$:\footnote{21}


\begin{schemedisplay}
(define (permutations s)
  (if (null? s)                    ; empty set?
      (list nil)                   ; sequence containing empty set
      (flatmap (lambda (x)
                 (map (lambda (p) (cons x p))
                      (permutations (remove x s))))
               s)))
\end{schemedisplay}
Notice how this strategy reduces the problem of generating
permutations of $S$ to the problem of generating the permutations of
sets with fewer elements than $S$.  In the terminal case, we work our
way down to the empty list, which represents a set of no elements.
For this, we generate \scheme|(list nil)|, which is a sequence with one
item, namely the set with no elements.  The \scheme|remove| procedure
used in \scheme|permutations| returns all the items in a given sequence
except for a given item.  This can be expressed as a simple filter:

\begin{schemedisplay}
(define (remove item sequence)
  (filter (lambda (x) (not (= x item)))
          sequence))
\end{schemedisplay}

\begin{Exercise}
\label{exc:2.40}
Define a procedure \scheme|unique-pairs| that, given an integer $n$,
generates the sequence of pairs $(i,j)$ with $ 1 \le j < i \le n$.
Use \scheme|unique-pairs| to simplify the definition of
\scheme|prime-sum-pairs| given above.
\end{Exercise}

\begin{Exercise}
\label{exc:2.41}
Write a procedure to find all ordered
triples of distinct positive integers $i$, $j$, and $k$ less than or
equal to a given integer $n$ that sum to a given integer $s$.
\end{Exercise}


\begin{Exercise}
\label{exc:2.42}

\begin{figure}
\centering
\begin{tikzpicture}
  []
  \foreach \i in {0,1,2,3,4,5,6,7,8} {
    \draw (0,\i) -- (8,\i);
    \draw (\i,0) -- (\i,8);
  }
  \draw (8,8) (0,0);
  \foreach \p in {(5,0), (2,1), (0,2), (6,3), (4,4), (7,5), (1,6), (3,7)}
  \draw \p +(0.2,0.2) -- +(0.8,0.2) -- +(0.9,0.9) -- +(0.8,0.8) -- +(0.7,0.9) -- +(0.6,0.8) -- +(0.5,0.9) -- +(0.4,0.8) -- +(0.3,0.9) -- +(0.2,0.8) -- +(0.1,0.9) -- cycle; 
\end{tikzpicture}
\end{figure}
<a name="%_fig_2.8"></a><div align=left><table width=100%><tr><td><img src="ch2-Z-G-23.gif" border="0">


</td></tr><caption align=bottom><div align=left><b>Figure 2.8:</b>  A solution to the eight-queens puzzle.</div></caption><tr><td>

</td></tr></table></div>
The ``eight-queens puzzle'' asks how to place eight queens on a
chessboard so that no queen is in check from any other (i.e., no two
queens are in the same row, column, or diagonal).  One possible
solution is shown in figure \ref{fig:2.8}.  One way to solve the
puzzle is to work across the board, placing a queen in each column.
Once we have placed $k - 1$ queens, we must place the $k$th queen in a
position where it does not check any of the queens already on the
board.  We can formulate this approach recursively: Assume that we
have already generated the sequence of all possible ways to place
$k - 1$ queens in the first $k - 1$ columns of the board.  For each of
these ways, generate an extended set of positions by placing a queen
in each row of the $k$th column.  Now filter these, keeping only
the positions for which the queen in the $k$th column is safe with
respect to the other queens.  This produces the sequence of all ways
to place $k$ queens in the first $k$ columns.  By continuing this
process, we will produce not only one solution, but all solutions to
the puzzle.


We implement this solution as a procedure \scheme|queens|, which returns
a sequence of all solutions to the problem of placing $n$ queens on an
$n \times n$ chessboard.  \scheme|Queens| has an internal procedure \scheme|queen-cols| that returns the sequence of all ways to place queens in
the first $k$ columns of the board.

\begin{schemedisplay}
(define (queens board-size)
  (define (queen-cols k)  
    (if (= k 0)
        (list empty-board)
        (filter
         (lambda (positions) (safe? k positions))
         (flatmap
          (lambda (rest-of-queens)
            (map (lambda (new-row)
                   (adjoin-position new-row k rest-of-queens))
                 (enumerate-interval 1 board-size)))
          (queen-cols (- k 1))))))
  (queen-cols board-size))
\end{schemedisplay}
In this procedure \scheme|rest-of-queens| is a way to place $k - 1$
queens in the first $k - 1$ columns, and \scheme|new-row| is a
proposed row in which to place the queen for the $k$th column.
Complete the program by implementing the representation for sets of
board positions, including the procedure \scheme|adjoin-position|,
which adjoins a new row-column position to a set of positions, and
\scheme|empty-board|, which represents an empty set of positions.  You
must also write the procedure \scheme|safe?|, which determines for a
set of positions, whether the queen in the $k$th column is safe with
respect to the others.  (Note that we need only check whether the new
queen is safe -- the other queens are already guaranteed safe with
respect to each other.)



\begin{Exercise}
\label{exc:2.43}
Louis Reasoner is having a terrible time doing exercise \ref{exc:2.42}.  His
\scheme|queens| procedure seems to work, but it runs extremely slowly.
(Louis never does manage to wait long enough for it to solve even the
$6 \times 6$ case.)  When Louis asks Eva Lu Ator for help, she points
out that he has interchanged the order of the nested mappings in the
\scheme|flatmap|, writing it as
\begin{schemedisplay}
(flatmap
 (lambda (new-row)
   (map (lambda (rest-of-queens)
          (adjoin-position new-row k rest-of-queens))
        (queen-cols (- k 1))))
 (enumerate-interval 1 board-size))
\end{schemedisplay}
Explain why this interchange makes the program run slowly.  Estimate
how long it will take Louis's program to solve the eight-queens
puzzle, assuming that the program in exercise \ref{exc:2.42} solves
the puzzle in time $T$.



\subsection{Example: A Picture Language}
\label{sec:2.2.4}


This section presents a simple language for drawing pictures that
illustrates the power of data abstraction and closure, and also
exploits higher-order procedures in an essential way.  The language is
designed to make it easy to experiment with patterns such as the ones
in figure \ref{fig:2.9}, which are composed of repeated elements that
are shifted and scaled.\footnote{22} In this language, the data
objects being combined are represented as procedures rather than as
list structure.  Just as \scheme|cons|, which satisfies the closure
property, allowed us to easily build arbitrarily complicated list
structure, the operations in this language, which also satisfy the
closure property, allow us to easily build arbitrarily complicated
patterns.

\begin{figure}
\placeholder{}
\caption{Designs generated with the picture language.}
\label{fig:2.9}
\end{figure}


\subsubsection*{The picture language}

When we began our study of programming in section \ref{sec:1.1}, we
emphasized the importance of describing a language by focusing on the
language's primitives, its means of combination, and its means of
abstraction.  We'll follow that framework here.

Part of the elegance of this picture language is that there is only
one kind of element, called a \textit{painter}.  A painter draws an
image that is shifted and scaled to fit within a designated
parallelogram-shaped frame.  For example, there's a primitive painter
we'll call \scheme|wave| that makes a crude line drawing, as shown in
figure \ref{fig:2.10}.  The actual shape of the drawing depends on the
frame -- all four images in figure \ref{fig:2.10} are produced by the
same \scheme|wave| painter, but with respect to four different frames.
Painters can be more elaborate than this: The primitive painter called
\scheme|rogers| paints a picture of MIT's founder, William Barton
Rogers, as shown in figure \ref{fig:2.11}.\footnote{23} The four
images in figure \ref{fig:2.11} are drawn with respect to the same
four frames as the \scheme|wave| images in figure \ref{fig:2.10}.


To combine images, we use various operations that construct new
painters from given painters.  For example, the \scheme|beside|
operation takes two painters and produces a new, compound painter that
draws the first painter's image in the left half of the frame and the
second painter's image in the right half of the frame.  Similarly,
\scheme|below| takes two painters and produces a compound painter that
draws the first painter's image below the second painter's image.
Some operations transform a single painter to produce a new painter.
For example, \scheme|flip-vert| takes a painter and produces a painter
that draws its image upside-down, and \scheme|flip-horiz| produces a
painter that draws the original painter's image left-to-right
reversed.

\begin{figure}
\centering
\placeholder
\caption{Images produced by the \texttt{wave} painter, with respect to
  four different frames.  The frames, shown with dotted lines, are not
  part of the images.}
\label{fig:2.10}
\end{figure}

\begin{figure}
  \centering
  \placeholder
  % TODO: get scheme working here
  \caption{Images of William Barton Rogers, founder and first
    president of MIT, painted with respect to the same four frames as
    in figure \ref{fig:2.10} (original image reprinted with the
    permission of the MIT Museum).}
  \label{fig:2.11}
\end{figure}


Figure \ref{fig:2.12} shows the drawing of a painter called
\scheme|wave4| that is built up in two stages starting from \scheme|wave|:
\begin{schemedisplay}
(define wave2 (beside wave (flip-vert wave)))
(define wave4 (below wave2 wave2))
\end{schemedisplay}

\begin{figure}
  \centering
\placeholder
  \begin{schemedisplay}
    (define wave2                         (define wave4
      (beside wave (flip-vert wave)))       (below wave2 wave2))
  \end{schemedisplay}
  % TODO: Get \scheme working here.
  \caption{Creating a complex figure, starting from the \texttt{wave} painter of figure \ref{fig:2.10}.}
  \label{fig:2.12}
\end{figure}

In building up a complex image in this manner we are exploiting the
fact that painters are closed under the language's means of
combination.  The \scheme|beside| or \scheme|below| of two painters is
itself a painter; therefore, we can use it as an element in making
more complex painters.  As with building up list structure using \scheme|cons|, the closure of our data under the means of combination is
crucial to the ability to create complex structures while using only a
few operations.


Once we can combine painters, we would like to be able to abstract
typical patterns of combining painters.
We will implement the painter operations as Scheme procedures.
This means that we don't need a special abstraction mechanism
in the picture language:
Since the means of combination
are ordinary Scheme procedures, we automatically have the capability
to do anything with painter operations that we can do with
procedures.
For example, we can abstract the pattern in \scheme|wave4| as


\begin{schemedisplay}
(define (flipped-pairs painter)
  (let ((painter2 (beside painter (flip-vert painter))))
    (below painter2 painter2)))
\end{schemedisplay}
and define \scheme|wave4| as an instance of this pattern:


\begin{schemedisplay}
(define wave4 (flipped-pairs wave))
\end{schemedisplay}

We can also define recursive operations.
Here's one that makes painters split and branch
towards the right as shown in figures \ref{fig:2.13}
and  \ref{fig:2.14}:
\begin{schemedisplay}
(define (right-split painter n)
  (if (= n 0)
      painter
      (let ((smaller (right-split painter (- n 1))))
        (beside painter (below smaller smaller)))))
\end{schemedisplay}

<a name="%_fig_2.13"></a><div align=left><table width=100%><tr><td><div align=left><img src="ch2-Z-G-36.gif" border="0">
                   
<img src="ch2-Z-G-37.gif" border="0"> </div>

\begin{schemedisplay}
     right-split \textit{n}                   corner-split \textit{n}
\end{schemedisplay}
</td></tr><caption align=bottom><div align=left><b>Figure 2.13:</b>  Recursive plans for \scheme|right-split| and \scheme|corner-split|.</div></caption><tr><td>

</td></tr></table></div> 

We can produce balanced patterns by branching upwards
as well as towards the right (see exercise \ref{exc:2.44}
and figures \ref{fig:2.13} and  \ref{fig:2.14}):

\begin{schemedisplay}
(define (corner-split painter n)
  (if (= n 0)
      painter
      (let ((up (up-split painter (- n 1)))
            (right (right-split painter (- n 1))))
        (let ((top-left (beside up up))
              (bottom-right (below right right))
              (corner (corner-split painter (- n 1))))
          (beside (below painter top-left)
                  (below bottom-right corner))))))
\end{schemedisplay}

<a name="%_fig_2.14"></a><div align=left><table width=100%><tr><td><div align=left><img src="ch2-Z-G-38.gif" border="0">
          
<img src="ch2-Z-G-39.gif" border="0"> </div>

\begin{schemedisplay}
     (right-split wave 4)         (right-split rogers 4)
\end{schemedisplay}
<div align=left><img src="ch2-Z-G-40.gif" border="0">
          
<img src="ch2-Z-G-41.gif" border="0"> </div>

\begin{schemedisplay}
    (corner-split wave 4)         (corner-split rogers 4)
\end{schemedisplay}
</td></tr><caption align=bottom><div align=left><b>Figure 2.14:</b>  The recursive operations \scheme|right-split| and \scheme|corner-split| applied to the painters \scheme|wave| and \scheme|rogers|.
Combining four \scheme|corner-split| figures produces
symmetric \scheme|square-limit| designs as shown
in figure \ref{fig:2.9}.</div></caption><tr><td>

</td></tr></table></div> 

By placing four copies of a \scheme|corner-split|
appropriately, we obtain a pattern called \scheme|square-limit|, whose
application to \scheme|wave| and \scheme|rogers| is shown in
figure \ref{fig:2.9}:
\begin{schemedisplay}
(define (square-limit painter n)
  (let ((quarter (corner-split painter n)))
    (let ((half (beside (flip-horiz quarter) quarter)))
      (below (flip-vert half) half))))
\end{schemedisplay}

\begin{Exercise}
\label{exc:2.44}
Define the procedure \scheme|up-split| used by \scheme|corner-split|.
It is similar to \scheme|right-split|, except that it switches the
roles of \scheme|below| and \scheme|beside|.





\subsubsection*{Higher-order operations}


In addition to abstracting patterns of combining painters, we can work
at a higher level, abstracting patterns of combining painter operations.
That is, we can view the painter operations as elements to manipulate
and can write means of combination for these elements -- procedures that
take painter operations as arguments and create new painter operations.

For example, \scheme|flipped-pairs| and \scheme|square-limit| each
arrange four copies of a painter's image in a square pattern; they differ
only in how they orient the copies.
One way to abstract this pattern of painter combination is with
the following procedure, which takes four one-argument painter operations
and produces a painter operation that transforms a given
painter with those four operations and arranges the results in a square.
\scheme|Tl|, \scheme|tr|, \scheme|bl|, and \scheme|br| are the
transformations to apply to the top left copy, the top right copy,
the bottom left copy, and the bottom right copy, respectively.

\begin{schemedisplay}
(define (square-of-four tl tr bl br)
  (lambda (painter)
    (let ((top (beside (tl painter) (tr painter)))
          (bottom (beside (bl painter) (br painter))))
      (below bottom top))))
\end{schemedisplay}
Then \scheme|flipped-pairs| can be defined in terms
of \scheme|square-of-four| as follows:\footnote{24}

\begin{schemedisplay}
(define (flipped-pairs painter)
  (let ((combine4 (square-of-four identity flip-vert
                                  identity flip-vert)))
    (combine4 painter)))
\end{schemedisplay}
and \scheme|square-limit| can be expressed as\footnote{25}

\begin{schemedisplay}
(define (square-limit painter n)
  (let ((combine4 (square-of-four flip-horiz identity
                                  rotate180 flip-vert)))
    (combine4 (corner-split painter n))))
\end{schemedisplay}

\begin{Exercise}
\label{exc:2.45}
\scheme|Right-split| and \scheme|up-split| can be expressed as
instances of a general splitting operation.
Define a procedure \scheme|split| with the property that evaluating
\begin{schemedisplay}
(define right-split (split beside below))
(define up-split (split below beside))
\end{schemedisplay}
produces procedures \scheme|right-split| and \scheme|up-split| with the same
behaviors as the ones already defined.




\subsubsection*{Frames}


Before we can show how to implement painters and their
means of combination, we must first consider
frames.  A frame can be described by three vectors -- an origin vector
and two edge vectors.  The origin vector specifies the offset of the
frame's origin from some absolute origin in the plane, and the edge
vectors specify the offsets of the frame's corners from its origin.
If the edges are perpendicular, the frame will be rectangular.
Otherwise the frame will be a more general parallelogram.

Figure \ref{fig:2.15} shows a frame and its associated vectors.  In
accordance with data abstraction, we need not be
specific yet about how frames are represented, other than to say that
there is a constructor \scheme|make-frame|, which takes three vectors and
produces a frame, and three corresponding selectors \scheme|origin-frame|, \scheme|edge1-frame|, and \scheme|edge2-frame| (see
exercise \ref{exc:2.47}).

<a name="%_fig_2.15"></a><div align=left><table width=100%><tr><td><img src="ch2-Z-G-42.gif" border="0">
</td></tr><caption align=bottom><div align=left><b>Figure 2.15:</b>  A frame is described by three vectors -- an
origin and two edges.</div></caption><tr><td>

</td></tr></table></div> 

We will use coordinates in the unit square (0<u><</u> \textit{x},\textit{y}<u><</u> 1)
to specify images.
With each frame, we associate a \textit{frame coordinate map}, which
will be used to shift and scale images to fit the frame.  The map
transforms the unit square into the frame by
mapping the vector <strong>\textit{v}</strong> = (\textit{x},\textit{y}) to the vector sum
<div align=left><img src="ch2-Z-G-43.gif" border="0"></div>
For example, (0,0) is mapped to the origin of the frame, (1,1) to
the vertex diagonally opposite the origin, and (0.5,0.5) to the
center of the frame.  We can create a frame's coordinate map with the
following procedure:\footnote{26}

\begin{schemedisplay}
(define (frame-coord-map frame)
  (lambda (v)
    (add-vect
     (origin-frame frame)
     (add-vect (scale-vect (xcor-vect v)
                           (edge1-frame frame))
               (scale-vect (ycor-vect v)
                           (edge2-frame frame))))))
\end{schemedisplay}
Observe that applying \scheme|frame-coord-map| to a frame returns
a procedure that, given a vector, returns a vector.
If the argument vector is in the unit square, the result vector
will be in the frame.  For example,
\begin{schemedisplay}
((frame-coord-map a-frame) (make-vect 0 0))
\end{schemedisplay}
returns the same vector as
\begin{schemedisplay}
(origin-frame a-frame)
\end{schemedisplay}

\begin{Exercise}
\label{exc:2.46}
A two-dimensional vector <strong>v</strong> running from the origin to a point
can be represented as a pair
consisting of an \textit{x}-coordinate and a \textit{y}-coordinate.  Implement a data
abstraction for vectors by giving a constructor \scheme|make-vect| and
corresponding selectors \scheme|xcor-vect| and \scheme|ycor-vect|.  In
terms of your selectors and constructor, implement procedures \scheme|add-vect|, \scheme|sub-vect|, and \scheme|scale-vect| that perform
the operations vector addition, vector subtraction, and multiplying a
vector by a scalar:
<div align=left><img src="ch2-Z-G-44.gif" border="0"></div>


\begin{Exercise}
\label{exc:2.47}
Here are two possible constructors for frames:
\begin{schemedisplay}
(define (make-frame origin edge1 edge2)
  (list origin edge1 edge2))

(define (make-frame origin edge1 edge2)
  (cons origin (cons edge1 edge2)))
\end{schemedisplay}
For each constructor supply the appropriate selectors to produce an
implementation for frames.




\subsubsection*{Painters}


A painter is represented as a procedure that, given a frame
as argument, draws a particular image shifted and scaled to fit the frame.
That is to say, if \scheme|p| is a painter and \scheme|f| is a frame, then we
produce \scheme|p|'s image in \scheme|f| by calling \scheme|p| with \scheme|f| as
argument.

The details of how primitive painters are implemented depend on the
particular characteristics of the graphics system and the type of
image to be drawn.  For instance, suppose we have a procedure \scheme|draw-line| that draws a line on the screen between two specified
points.  Then we can create painters for line drawings, such as the
\scheme|wave| painter in figure \ref{fig:2.10}, from lists of line
segments as follows:\footnote{27}\begin{schemedisplay}
(define (segments->painter segment-list)
  (lambda (frame)
    (for-each
     (lambda (segment)
       (draw-line
        ((frame-coord-map frame) (start-segment segment))
        ((frame-coord-map frame) (end-segment segment))))
     segment-list)))
\end{schemedisplay}
The segments are given using coordinates with respect to the unit
square.  For each segment in the list, the painter transforms the
segment endpoints with the frame coordinate map and draws a line
between the transformed points.

Representing painters as procedures erects a powerful abstraction
barrier in the picture language.  We can create and intermix
all sorts of primitive painters, based on a variety of graphics
capabilities. The details of their implementation do not matter.  Any
procedure can serve as a painter, provided that it takes a frame as
argument and draws something scaled to fit the frame.\footnote{28}

\begin{Exercise}
\label{exc:2.48}
A directed line segment in the
plane can be represented as a pair of vectors -- the
vector running from the origin to the start-point of the segment, and
the vector running from the origin to the end-point of the segment.
Use your vector representation from exercise \ref{exc:2.46} to
define a representation for segments with a constructor \scheme|make-segment| and selectors \scheme|start-segment| and \scheme|end-segment|.



\begin{Exercise}
\label{exc:2.49}
Use \scheme|segments->painter| to define the following primitive painters:

a.  The painter that draws the outline of the designated frame.

b.  The painter that draws an ``X'' by connecting opposite corners of
the frame.

c.  The painter that draws a diamond shape by connecting the midpoints of
the sides of the frame.

d.  The \scheme|wave| painter.




\subsubsection*{Transforming and combining painters}


An operation on painters (such as \scheme|flip-vert| or \scheme|beside|)
works by creating a painter that invokes the original painters
with respect to frames derived from the argument frame.
Thus, for example, \scheme|flip-vert| doesn't have to know how a painter
works in order to flip it -- it just has to know how to turn a frame
upside down:
The flipped painter just uses the original painter,
but in the inverted frame.

Painter operations are based on
the procedure \scheme|transform-painter|, which takes as arguments a painter and
information on how to transform a frame and
produces a new painter.  The transformed painter, when called on a frame,
transforms the frame and
calls the original painter on the transformed frame.
The arguments to \scheme|transform-painter| are points (represented as vectors)
that specify the corners of the new frame:
When mapped into
the frame, the first point specifies the new frame's origin
and the other two specify the ends of its edge vectors.
Thus, arguments within the
unit square specify a frame contained within the original frame.

\begin{schemedisplay}
(define (transform-painter painter origin corner1 corner2)
  (lambda (frame)
    (let ((m (frame-coord-map frame)))
      (let ((new-origin (m origin)))
        (painter
         (make-frame new-origin
                     (sub-vect (m corner1) new-origin)
                     (sub-vect (m corner2) new-origin)))))))
\end{schemedisplay}

Here's how to flip painter images vertically:
\begin{schemedisplay}
(define (flip-vert painter)
  (transform-painter painter
                     (make-vect 0.0 1.0)   \textit{; new \scheme|origin|}
                     (make-vect 1.0 1.0)   \textit{; new end of \scheme|edge1|}
                     (make-vect 0.0 0.0))) \textit{; new end of \scheme|edge2|}
\end{schemedisplay}
Using \scheme|transform-painter|, we can easily define new transformations.
For example, we can define a painter that shrinks its image to the
upper-right quarter of the frame it is given:
\begin{schemedisplay}
(define (shrink-to-upper-right painter)
  (transform-painter painter
                     (make-vect 0.5 0.5)
                     (make-vect 1.0 0.5)
                     (make-vect 0.5 1.0)))
\end{schemedisplay}
Other transformations rotate images counterclockwise by 90 degrees\footnote{29}
\begin{schemedisplay}
(define (rotate90 painter)
  (transform-painter painter
                     (make-vect 1.0 0.0)
                     (make-vect 1.0 1.0)
                     (make-vect 0.0 0.0)))
\end{schemedisplay}
or squash images towards the center of the frame:\footnote{30}
\begin{schemedisplay}
(define (squash-inwards painter)
  (transform-painter painter
                     (make-vect 0.0 0.0)
                     (make-vect 0.65 0.35)
                     (make-vect 0.35 0.65)))
\end{schemedisplay}

Frame transformation is also the key to
defining means of combining two or more painters.
The \scheme|beside| procedure,
for example, takes two painters, transforms them
to paint in the left and right halves of an argument frame respectively,
and produces a new, compound painter.
When the compound painter is given a frame, it
calls the first transformed painter to paint in the left half of
the frame and calls the second transformed painter to paint in the
right half of the frame:
\begin{schemedisplay}
(define (beside painter1 painter2)
  (let ((split-point (make-vect 0.5 0.0)))
    (let ((paint-left
           (transform-painter painter1
                              (make-vect 0.0 0.0)
                              split-point
                              (make-vect 0.0 1.0)))
          (paint-right
           (transform-painter painter2
                              split-point
                              (make-vect 1.0 0.0)
                              (make-vect 0.5 1.0))))
      (lambda (frame)
        (paint-left frame)
        (paint-right frame)))))
\end{schemedisplay}

Observe how the painter data abstraction, and in particular the
representation of painters as procedures, makes \scheme|beside| easy to
implement.  The \scheme|beside| procedure need not know anything
about the details of the component painters other than that each
painter will draw something in its designated frame.

\begin{Exercise}
\label{exc:2.50}
Define the transformation \scheme|flip-horiz|, which flips
painters horizontally, and transformations that rotate
painters counterclockwise by 180 degrees and 270 degrees.



\begin{Exercise}
\label{exc:2.51}
Define the \scheme|below| operation for painters.  \scheme|Below| takes two
painters as arguments.  The resulting painter, given a frame,
draws with the first painter in the
bottom of the frame and with the second painter in the top.  Define \scheme|below| in two different ways -- first by writing a procedure that is
analogous to the \scheme|beside| procedure given above, and
again in terms of \scheme|beside| and suitable
rotation operations (from exercise \ref{exc:2.50}).




\subsubsection*{Levels of language for robust design}

The picture language exercises some of the critical ideas
we've introduced about abstraction with procedures and data.  The
fundamental data abstractions, painters, are implemented using
procedural representations, which enables the language to
handle different basic drawing capabilities in a uniform way.  The
means of combination satisfy the closure property, which permits us to
easily build up complex designs.  Finally, all the tools for
abstracting procedures are available to us for abstracting means of
combination for painters.


We have also obtained a glimpse of another crucial idea about
languages and program design.  This is the approach of <em>stratified
design</em>, the notion that a complex system should be structured as a
sequence of levels that are described using a sequence of languages.
Each level is constructed by combining parts that are regarded as
primitive at that level, and the parts constructed at each level are
used as primitives at the next level.  The language used at each level
of a stratified design has primitives, means of combination, and means
of abstraction appropriate to that level of detail.

Stratified design pervades the engineering of complex systems.  For
example, in computer engineering, resistors and transistors are
combined (and described using a language of analog circuits) to
produce parts such as and-gates and or-gates, which form the
primitives of a language for digital-circuit design.\footnote{31}
These parts are combined to build
processors, bus structures, and memory systems, which are in turn
combined to form computers, using languages appropriate to computer
architecture.  Computers are combined to form distributed systems,
using languages appropriate for describing network interconnections,
and so on.

As a tiny example of stratification, our picture language uses
primitive elements (primitive painters) that are created using a
language that specifies points and lines to provide the lists of line
segments for \scheme|segments->painter|, or the
shading details for a painter like \scheme|rogers|.  The bulk of our
description of the picture language focused on combining these
primitives, using geometric combiners such as \scheme|beside| and \scheme|below|.  We also worked at a higher level, regarding \scheme|beside| and
\scheme|below| as primitives to be manipulated in a language whose
operations, such as \scheme|square-of-four|, capture common patterns of
combining geometric combiners.

Stratified design helps make programs \textit{robust}, that is, it makes
it likely that small changes in a specification will require
correspondingly small changes in the program.  For instance, suppose
we wanted to change the image based on \scheme|wave| shown in
figure \ref{fig:2.9}.  We could work at the lowest level
to change the detailed appearance of the \scheme|wave| element; we could
work at the middle level to change the way \scheme|corner-split|
replicates the \scheme|wave|; we could work at the highest level to
change how \scheme|square-limit| arranges the four copies of the corner.
In general, each level of a stratified design provides a different
vocabulary for expressing the characteristics of the system, and a
different kind of ability to change it.

\begin{Exercise}
\label{exc:2.52}
Make changes to the square limit of \scheme|wave| shown in
figure \ref{fig:2.9} by working at each of the levels
described above.  In particular:

a.  Add some segments to the primitive \scheme|wave| painter
of exercise  \ref{exc:2.49} (to add a smile, for example).

b.  Change the pattern constructed by \scheme|corner-split|
(for example, by using only one copy of the
\scheme|up-split| and \scheme|right-split| images instead of two).

c.  Modify the version of \scheme|square-limit| that uses \scheme|square-of-four|
so as to assemble the corners in a different pattern.  (For example, you
might make the big Mr. Rogers look outward from each corner of the square.)


<div class=smallprint><hr></div>
% Footnote 6
The use of the word ``closure'' here comes from abstract algebra,
where a set of elements is said to be closed under an operation if
applying the operation to elements in the set produces an element that
is again an element of the set.  The Lisp community
also (unfortunately) uses the word ``closure'' to describe a totally unrelated
concept: A closure is an implementation technique for representing
procedures with free variables.  We do not use the word ``closure'' in
this second sense in this book.

% Footnote 7
The notion that a means of
combination should satisfy closure is a straightforward idea.
Unfortunately, the data combiners provided in many popular programming
languages do not satisfy closure, or make closure cumbersome to
exploit.  In Fortran or Basic, one typically combines data elements by
assembling them into arrays -- but one cannot form arrays whose
elements are themselves arrays.  Pascal and C admit structures whose
elements are structures.  However, this requires that the programmer
manipulate pointers explicitly, and adhere to the restriction that
each field of a structure can contain only elements of a prespecified form.
Unlike
Lisp with its pairs, these languages have no built-in general-purpose
glue that makes it easy to manipulate compound data in a uniform way.
This limitation lies behind Alan Perlis's comment in his foreword to
this book: ``In Pascal the plethora of declarable data structures
induces a specialization within functions that inhibits and penalizes
casual cooperation.  It is better to have 100 functions operate on one
data structure than to have 10 functions operate on 10 data
structures.''

% Footnote 8
In this book, we use \textit{list} to mean a chain of
pairs terminated by the end-of-list marker.  In contrast, the term
\textit{list structure} refers to any data structure made out of pairs,
not just to lists.

% Footnote 9
Since nested applications of \scheme|car| and \scheme|cdr|
are cumbersome to write, Lisp dialects provide abbreviations for
them -- for instance,
<div align=left><img src="ch2-Z-G-14.gif" border="0"></div>
The names of all such procedures start with \scheme|c| and end with \scheme|r|.  Each \scheme|a| between them stands for a \scheme|car| operation and
each \scheme|d| for a \scheme|cdr| operation, to be applied in the same order
in which they appear in the name.  The names \scheme|car| and \scheme|cdr|
persist because simple combinations like \scheme|cadr| are
pronounceable.

% Footnote 10
It's remarkable how much energy in the
standardization of Lisp dialects has been dissipated in arguments that
are literally over nothing: Should \scheme|nil| be an ordinary name?
Should the value of \scheme|nil| be a symbol?  Should it be a list?
Should it be a pair?  In Scheme, \scheme|nil| is an ordinary name,
which we use in this section as a variable whose value is
the end-of-list marker (just as \scheme|true| is an ordinary variable
that has a true value).  Other dialects of
Lisp, including Common Lisp, treat \scheme|nil| as a special symbol.  The
authors of this book, who have endured too many language
standardization brawls, would like to avoid the entire issue.  Once we
have introduced quotation in section \ref{sec:2.3}, we will
denote the empty list as \scheme|'()| and dispense with the
variable \scheme|nil| entirely.

% Footnote 11
To define \scheme|f| and \scheme|g| using
\scheme|lambda| we would write
\begin{schemedisplay}
(define f (lambda (x y . z) <\textit{body}>))
(define g (lambda w <\textit{body}>))
\end{schemedisplay}

% Footnote 12
Scheme
standardly provides a \scheme|map| procedure that is more general
than the one described here.
This more general \scheme|map|
takes a procedure of \textit{n} arguments, together with \textit{n} lists, and
applies the procedure to all the first elements of
the lists, all the second elements of the lists, and so on,
returning a list of the results.  For example:
\begin{schemedisplay}
(map + (list 1 2 3) (list 40 50 60) (list 700 800 900))
<i>(741 852 963)</i>

(map (lambda (x y) (+ x (* 2 y)))
     (list 1 2 3)
     (list 4 5 6))
<i>(9 12 15)</i>
\end{schemedisplay}



% Footnote 13
The order of the
first two clauses in the \scheme|cond| matters, since the empty list
satisfies \scheme|null?| and also is not a pair.

% Footnote 14
This is, in fact, precisely the \scheme|fringe| procedure from
exercise \ref{exc:2.28}.  Here we've renamed it to emphasize that
it is part of a family of general sequence-manipulation procedures.

% Footnote 15
Richard Waters (1979)
developed a program that automatically analyzes traditional Fortran
programs, viewing them in terms of maps, filters, and accumulations.
He found that fully 90 percent of the code in the Fortran Scientific
Subroutine Package fits neatly into this paradigm.  One of the reasons
for the success of Lisp as a programming language is that lists
provide a standard medium for expressing ordered collections so that
they can be manipulated using higher-order operations.  The
programming language APL owes much of its power and appeal to a
similar choice. In APL all data are represented as arrays, and there is a
universal and convenient set of generic operators for all sorts of
array operations.

% Footnote 16
According to Knuth (1981), this rule was formulated by
W. G. Horner early in the nineteenth century, but the method was
actually used by Newton over a hundred years earlier.  Horner's rule
evaluates the polynomial using fewer additions and multiplications
than does the straightforward method of first computing \textit{a}<sub>\textit{n}</sub> \textit{x}<sup>\textit{n}</sup>,
then adding \textit{a}<sub>\textit{n}-1</sub>\textit{x}<sup>\textit{n}-1</sup>, and so on.  In fact, it is possible to
prove that any algorithm for evaluating arbitrary polynomials must use
at least as many additions and multiplications as does Horner's rule,
and thus Horner's rule is an optimal algorithm for polynomial
evaluation.  This was proved (for the number of additions) by
A. M. Ostrowski in a 1954 paper that essentially founded the modern
study of optimal algorithms.  The analogous statement for
multiplications was proved by V. Y. Pan in 1966.  The book by Borodin
and Munro (1975) provides an overview of these and other results about
optimal algorithms.

% Footnote 17
This definition uses the
extended version of \scheme|map| described in footnote <a href="#footnote_Temp_166">12</a>.

% Footnote 18
This approach to nested mappings was shown
to us by David Turner, whose languages KRC and Miranda provide elegant
formalisms for dealing with these constructs.  The examples in this
section (see also exercise \ref{exc:2.42}) are adapted from Turner
1981.  In section \ref{sec:3.5.3}, we'll see how this
approach generalizes to infinite sequences.

% Footnote 19
We're
representing a pair here as a list of two elements rather than as a
Lisp pair.  Thus, the ``pair'' (\textit{i},\textit{j}) is represented as \begin{schemedisplay}
(list i
j)\end{schemedisplay}, not \scheme|(cons i j)|.

% Footnote 20
The set \textit{S} - \textit{x} is the set of all elements
of \textit{S}, excluding \textit{x}.

% Footnote 21
Semicolons in Scheme code are used to
introduce \textit{comments}.  Everything from the semicolon to the end of
the line is ignored by the interpreter.  In this book we don't use
many comments; we try to make our programs self-documenting by using
descriptive names.

% Footnote 22
The picture language is based on the language
Peter Henderson created to construct
images like M.C. Escher's ``Square Limit'' woodcut (see Henderson 1982).
The woodcut incorporates a
repeated scaled pattern, similar to the arrangements drawn using
the \scheme|square-limit| procedure in this section.

% Footnote 23
William Barton Rogers (1804-1882) was the founder and first president
of MIT.  A geologist and talented teacher, he taught at William and
Mary College and at the University of Virginia.  In 1859 he moved to
Boston, where he had more time for research, worked on a plan
for establishing a ``polytechnic institute,'' and served as
Massachusetts's first State Inspector of Gas Meters.

When MIT was established in 1861, Rogers was elected its first
president.  Rogers espoused an ideal of ``useful learning'' that was
different from the university education of the time, with its
overemphasis on the classics, which, as he wrote, ``stand in the way of
the broader, higher and more practical instruction and discipline of
the natural and social sciences.''  This education was likewise to be
different from narrow trade-school education.  In Rogers's words:
<blockquote>
The world-enforced distinction between the practical and the
scientific worker is utterly futile, and the whole experience of
modern times has demonstrated its utter worthlessness.
</blockquote>

Rogers served as president of MIT until 1870, when he resigned due to
ill health.  In 1878 the second president of MIT, John Runkle,
resigned under the pressure of a financial crisis brought on by the
Panic of 1873 and strain of fighting off attempts by Harvard to take
over MIT.  Rogers returned to hold the office of president until
1881.

Rogers collapsed and died while addressing MIT's graduating class at
the commencement exercises of 1882.  Runkle quoted Rogers's last
words in a memorial address delivered that same year:
<blockquote>
``As I stand here today and see what the Institute is, \scheme|...| I call
to mind the beginnings of science.  I remember one hundred and fifty
years ago Stephen Hales published a pamphlet on the subject of
illuminating gas, in which he stated that his researches had
demonstrated that 128 grains of bituminous coal -- ''
``Bituminous coal,'' these were his last words on earth.  Here he bent
forward, as if consulting some notes on the table before him, then
slowly regaining an erect position, threw up his hands, and was
translated from the scene of his earthly labors and triumphs to ``the
tomorrow of death,'' where the mysteries of life are solved, and the
disembodied spirit finds unending satisfaction in contemplating the
new and still unfathomable mysteries of the infinite future.
</blockquote>
In the words of  Francis A. Walker
(MIT's third president):
<blockquote>
All his life he had borne himself most faithfully and heroically, and
he died as so good a knight would surely have wished, in harness, at
his post, and in the very part and act of public duty.
</blockquote>

% Footnote 24
Equivalently, we could
write
\begin{schemedisplay}
(define flipped-pairs
  (square-of-four identity flip-vert identity flip-vert))
\end{schemedisplay}


% Footnote 25
\scheme|Rotate180|
rotates a painter by 180 degrees (see exercise \ref{exc:2.50}).
Instead of \scheme|rotate180| we could say \scheme|(compose flip-vert flip-horiz)|, using
the \scheme|compose| procedure from exercise \ref{exc:1.42}.

% Footnote 26
\scheme|Frame-coord-map| uses
the vector operations described in exercise \ref{exc:2.46} below, which we
assume have been implemented using some representation for vectors.
Because of data abstraction, it doesn't matter what this vector
representation is, so long as the vector operations behave correctly.

% Footnote 27
\scheme|Segments->painter| uses the representation for line
segments described in exercise \ref{exc:2.48} below.
It also uses the \scheme|for-each| procedure described in exercise \ref{exc:2.23}.

% Footnote 28
For example, the \scheme|rogers| painter of
figure \ref{fig:2.11} was constructed from a gray-level image.
For each point in a given frame,
the \scheme|rogers| painter determines the point in the image that is mapped to it
under the frame coordinate map, and shades it
accordingly.  By allowing different types of painters, we are capitalizing on the
abstract data idea discussed in section \ref{sec:2.1.3}, where we
argued that a rational-number representation could be anything at all that
satisfies an appropriate condition.  Here we're using the fact that a
painter can be implemented in any way at all, so long as it draws
something in the designated frame.  Section \ref{sec:2.1.3} also
showed how pairs could be implemented as procedures.  Painters are our
second example of a procedural representation for data.

% Footnote 29
\scheme|Rotate90| is a pure rotation only for square
frames, because it also stretches and shrinks the image to fit into
the rotated frame.

% Footnote 30
The diamond-shaped images in figures \ref{fig:2.10}
and \ref{fig:2.11} were created with \scheme|squash-inwards| applied to
\scheme|wave| and \scheme|rogers|.

% Footnote 31
Section \ref{sec:3.3.4} describes one such language.

</div>


\end{document}
