% -*- TeX-master: "sicp.tex" -*-

\chapter{Building Abstractions with Data}
\label{chap:2}

\epigraph{
We now come to the decisive step of mathematical abstraction: we
forget about what the symbols stand for. \scheme|...|[The mathematician]
need not be idle; there are many operations which he may carry out
with these symbols, without ever having to look at the things they
stand for.}{Hermann Weyl, \textit{The Mathematical Way of Thinking}}

We concentrated in chapter 1 on computational processes and on the
role of procedures in program design.  We saw how to use primitive
data (numbers) and primitive operations (arithmetic operations), how
to combine procedures to form compound procedures through composition,
conditionals, and the use of parameters, and how to abstract
procedures by using \scheme|define|.  We saw that a procedure can be
regarded as a pattern for the local evolution of a process, and we
classified, reasoned about, and performed simple algorithmic analyses
of some common patterns for processes as embodied in procedures.  We
also saw that higher-order procedures enhance the power of our
language by enabling us to manipulate, and thereby to reason in terms
of, general methods of computation.  This is much of the essence of
programming.

In this chapter we are going to look at more complex data.  All the
procedures in chapter 1 operate on simple numerical data, and simple
data are not sufficient for many of the problems we wish to address
using computation.  Programs are typically designed to model complex
phenomena, and more often than not one must construct computational
objects that have several parts in order to model real-world phenomena
that have several aspects.  Thus, whereas our focus in chapter 1 was
on building abstractions by combining procedures to form compound
procedures, we turn in this chapter to another key aspect of any
programming language: the means it provides for building abstractions
by combining data objects to form \textit{compound data}.

Why do we want compound data in a programming language?  For the same
reasons that we want compound procedures: to elevate the conceptual
level at which we can design our programs, to increase the modularity
of our designs, and to enhance the expressive power of our language.
Just as the ability to define procedures enables us to deal with
processes at a higher conceptual level than that of the primitive
operations of the language, the ability to construct compound data
objects enables us to deal with data at a higher conceptual level than
that of the primitive data objects of the language.

Consider the task of designing a system to perform arithmetic with
rational numbers.  We could imagine an operation \scheme|add-rat| that
takes two rational numbers and produces their sum.  In terms of simple
data, a rational number can be thought of as two integers: a numerator
and a denominator.  Thus, we could design a program in which each
rational number would be represented by two integers (a numerator and
a denominator) and where \scheme|add-rat| would be implemented by two
procedures (one producing the numerator of the sum and one producing
the denominator).  But this would be awkward, because we would then
need to explicitly keep track of which numerators corresponded to
which denominators.  In a system intended to perform many operations
on many rational numbers, such bookkeeping details would clutter the
programs substantially, to say nothing of what they would do to our
minds.  It would be much better if we could ``glue together'' a
numerator and denominator to form a pair -- a \textit{compound data
  object} -- that our programs could manipulate in a way that would be
consistent with regarding a rational number as a single conceptual
unit.

The use of compound data also enables us to increase the modularity of
our programs.  If we can manipulate rational numbers directly as
objects in their own right, then we can separate the part of our
program that deals with rational numbers per se from the details of
how rational numbers may be represented as pairs of integers.  The
general technique of isolating the parts of a program that deal with
how data objects are represented from the parts of a program that deal
with how data objects are used is a powerful design methodology called
\textit{data abstraction}.  We will see how data abstraction makes
programs much easier to design, maintain, and modify.

The use of compound data leads to a real increase in the expressive
power of our programming language.  Consider the idea of forming a
``linear combination'' $ax + by$.  We might like to write a procedure
that would accept $a$, $b$, $x$, and $y$ as arguments and return the
value of $ax + by$.  This presents no difficulty if the arguments are
to be numbers, because we can readily define the procedure

\begin{schemedisplay}
(define (linear-combination a b x y) 
  (+ (* a x) (* b y)))
\end{schemedisplay}
But suppose we are not concerned only with numbers.  Suppose we would
like to express, in procedural terms, the idea that one can form
linear combinations whenever addition and multiplication are
defined -- for rational numbers, complex numbers, polynomials, or
whatever.  We could express this as a procedure of the form

\begin{schemedisplay}
(define (linear-combination a b x y)     
  (add (mul a x) (mul b y))) 
\end{schemedisplay}
where \scheme|add| and \scheme|mul| are not the primitive procedures
\scheme|+| and \scheme|*| but rather more complex things that will
perform the appropriate operations for whatever kinds of data we pass
in as the arguments \scheme|a|, \scheme|b|, \scheme|x|, and
\scheme|y|. The key point is that the only thing
\scheme|linear-combination| should need to know about \scheme|a|,
\scheme|b|, \scheme|x|, and \scheme|y| is that the procedures
\scheme|add| and \scheme|mul| will perform the appropriate
manipulations.  From the perspective of the procedure
\scheme|linear-combination|, it is irrelevant what \scheme|a|,
\scheme|b|, \scheme|x|, and \scheme|y| are and even more irrelevant
how they might happen to be represented in terms of more primitive
data.  This same example shows why it is important that our
programming language provide the ability to manipulate compound
objects directly: Without this, there is no way for a procedure such
as \scheme|linear-combination| to pass its arguments along to
\scheme|add| and \scheme|mul| without having to know their detailed
structure.\footnote{The ability to directly manipulate procedures
  provides an analogous increase in the expressive power of a
  programming language.  For example, in section \ref{sec:1.3.1} we
  introduced the \scheme|sum| procedure, which takes a procedure
  \scheme|term| as an argument and computes the sum of the values of
  \scheme|term| over some specified interval.  In order to define
  \scheme|sum|, it is crucial that we be able to speak of a procedure
  such as \scheme|term| as an entity in its own right, without regard
  for how \scheme|term| might be expressed with more primitive
  operations.  Indeed, if we did not have the notion of ``a
  procedure,'' it is doubtful that we would ever even think of the
  possibility of defining an operation such as \scheme|sum|.
  Moreover, insofar as performing the summation is concerned, the
  details of how \scheme|term| may be constructed from more primitive
  operations are irrelevant.  }

We begin this chapter by implementing the rational-number arithmetic
system mentioned above.  This will form the background for our
discussion of compound data and data abstraction.  As with compound
procedures, the main issue to be addressed is that of abstraction as a
technique for coping with complexity, and we will see how data
abstraction enables us to erect suitable \textit{abstraction barriers}
between different parts of a program.

We will see that the key to forming compound data is that a
programming language should provide some kind of ``glue'' so that data
objects can be combined to form more complex data objects.  There are
many possible kinds of glue.  Indeed, we will discover how to form
compound data using no special ``data'' operations at all, only
procedures.  This will further blur the distinction between
``procedure'' and ``data,'' which was already becoming tenuous toward
the end of chapter 1.  We will also explore some conventional
techniques for representing sequences and trees.  One key idea in
dealing with compound data is the notion of \textit{closure} -- that the
glue we use for combining data objects should allow us to combine not
only primitive data objects, but compound data objects as well.
Another key idea is that compound data objects can serve as \textit{conventional interfaces} for combining program modules in
mix-and-match ways.  We illustrate some of these ideas by presenting a
simple graphics language that exploits closure.

We will then augment the representational power of our language by
introducing \textit{symbolic expressions} -- data whose elementary parts
can be arbitrary symbols rather than only numbers.  We explore various
alternatives for representing sets of objects.  We will find that,
just as a given numerical function can be computed by many different
computational processes, there are many ways in which a given data
structure can be represented in terms of simpler objects, and the
choice of representation can have significant impact on the time and
space requirements of processes that manipulate the data.  We will
investigate these ideas in the context of symbolic differentiation,
the representation of sets, and the encoding of information.

Next we will take up the problem of working with data that may be
represented differently by different parts of a program.  This leads
to the need to implement \textit{generic operations}, which must handle
many different types of data.  Maintaining modularity in the
presence of generic operations requires more powerful abstraction
barriers than can be erected with simple data abstraction alone.  In
particular, we introduce \textit{data-directed programming} as a
technique that allows individual data representations to be designed
in isolation and then combined \textit{additively} (i.e., without
modification).  To illustrate the power of this approach to system
design, we close the chapter by applying what we have learned to the
implementation of a package for performing symbolic arithmetic on
polynomials, in which the coefficients of the polynomials can be
integers, rational numbers, complex numbers, and even other
polynomials.

